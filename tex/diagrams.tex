\documentclass[12pt,twoside,singlespace]{mitthesis}
%\usepackage{lgrind}
%% These have been added at the request of the MIT Libraries, because
%% some PDF conversions mess up the ligatures.  -LB, 1/22/2014
\usepackage{tikz-feynman}
\usepackage{hyperref}
\usepackage{graphicx}
\usepackage{xspace}
\usepackage{xcolor}
\usepackage{cmap}
\usepackage[T1]{fontenc}
\usepackage{multirow}
\usepackage{textcomp}
\usepackage{siunitx}
\usepackage{rotating}
\pagestyle{plain}
\definecolor{violet}{RGB}{161,0,201}
\definecolor{mitred}{RGB}{163, 31, 52}


\makeatletter
\def\hlinewd#1{%
	\noalign{\ifnum0=`} \fi \hrule \@height #1 \futurelet % 
   \reserved@a\@xhline}
\makeatother

\def\checkmark{\tikz\fill[scale=0.4](0,.35) -- (.25,0) -- (1,.7) -- (.25,.15) -- cycle;} 
\newcommand{\fixme}[1]{{\color{red}\sffamily{\bfseries{}FIXME:} #1}}
\newlength\cmsFigWidth
\setlength\cmsFigWidth{0.45\textwidth}

\newcommand{\eventDisplayCaption}{The charged particle trajectories are in indigo. The $\met$ is the magenta arrow. The electrons are in gold. Photons are shown as yellow light rays. The HCAL and ECAL deposits associated with PF candidates are the darker cobalt blue prisms and the lighter royal blue prisms, respectively; their length represents energy. The distant wireframe and blue boxes represent the muon systems.}

\newcommand{\termLB}{\textcolor{mitred}{\pmb{\bigg[}}}
\newcommand{\termRB}{\textcolor{mitred}{\pmb{\bigg]}}}
\newcommand{\dU}{\ensuremath{d_{\textsf{U}}}\xspace}
\newcommand{\LU}{\ensuremath{\Lambda_{\textsf{U}}}\xspace}
%\newcommand{\Z}{\mbox{Z}}
%\newcommand{\W}{\mbox{W}}
\newcommand{\Z}{\ensuremath{\mathrm{Z}}}
\newcommand{\W}{\ensuremath{\mathrm{W}}}
%\newcommand{\met}{\ensuremath{\mspace{3mu}/\mspace{-12.0mu}E_{T}}}
%\newcommand{\metvector}{\ensuremath{\mspace{3mu}/\mspace{-12.0mu}\vec{E}_{T}}}
\newcommand{\mc}{Monte Carlo}
\newcommand{\MC}{\ensuremath{\textrm{MC}}\xspace}
\newcommand{\SM}{\mbox{SM}}
\newcommand{\lumiunc}{2.5\%}
%\newcommand{\lumi}{1.XXX\fbinv}%{1048~\pbinv}}
\newcommand{\invpb}{\ensuremath{\textrm{pb}^{\scriptscriptstyle -1}}}
\newcommand{\invfb}{\ensuremath{\textrm{fb}^{\scriptscriptstyle -1}}}
%\newcommand{\mev}{\ensuremath{\,\textrm{MeV}}}
%\newcommand{\gev}{\ensuremath{\,\textrm{GeV}}}
\newcommand{\tev}{\ensuremath{\,\textrm{TeV}}}
\newcommand{\Red}{\color{red}}
\newcommand{\Zmumu}{\ensuremath{Z\rightarrow \mu\mu}}
\newcommand{\Znunu}{\ensuremath{Z\rightarrow \nu\bar{\nu}}}
\newcommand{\Zboson}{\ensuremath{Z^0}}
\newcommand{\Zee}{\ensuremath{Z\rightarrow ee}}
\newcommand{\Zll}{\ensuremath{Z\rightarrow \ell\ell}\mbox{ with }\ensuremath{\ell=e,\mu}\xspace}
\newcommand{\rapidity}{\ensuremath{y}}
\newcommand{\pseudorapidity}{\ensuremath{\eta}}
\newcommand{\lepton}{\ensuremath{\ell}}
\newcommand{\TL}{{\bf Top-Left}}
\newcommand{\TR}{{\bf Top-Right}}
\newcommand{\BL}{{\bf Bottom-Left}}
\newcommand{\BR}{{\bf Bottom-Right}}
%\newcommand{\CL}{{\bf Center-Left}}
\newcommand{\CR}{{\bf Center-Right}}
\newcommand{\T}{{\bf Top}}
\newcommand{\B}{{\bf Bottom}}
\newcommand{\Right}{{\bf Right}}
\newcommand{\Left}{{\bf Left}}
\newcommand{\Center}{{\bf Center}}
\newcommand{\pythia}{{\tt \scshape pythia8}}
\newcommand{\geant}{{\tt \scshape geant}}
\newcommand{\madgraph}{\mbox{\tt \scshape MadGraph5}}
\newcommand{\powheg}{{\tt \scshape powheg}}
\newcommand{\sherpa}{{\tt \scshape sherpa}}
\newcommand{\herwigpp}{{\tt \scshape herwig++}}
\newcommand{\blackhat}{{\tt \scshape BlackHat}}
\newcommand{\amcatnlo}{\mbox{\tt \scshape MadGraph5\_aMC@NLO}}
\newcommand{\fewz}{\mbox{\tt FEWZ}}
\newcommand{\CMS}{{\tt CMS}\xspace}
\newcommand{\QCD}{{\tt QCD}\xspace}
\newcommand{\ptZ}{\ensuremath{p_T^{Z}}}
\newcommand{\ptG}{\ensuremath{p_T^{\gamma}}}
\newcommand{\phiso}{\ensuremath{PFIso_\textrm{photon}}}
\newcommand{\njets}{\ensuremath{n_\textrm{jets}}}
\newcommand{\phiStar}{\ensuremath{\phi^{\scriptscriptstyle *}_\eta}}
\newcommand{\RooUnfold}{{\scshape RooUnfold}}

%%% shortcut
\newcolumntype{d}[1]{D{.}{.}{#1}}

\newcommand{\tw}{\ensuremath{\mathrm{tW}}}
\newcommand{\dyee}{\ensuremath{Z/\gamma^*\to e^+e^-}}
\newcommand{\dymm}{\ensuremath{Z/\gamma^*\to\mu^+\mu^-}}
\newcommand{\dytt}{\ensuremath{Z/\gamma^*\to\tau^+\tau^-}}
\newcommand{\dyll}{\ensuremath{Z/\gamma^*\to\ell^+\ell^-}}
\newcommand{\WW}{\ensuremath{\W^+\W^-}}
\newcommand{\Lep}{\ensuremath{\ell}}
\newcommand{\mll}{\ensuremath{m_{\Lep\Lep}}}
\newcommand{\CLb}{\ensuremath{CL_\mathrm{b}}}

\newcommand{\nanob}{\mbox{{\rm ~nb}~}}
\newcommand{\fb}{\ensuremath{\mathrm{fb}}}
\newcommand{\pb}{\ensuremath{\mathrm{pb}}}
\newcommand{\ifb}{\ensuremath{\mathrm{fb^{-1}}}}
\newcommand{\ipb}{\ensuremath{\mathrm{pb^{-1}}}}
\newcommand{\grad}{\ensuremath{^{\circ}}}
%
% Special user made math symbols
%
\newcommand{\lsim}{\raisebox{-1.5mm}{$\:\stackrel{\textstyle{<}}{\textstyle{\sim}}\:$}}
\newcommand{\gsim}{\raisebox{-1.5mm}{$\:\stackrel{\textstyle{>}}{\textstyle{\sim}}\:$}}

% particles

\newcommand{\pipm}{\ensuremath{\pi^{\pm}}}
\newcommand{\pizero}{\ensuremath{\pi^{0}}}
\newcommand{\Hi}{\ensuremath{\mathrm{H}}}
\newcommand{\V}{\ensuremath{\mathrm{V}}}
\newcommand{\Wjets}{\ensuremath{\mathrm{W+jets}}}
\newcommand{\Zjets}{\ensuremath{\mathrm{Z+jets}}}
\newcommand{\Wt}{\ensuremath{\mathrm{Wt}}}
\newcommand{\Wstar}{\ensuremath{\mathrm{W}^{*}}}
\newcommand{\Wparenthesisstar}{\ensuremath{\mathrm{W}^{(*)}}}
\newcommand{\Zstar}{\ensuremath{\mathrm{Z}^{*}}}
%\newcommand{\Wpm}{\ensuremath{\W^{\pm}}}
\newcommand{\ZZ}{\ensuremath{\Z\Z}}
\newcommand{\WZ}{\ensuremath{\W\Z}}
\newcommand{\El}{\ensuremath{\mathrm{e}}}
\newcommand{\Elp}{\ensuremath{\mathrm{e}^{+}}}
\newcommand{\Elm}{\ensuremath{\mathrm{e}^{-}}}
\newcommand{\Elpm}{\ensuremath{\mathrm{e}^{\pm}}}
\newcommand{\Elmp}{\ensuremath{\mathrm{e}^{\mp}}}
\newcommand{\M}{\ensuremath{\mu}}
\newcommand{\Mp}{\ensuremath{\mu^{+}}}
\newcommand{\Mm}{\ensuremath{\mu^{-}}}
\newcommand{\Mpm}{\ensuremath{\mu^{\pm}}}
\newcommand{\Mmp}{\ensuremath{\mu^{\mp}}}
\newcommand{\Tau}{\ensuremath{\tau}}
\newcommand{\Nu}{\ensuremath{\nu}}
\newcommand{\Nubar}{\ensuremath{\bar{\nu}}}
\newcommand{\Lepp}{\ensuremath{\ell^{+}}}
\newcommand{\Lepm}{\ensuremath{\ell^{-}}}
\newcommand{\Lprime}{\ensuremath{\Lep^{\prime}}}
\newcommand{\Prot}{\ensuremath{\mathrm{p}}}
\newcommand{\Pbar}{\ensuremath{\bar{\mathrm{p}}}}
\newcommand{\PP}{\Prot\Prot}
\newcommand{\PPbar}{\Prot\Pbar}
\newcommand{\qq}{\ensuremath{\mathrm{q}\mathrm{q}}}
%\newcommand{\bbbar}{\ensuremath{\mathrm{b}\bar{\mathrm{b}}}}
\newcommand{\Wtb}{\ensuremath{\W\mathrm{t}\mathrm{b}}}
\newcommand{\Top}{\ensuremath{\mathrm{t}}}
\newcommand{\Bot}{\ensuremath{\mathrm{b}}}
\newcommand{\Atop}{\ensuremath{\bar{\mathrm{t}}}}
\newcommand{\Abot}{\ensuremath{\bar{\mathrm{b}}}}
\newcommand{\WH}{\ensuremath{\W\Hi}}
\newcommand{\ZH}{\ensuremath{\Z\Hi}}
% arrow
\newcommand{\To}{\ensuremath{\rightarrow}}

% masses
\newcommand{\mHi}{\ensuremath{m_{\mathrm{H}}}}
\newcommand{\mW}{\ensuremath{m_{\mathrm{W}}}}
\newcommand{\mZ}{\ensuremath{m_{\mathrm{Z}}}}
\newcommand{\mt}{\ensuremath{m_{T}}}

% kinematics
\newcommand{\ptveto}{\ensuremath{\pt^\mathrm{veto}}}
\newcommand{\ptl}{\ensuremath{p_\perp^{\Lep}}}
\newcommand{\ptlmax}{\ensuremath{p_{\mathrm{T}}^{\Lep,\mathrm{max}}}}
\newcommand{\ptlmin}{\ensuremath{p_{\mathrm{T}}^{\Lep,\mathrm{min}}}}
\newcommand{\Et}{\ensuremath{E_\mathrm{T}}}
\newcommand{\met}{\ensuremath{\Et^{\mathrm{miss}}}}
\newcommand{\delphill}{\ensuremath{\Delta\phi_{\Lep\Lep}}}
\newcommand{\deletall}{\ensuremath{\Delta\eta_{\Lep\Lep}}}
\newcommand{\delphimetl}{\ensuremath{\Delta\phi_{\met\Lep}}}
\newcommand{\delR}{\ensuremath{\Delta R}}
\newcommand{\Eta}{\ensuremath{\eta}}
\newcommand{\mT}{\ensuremath{m_{\mathrm{T}}^{\ell\ell\met}}}
\newcommand{\vmet}{\ensuremath{\vec{E}_\mathrm{T}}^{\text{miss}}}
\newcommand{\vg}{\ensuremath{\vec{\gamma}_\mathrm{T}}}
\newcommand{\delphillmetg}{\ensuremath{\Delta\phi(\Lep\Lep,\vmet+\vg})}

\newcommand{\pfmet}{\ensuremath{E_\mathrm{T,PF}^{\mathrm{miss}}}}
%efficiencies
\newcommand{\effsig}{\ensuremath{\varepsilon_{\mathrm{bkg}}^{\mathrm{S}}}}
\newcommand{\effnorm}{\ensuremath{\varepsilon_{\mathrm{bkg}}^{\mathrm{N}}}}
\newcommand{\Nsig}{\ensuremath{N_{\mathrm{bkg}}^{\mathrm{S}}}}
\newcommand{\Nnorm}{\ensuremath{N_{\mathrm{bkg}}^{\mathrm{N}}}}

% processes
\newcommand{\zee}{\ensuremath{Z\to e^+e^-}}
\newcommand{\zmm}{\ensuremath{Z\to\mu^+\mu^-}}
\newcommand{\ztt}{\ensuremath{Z\to\tau^+\tau^-}}
\newcommand{\zll}{\ensuremath{Z\to\ell^+\ell^-}}
%\newcommand{\ttbar}{\ensuremath{t\bar{t}}}
\newcommand{\ppww}{\ensuremath{pp \to W^+W^-}}
\newcommand{\wwll}{\ensuremath{WW\to \ell^+\ell^-}}
\newcommand{\wwlnln}{\ensuremath{W^+W^-\to \ell^+\nu \ell^-\bar{\nu}}}
\newcommand{\hww}{\ensuremath{\Hi \to \WW}}
\newcommand{\wz}{\ensuremath{WZ}}
\newcommand{\zz}{\ensuremath{ZZ}}
\newcommand{\wgamma}{\ensuremath{W\gamma}}
\newcommand{\wjets}{\ensuremath{W+}jets}
\newcommand{\singletopt}{\ensuremath{t} ($t$-chan)}
\newcommand{\singletops}{\ensuremath{t} ($s$-chan)}

%other
\def\fixme{({\bf \color{red}FIXME})}
\newcommand{\ee}{\ensuremath{ee}}
\newcommand{\emu}{\ensuremath{e\mu}}

% integrated luminosity
\newcommand{\usedLumiWithSyst}{35.9~\pm~0.9~\ifb}
\newcommand{\usedLumi}{35.9~\ifb}

%%%%%%%%%%%%%%%%%%%%%%%%%%%%%%%%%%%%%%%%%%%%%%%%%%%%%%%%%%%%%%%%%%%%%%
%                                                                    %
%  This is pennames.sty                                              %
%                                                                    %
%  It contains the definition of the short names for the PEN         %
%  Elementary Particle Naming Scheme, described in CNL 203, pp 8-11  %
%                                                                    %
%  Version 1.0: Original version -  4 Oct 1991 (evh)                 %
%          1.1: \def,\relax\ifmmode instead of \mbox                 %
%               16 Oct 1991 (mg)                                     %   
%          1.2: Corrections for upsilon and psi - 21 Oct 1991 (evh)  %
%          1.3: Line lenghts < 80 charcaters - 22 Oct 1991 (mg)      %
%          1.4: Add definitions for NFSS (\mathrm instead of \rm)    %
%               27 May 1993 (mg)                                     %   
%          1.5: Add definitions \PaD, \PaDz, \PaB, \PaBz             %
%               \Pq, \Paq, \Pqd, \Paqd, \Pqu, \Paqu, \Pqs, \Paqs     %
%               \Pqc, \Paqc, \Pqb, \Paqb, \Pqt, \Paqt, PaP, PagL     %
%               \PagSm, \PagSp, \PagSz, \PagXz, \PagXp, \PagOp, \PaSq%
%               12 Jul 1993 (mg)                                     %   
%          1.6: Include \relax to force expansion of \if (DCa)       %
%               Add % at end of every command to eliminate possible  %
%               parasitic white space.                               %
%                6 Feb 1994 (mg)                                     %   
%          2.0: Adapt to LaTeX2e with \ensuremath command            %
%                30 Jan 1995 (mg)                                    %   
%          3.0: Make latex2e reference and define                    %
%               \newcommand and/or \ensuremath is undefined.         %
%                 3 Apr 1996 (mg)                                    %   
%                                                                    %
%  Authors: Michel Goossens and Eric van Herwijnen                   %
%           CERN, Geneva, Switzerland                                %
%                                                                    %
%  Last Mod.  3 Apr 1996 (mg)                                        %
%                                                                    %
%%%%%%%%%%%%%%%%%%%%%%%%%%%%%%%%%%%%%%%%%%%%%%%%%%%%%%%%%%%%%%%%%%%%%%
 \newcommand{\PAz}{\ensuremath{\mathrm{A^0}}}
 \newcommand{\PBm}{\ensuremath{{\mathrm{B}^{-}}}}
 \newcommand{\PBpm}{\ensuremath{{\mathrm{B}^{\pm}}}}
 \newcommand{\PBp}{\ensuremath{{\mathrm{B}^{+}}}}
 \newcommand{\PBz}{\ensuremath{{\mathrm{B}^0}}}
 \newcommand{\PB}{\ensuremath{{\mathrm{B}}}}
 \newcommand{\PDiz}{\ensuremath{{\mathrm{D}_{1}(2420)^0}}}
 \newcommand{\PDm}{\ensuremath{\mathrm{D^-}}}
 \newcommand{\PDpm}{\ensuremath{\mathrm{D^{\pm}}}}
 \newcommand{\PDp}{\ensuremath{\mathrm{D^+}}}
 \newcommand{\PDstiiz}{\ensuremath{{\mathrm{D}^{\ast}_{2}(2460)^0}}}
 \newcommand{\PDstpm}{\ensuremath{{\mathrm{D}^{\ast}(2010)^{\pm}}}}
 \newcommand{\PDstz}{\ensuremath{{\mathrm{D}^{\ast}(2010)^0}}}
 \newcommand{\PDz}{\ensuremath{\mathrm{D^0}}}
 \newcommand{\PD}{\ensuremath{\mathrm{D}}}
 \newcommand{\PEz}{\ensuremath{\mathrm{E^0}}}
 \newcommand{\PHpm}{\ensuremath{\mathrm{H^{\pm}}}}
 \newcommand{\PHz}{\ensuremath{\mathrm{H^0}}}
 \newcommand{\PJgy}{\ensuremath{\mathrm{J /\psi(1S)}}}
 \newcommand{\PKeiii}{\ensuremath{\mathrm{K_{e3}}}}
 \newcommand{\PKgmiii}{\ensuremath{\mathrm{K_{\mu 3}}}}
 \newcommand{\PKia}{\ensuremath{\mathrm{K_1(1400)}}}
 \newcommand{\PKii}{\ensuremath{\mathrm{K_2(1770)}}}
 \newcommand{\PKi}{\ensuremath{\mathrm{K_1(1270)}}}
 \newcommand{\PKm}{\ensuremath{\mathrm{K^-}}}
 \newcommand{\PKpm}{\ensuremath{\mathrm{K^{\pm}}}}
 \newcommand{\PKp}{\ensuremath{\mathrm{K^+}}}
 \newcommand{\PKsta}{\ensuremath{\mathrm{K^{\ast}(1370)}}}
 \newcommand{\PKstb}{\ensuremath{\mathrm{K^{\ast}(1680)}}}
 \newcommand{\PKstiii}{\ensuremath{\mathrm{K^{\ast}_3(1780)}}}
 \newcommand{\PKstii}{\ensuremath{\mathrm{K^{\ast}_2(1430)}}}
 \newcommand{\PKstiv}{\ensuremath{\mathrm{K^{\ast}_4(2045)}}}
 \newcommand{\PKstz}{\ensuremath{\mathrm{K^{\ast}_0(1430)}}}
 \newcommand{\PKst}{\ensuremath{\mathrm{K^{\ast}(892)}}}
 \newcommand{\PKzL}{\ensuremath{\mathrm{K^0_L}}}
 \newcommand{\PKzS}{\ensuremath{\mathrm{K^0_S}}}
 \newcommand{\PKzeiii}{\ensuremath{\mathrm{K^0_{e3}}}}
 \newcommand{\PKzgmiii}{\ensuremath{\mathrm{K^0_{\mu 3}}}}
 \newcommand{\PKz}{\ensuremath{\mathrm{K^0}}}
 \newcommand{\PK}{\ensuremath{\mathrm{K}}}
 \newcommand{\PLpm}{\ensuremath{\mathrm{L^{\pm}}}}
 \newcommand{\PLz}{\ensuremath{\mathrm{L^0}}}
 \newcommand{\PN}{\ensuremath{\mathrm{N}}}
 \newcommand{\PNa}{\ensuremath{\mathrm{N(1440)P_{11}}}}
 \newcommand{\PNb}{\ensuremath{\mathrm{N(1520)D_{13}}}}
 \newcommand{\PNc}{\ensuremath{\mathrm{N(1535)S_{11}}}}
 \newcommand{\PNd}{\ensuremath{\mathrm{N(1650)S_{11}}}}
 \newcommand{\PNe}{\ensuremath{\mathrm{N(1675)D_{15}}}}
 \newcommand{\PNf}{\ensuremath{\mathrm{N(1680)F_{15}}}}
 \newcommand{\PNg}{\ensuremath{\mathrm{N(1700)D_{13}}}}
 \newcommand{\PNh}{\ensuremath{\mathrm{N(1710)P_{11}}}}
 \newcommand{\PNi}{\ensuremath{\mathrm{N(1720)P_{13}}}}
 \newcommand{\PNj}{\ensuremath{\mathrm{N(2190)G_{17}}}}
 \newcommand{\PNk}{\ensuremath{\mathrm{N(2220)H_{19}}}}
 \newcommand{\PNl}{\ensuremath{\mathrm{N(2250)G_{19}}}}
 \newcommand{\PNm}{\ensuremath{\mathrm{N(2600)I_{1,11}}}}
 \newcommand{\PSHpm}{\ensuremath{\mathrm{\widetilde{H}^{\pm_j}}}}
 \newcommand{\PSHz}{\ensuremath{\mathrm{\widetilde{H}^0_j}}}
 \newcommand{\PSWpm}{\ensuremath{\mathrm{\widetilde{W}^{\pm}}}}
 \newcommand{\PSZz}{\ensuremath{\mathrm{\widetilde{Z}^0}}}
 \newcommand{\PSe}{\ensuremath{\mathrm{\widetilde{e}}}}
 \newcommand{\PSgg}{\ensuremath{\mathrm{\widetilde{\gamma}}}}
 \newcommand{\PSgm}{\ensuremath{\mathrm{\widetilde{\mu}}}}
 \newcommand{\PSgn}{\ensuremath{\mathrm{\widetilde{\nu}}}}
 \newcommand{\PSgt}{\ensuremath{\mathrm{\widetilde{\tau}}}}
 \newcommand{\PSgxpm}{\ensuremath{\mathrm{\widetilde{\chi}^{\pm_i}}}}
 \newcommand{\PSgxz}{\ensuremath{\mathrm{\widetilde{\chi}^0_i}}}
 \newcommand{\PSg}{\ensuremath{\mathrm{\widetilde{g}}}}
 \newcommand{\PSq}{\ensuremath{\mathrm{\widetilde{q}}}}
 \newcommand{\PWR}{\ensuremath{\mathrm{W_R}}}
 \newcommand{\PWm}{\ensuremath{\mathrm{W^-}}}
 \newcommand{\PWpr}{\ensuremath{\mathrm{W^{'}}}}%\prime
 \newcommand{\PWp}{\ensuremath{\mathrm{W^+}}}
 \newcommand{\PW}{\ensuremath{\mathrm{W}}}
 \newcommand{\PZLR}{\ensuremath{\mathrm{Z_{LR}}}}
 \newcommand{\PZgc}{\ensuremath{\mathrm{Z_{\chi}}}}
 \newcommand{\PZge}{\ensuremath{\mathrm{Z_{\eta}}}}
 \newcommand{\PZgy}{\ensuremath{\mathrm{Z_{\psi}}}}
 \newcommand{\PZi}{\ensuremath{\mathrm{Z_1}}}
 \newcommand{\PZz}{\ensuremath{\mathrm{Z^0}}}
 \newcommand{\PaBz}{\ensuremath{\mathrm{\overline{B}}^0}}
 \newcommand{\PaB}{\ensuremath{\mathrm{\overline{B}}}}
 \newcommand{\PaDz}{\ensuremath{\mathrm{\overline{D}^0}}}
 \newcommand{\PaD}{\ensuremath{\overline{\mathrm{D}}}}
 \newcommand{\PaKz}{\ensuremath{\mathrm{\overline{K}^0}}}
 \newcommand{\PaSq}{\ensuremath{\mathrm{\overline{\widetilde{q}}}}}
 \newcommand{\PagL}{\ensuremath{\mathrm{\overline{\Lambda}}}}
 \newcommand{\PagOp}{\ensuremath{\mathrm{\overline{\Omega}^+}}}
 \newcommand{\PagSm}{\ensuremath{\mathrm{\overline{\Sigma}^-}}}
 \newcommand{\PagSp}{\ensuremath{\mathrm{\overline{\Sigma}^+}}}
 \newcommand{\PagSz}{\ensuremath{\mathrm{\overline{\Sigma}^0}}}
 \newcommand{\PagXp}{\ensuremath{\mathrm{\overline{\Xi}^+}}}
 \newcommand{\PagXz}{\ensuremath{\mathrm{\Xi^0}}}
 \newcommand{\Pagne}{\ensuremath{\mathrm{\overline{\nu}_{e}}}}
 \newcommand{\Pagngm}{\ensuremath{\mathrm{\overline{\nu}_{\mu}}}}
 \newcommand{\Pagngt}{\ensuremath{\mathrm{\overline{\nu}_{\tau}}}}
 \newcommand{\Paii}{\ensuremath{\mathrm{a_2(1320)}}}
 \newcommand{\Pai}{\ensuremath{\mathrm{a_1(1260)}}}
 \newcommand{\Pap}{\ensuremath{\mathrm{\overline{p}}}}
 \newcommand{\Paqb}{\ensuremath{\mathrm{\overline{q}_b}}}
 \newcommand{\Paqc}{\ensuremath{\mathrm{\overline{q}_c}}}
 \newcommand{\Paqd}{\ensuremath{\mathrm{\overline{q}_d}}}
 \newcommand{\Paqs}{\ensuremath{\mathrm{\overline{q}_s}}}
 \newcommand{\Paqt}{\ensuremath{\mathrm{\overline{q}_t}}}
 \newcommand{\Paqu}{\ensuremath{\mathrm{\overline{q}_u}}}
 \newcommand{\Paq}{\ensuremath{\mathrm{\overline{q}}}}
 \newcommand{\Paz}{\ensuremath{\mathrm{a_0(980)}}}
 \newcommand{\Pbgcia}{\ensuremath{\mathrm{{\chi}_{b1}(2P)}}}
 \newcommand{\Pbgciia}{\ensuremath{\mathrm{{\chi}_{b2}(2P)}}}
 \newcommand{\Pbgcii}{\ensuremath{\mathrm{{\chi}_{b2}(1P)}}}
 \newcommand{\Pbgci}{\ensuremath{\mathrm{{\chi}_{b1}(1P)}}}
 \newcommand{\Pbgcza}{\ensuremath{\mathrm{{\chi}_{b0}(2P)}}}
 \newcommand{\Pbgcz}{\ensuremath{\mathrm{{\chi}_{b0}(1P)}}}
 \newcommand{\Pbi}{\ensuremath{\mathrm{b_1(1235)}}}
 \newcommand{\PcgLp}{\ensuremath{\mathrm{\Lambda_c^+}}}
 \newcommand{\PcgS}{\ensuremath{\mathrm{\Sigma_c(2455)}}}
 \newcommand{\PcgXp}{\ensuremath{\mathrm{\Xi_c^+}}}
 \newcommand{\PcgXz}{\ensuremath{\mathrm{\Xi_c^0}}}
 \newcommand{\Pcgcii}{\ensuremath{\mathrm{{\chi}_{c2}(1P)}}}
 \newcommand{\Pcgci}{\ensuremath{\mathrm{{\chi}_{c1}(1P)}}}
 \newcommand{\Pcgcz}{\ensuremath{\mathrm{{\chi}_{c0}(1P)}}}
 \newcommand{\Pcgh}{\ensuremath{\mathrm{{\eta}_{c}(1S)}}}
 \newcommand{\Pem}{\ensuremath{\mathrm{e}^-}}
 \newcommand{\Pep}{\ensuremath{\mathrm{e}^+}}
 \newcommand{\Pe}{\ensuremath{\mathrm{e}}}
 \newcommand{\Pfia}{\ensuremath{\mathrm{f}_1(1390)}}
 \newcommand{\Pfib}{\ensuremath{\mathrm{f}_1(1510)}}
 \newcommand{\Pfiia}{\ensuremath{\mathrm{f}_2(1720)}}
 \newcommand{\Pfiib}{\ensuremath{\mathrm{f}_2(2010)}}
 \newcommand{\Pfiic}{\ensuremath{\mathrm{f}_2(2300)}}
 \newcommand{\Pfiid}{\ensuremath{\mathrm{f}_2(2340)}}
 \newcommand{\Pfiipr}{\ensuremath{\mathrm{f}^{'}_2(1525)}}%\prime
 \newcommand{\Pfii}{\ensuremath{\mathrm{f}_2(1270)}}
 \newcommand{\Pfiv}{\ensuremath{\mathrm{f}_4(2050)}}
 \newcommand{\Pfi}{\ensuremath{\mathrm{f}_1(1285)}}
 \newcommand{\Pfza}{\ensuremath{\mathrm{f}_0(1400)}}
 \newcommand{\Pfzb}{\ensuremath{\mathrm{f}_0(1590)}}
 \newcommand{\Pfz}{\ensuremath{\mathrm{f}_0(975)}}
 \newcommand{\PgD}{\ensuremath{\mathrm{\Delta}}}
 \newcommand{\PgDa}{\ensuremath{\mathrm{\Delta(1232)P_{33}}}}
 \newcommand{\PgDb}{\ensuremath{\mathrm{\Delta(1620)S_{31}}}}
 \newcommand{\PgDc}{\ensuremath{\mathrm{\Delta(1700)D_{33}}}}
 \newcommand{\PgDd}{\ensuremath{\mathrm{\Delta(1900)S_{31}}}}
 \newcommand{\PgDe}{\ensuremath{\mathrm{\Delta(1905)F_{35}}}}
 \newcommand{\PgDf}{\ensuremath{\mathrm{\Delta(1910)P_{31}}}}
 \newcommand{\PgDh}{\ensuremath{\mathrm{\Delta(1920)P_{33}}}}
 \newcommand{\PgDi}{\ensuremath{\mathrm{\Delta(1930)D_{35}}}}
 \newcommand{\PgDj}{\ensuremath{\mathrm{\Delta(1950)F_{37}}}}
 \newcommand{\PgDk}{\ensuremath{\mathrm{\Delta(2420)H_{3,11}}}}
 \newcommand{\PgL}{\ensuremath{\mathrm{\Lambda}}}
 \newcommand{\PgLa}{\ensuremath{\mathrm{\Lambda(1405) S_{01}}}}
 \newcommand{\PgLb}{\ensuremath{\mathrm{\Lambda(1520) D_{03}}}}
 \newcommand{\PgLc}{\ensuremath{\mathrm{\Lambda(1600) P_{01}}}}
 \newcommand{\PgLd}{\ensuremath{\mathrm{\Lambda(1670) S_{01}}}}
 \newcommand{\PgLe}{\ensuremath{\mathrm{\Lambda(1690) D_{03}}}}
 \newcommand{\PgLf}{\ensuremath{\mathrm{\Lambda(1800) S_{01}}}}
 \newcommand{\PgLg}{\ensuremath{\mathrm{\Lambda(1810) P_{01}}}}
 \newcommand{\PgLh}{\ensuremath{\mathrm{\Lambda(1820) F_{05}}}}
 \newcommand{\PgLi}{\ensuremath{\mathrm{\Lambda(1830) D_{05}}}}
 \newcommand{\PgLj}{\ensuremath{\mathrm{\Lambda(1890) P_{03}}}}
 \newcommand{\PgLk}{\ensuremath{\mathrm{\Lambda(2100) G_{07}}}}
 \newcommand{\PgLl}{\ensuremath{\mathrm{\Lambda(2110) F_{05}}}}
 \newcommand{\PgLm}{\ensuremath{\mathrm{\Lambda(2350) H_{09}}}}
 \newcommand{\PgO}{\ensuremath{\mathrm{\Omega}}}
 \newcommand{\PgOm}{\ensuremath{\mathrm{\Omega^-}}}
 \newcommand{\PgOma}{\ensuremath{\mathrm{\Omega(2250)^-}}}
 \newcommand{\PgS}{\ensuremath{\mathrm{\Sigma}}}
 \newcommand{\PgSa}{\ensuremath{\mathrm{\Sigma(1385) P_{13}}}}
 \newcommand{\PgSb}{\ensuremath{\mathrm{\Sigma(1660) P_{11}}}}
 \newcommand{\PgSc}{\ensuremath{\mathrm{\Sigma(1670) D_{13}}}}
 \newcommand{\PgSd}{\ensuremath{\mathrm{\Sigma(1750) S_{11}}}}
 \newcommand{\PgSe}{\ensuremath{\mathrm{\Sigma(1775) D_{15}}}}
 \newcommand{\PgSf}{\ensuremath{\mathrm{\Sigma(1915) F_{15}}}}
 \newcommand{\PgSg}{\ensuremath{\mathrm{\Sigma(1940) D_{13}}}}
 \newcommand{\PgSh}{\ensuremath{\mathrm{\Sigma(2030) F_{17}}}}
 \newcommand{\PgSi}{\ensuremath{\mathrm{\Sigma(2050)}}}
 \newcommand{\PgSm}{\ensuremath{\mathrm{\Sigma^-}}}
 \newcommand{\PgSp}{\ensuremath{\mathrm{\Sigma^+}}}
 \newcommand{\PgSz}{\ensuremath{\mathrm{\Sigma^0}}}
 \newcommand{\PgU}{\ensuremath{\mathrm{\Upsilon}}}
 \newcommand{\PgUa}{\ensuremath{\mathrm{\Upsilon(1S)}}}
 \newcommand{\PgUb}{\ensuremath{\mathrm{\Upsilon(2S)}}}
 \newcommand{\PgUc}{\ensuremath{\mathrm{\Upsilon(3S)}}}
 \newcommand{\PgUd}{\ensuremath{\mathrm{\Upsilon(3S)}}}
 \newcommand{\PgUe}{\ensuremath{\mathrm{\Upsilon(10860)}}}
 \newcommand{\PgUf}{\ensuremath{\mathrm{\Upsilon(11020)}}}
 \newcommand{\PgX}{\ensuremath{\mathrm{\Xi}}}
 \newcommand{\PgXa}{\ensuremath{\mathrm{\Xi(1530)P_{13}}}}
 \newcommand{\PgXb}{\ensuremath{\mathrm{\Xi(1690)}}}
 \newcommand{\PgXc}{\ensuremath{\mathrm{\Xi(1820)D_{13}}}}
 \newcommand{\PgXd}{\ensuremath{\mathrm{\Xi(1950)}}}
 \newcommand{\PgXe}{\ensuremath{\mathrm{\Xi(2030)}}}
 \newcommand{\PgXm}{\ensuremath{\mathrm{\Xi^-}}}
 \newcommand{\PgXz}{\ensuremath{\mathrm{\overline{\Xi}^0}}}
 \newcommand{\Pgfa}{\ensuremath{\mathrm{\phi(1680)}}}
 \newcommand{\Pgfiii}{\ensuremath{\mathrm{\phi_3(1850)}}}
 \newcommand{\Pgf}{\ensuremath{\mathrm{\phi(1020)}}}
 \newcommand{\Pgg}{\ensuremath{\mathrm{\gamma}}}
 \newcommand{\Pgha}{\ensuremath{\mathrm{\eta(1295)}}}
 \newcommand{\Pghb}{\ensuremath{\mathrm{\eta(1440)}}}
 \newcommand{\Pghpr}{\ensuremath{\mathrm{\eta^{'}(958)}}}%\prime
 \newcommand{\Pgh}{\ensuremath{\mathrm{\eta}}}
 \newcommand{\Pgmm}{\ensuremath{\mathrm{\mu^-}}}
 \newcommand{\Pgmp}{\ensuremath{\mathrm{\mu^+}}}
 \newcommand{\Pgm}{\ensuremath{\mathrm{\mu}}}
 \newcommand{\Pgne}{\ensuremath{\mathrm{\nu_{e}}}}
 \newcommand{\Pgngm}{\ensuremath{\mathrm{\nu_{\mu}}}}
 \newcommand{\Pgngt}{\ensuremath{\mathrm{\nu_{\tau}}}}
 \newcommand{\Pgoa}{\ensuremath{\mathrm{\omega(1390)}}}
 \newcommand{\Pgob}{\ensuremath{\mathrm{\omega(1600)}}}
 \newcommand{\Pgoiii}{\ensuremath{\mathrm{\omega_3(1670)}}}
 \newcommand{\Pgo}{\ensuremath{\mathrm{\omega(783)}}}
 \newcommand{\Pgpa}{\ensuremath{\mathrm{\pi(1300)}}}
 \newcommand{\Pgpii}{\ensuremath{\mathrm{\pi_2(1670)}}}
 \newcommand{\Pgpm}{\ensuremath{\mathrm{\pi^-}}}
 \newcommand{\Pgppm}{\ensuremath{\mathrm{\pi^{\pm }}}}
 \newcommand{\Pgpp}{\ensuremath{\mathrm{\pi^+}}}
 \newcommand{\Pgpz}{\ensuremath{\mathrm{\pi^0}}}
 \newcommand{\Pgp}{\ensuremath{\mathrm{\pi}}}
 \newcommand{\Pgra}{\ensuremath{\mathrm{\rho(1450)}}}
 \newcommand{\Pgrb}{\ensuremath{\mathrm{\rho(1700)}}}
 \newcommand{\Pgriii}{\ensuremath{\mathrm{\rho_3(1690)}}}
 \newcommand{\Pgr}{\ensuremath{\mathrm{\rho(770)}}}
 \newcommand{\Pgt}{\ensuremath{\mathrm{\tau}}}
 \newcommand{\Pgya}{\ensuremath{\mathrm{\psi(3770)}}}
 \newcommand{\Pgyb}{\ensuremath{\mathrm{\psi(4040)}}}
 \newcommand{\Pgyc}{\ensuremath{\mathrm{\psi(4160)}}}
 \newcommand{\Pgyd}{\ensuremath{\mathrm{\psi(4415)}}}
 \newcommand{\Pgy}{\ensuremath{\mathrm{\psi(2S)}}}
 \newcommand{\Phia}{\ensuremath{\mathrm{h_1(1170)}}}
 \newcommand{\Pn}{\ensuremath{\mathrm{n}}}
 \newcommand{\Pp}{\ensuremath{\mathrm{p}}}
 \newcommand{\Pqb}{\ensuremath{\mathrm{q_b}}}
 \newcommand{\Pqc}{\ensuremath{\mathrm{q_c}}}
 \newcommand{\Pqd}{\ensuremath{\mathrm{q_d}}}
 \newcommand{\Pqs}{\ensuremath{\mathrm{q_s}}}
 \newcommand{\Pqt}{\ensuremath{\mathrm{q_t}}}
 \newcommand{\Pqu}{\ensuremath{\mathrm{q_u}}}
 \newcommand{\Pq}{\ensuremath{\mathrm{q}}}
 \newcommand{\PsDipm}{\ensuremath{\mathrm{D_{s1}(2536)^{\pm}}}}
 \newcommand{\PsDm}{\ensuremath{\mathrm{D_{s}^-}}}
 \newcommand{\PsDp}{\ensuremath{\mathrm{D_{s}^+}}}
 \newcommand{\PsDst}{\ensuremath{\mathrm{D_{s}^{\ast}}}}
\endinput

\def\Fileversion$#1: #2 ${\gdef\fileversion{#2}}
\def\Filedate$#1: #2-#3-#4 #5 ${\gdef\filedate{#2/#3/#4}}
\Fileversion$Revision: 460316 $
\Filedate$Date: 2018-05-15 06:41:40 -0400 (Tue, 15 May 2018) $
%%%%%%%%%%%%%%%%%%%%%%%%%%%%%%%%%%%%%%%%%%%%%%%%%%%%%%%%%%%%%%%%%%%%
%
%  CMS Common definitions style file
%
%  N.B. use of \newcommand rather than \newcommand means
%       that a definition is ignored if already specified
%
%                                              L. Taylor 18 Feb 2005
%%%%%%%%%%%%%%%%%%%%%%%%%%%%%%%%%%%%%%%%%%%%%%%%%%%%%%%%%%%%%%%%%%%%
\NeedsTeXFormat{LaTeX2e}
\ProvidesPackage{ptdr-definitions}[\filedate\space CMS Additional Macro Definitions (\fileversion)]
\RequirePackage{xspace}
\RequirePackage{amsmath}

% Some shorthand
% turn off italics
\newcommand {\etal}{\mbox{et al.}\xspace} %et al. - no preceding comma
\newcommand {\ie}{\mbox{i.e.}\xspace}     %i.e.
\newcommand {\eg}{\mbox{e.g.}\xspace}     %e.g.
\newcommand {\etc}{\mbox{etc.}\xspace}     %etc.
\newcommand {\vs}{\mbox{\sl vs.}\xspace}      %vs.
\newcommand {\mdash}{\ensuremath{\mathrm{-}}} % for use within formulas
\providecommand {\NA}{\ensuremath{\text{---}}}    % for Not applicable (or available). Needs to be renewcommanded for APS to \cdots

% some terms whose definition we may change
\newcommand {\Lone}{Level-1\xspace} % Level-1 or L1 ?
\newcommand {\Ltwo}{Level-2\xspace}
\newcommand {\Lthree}{Level-3\xspace}

% Some software programs (alphabetized)
\newcommand{\ACERMC} {\textsc{AcerMC}\xspace}
\newcommand{\ALPGEN} {{\textsc{alpgen}}\xspace}
\newcommand{\BLACKHAT} {{\textsc{BlackHat}}\xspace}
\newcommand{\CALCHEP} {{\textsc{CalcHEP}}\xspace}
\newcommand{\CHARYBDIS} {{\textsc{charybdis}}\xspace}
\newcommand{\CMKIN} {\textsc{cmkin}\xspace}
\newcommand{\CMSIM} {{\textsc{cmsim}}\xspace}
\newcommand{\CMSSW} {{\textsc{cmssw}}\xspace}
\newcommand{\COBRA} {{\textsc{cobra}}\xspace}
\newcommand{\COCOA} {{\textsc{cocoa}}\xspace}
\newcommand{\COMPHEP} {\textsc{CompHEP}\xspace}
\newcommand{\EVTGEN} {{\textsc{evtgen}}\xspace}
\newcommand{\FAMOS} {{\textsc{famos}}\xspace}
\newcommand{\FASTJET} {{\textsc{FastJet}}\xspace}
\newcommand{\FEWZ} {{\textsc{fewz}}\xspace}
\newcommand{\GARCON} {\textsc{garcon}\xspace}
\newcommand{\GARFIELD} {{\textsc{garfield}}\xspace}
\newcommand{\GEANE} {{\textsc{geane}}\xspace}
\newcommand{\GEANTfour} {{\textsc{Geant4}}\xspace}
\newcommand{\GEANTthree} {{\textsc{geant3}}\xspace}
\newcommand{\GEANT} {{\textsc{geant}}\xspace}
\newcommand{\HDECAY} {\textsc{hdecay}\xspace}
\newcommand{\HERWIG} {{\textsc{herwig}}\xspace}
\newcommand{\HERWIGpp} {{\textsc{herwig++}}\xspace}
\newcommand{\POWHEG} {{\textsc{powheg}}\xspace}
\newcommand{\HIGLU} {{\textsc{higlu}}\xspace}
\newcommand{\HIJING} {{\textsc{hijing}}\xspace}
\newcommand{\HYDJET} {{\textsc{hydjet}}\xspace}
\newcommand{\IGUANA} {\textsc{iguana}\xspace}
\newcommand{\ISAJET} {{\textsc{isajet}}\xspace}
\newcommand{\ISAPYTHIA} {{\textsc{isapythia}}\xspace}
\newcommand{\ISASUGRA} {{\textsc{isasugra}}\xspace}
\newcommand{\ISASUSY} {{\textsc{isasusy}}\xspace}
\newcommand{\ISAWIG} {{\textsc{isawig}}\xspace}
\newcommand{\MADGRAPH} {\textsc{MadGraph}\xspace}
\newcommand{\MCATNLO} {\textsc{mc@nlo}\xspace}
\newcommand{\MCFM} {\textsc{mcfm}\xspace}
\newcommand{\MILLEPEDE} {{\textsc{millepede}}\xspace}
\newcommand{\ORCA} {{\textsc{orca}}\xspace}
\newcommand{\OSCAR} {{\textsc{oscar}}\xspace}
\newcommand{\PHOTOS} {\textsc{photos}\xspace}
\newcommand{\PROSPINO} {\textsc{prospino}\xspace}
\newcommand{\PYTHIA} {{\textsc{pythia}}\xspace}
\newcommand{\SHERPA} {{\textsc{sherpa}}\xspace}
\newcommand{\TAUOLA} {\textsc{tauola}\xspace}
\newcommand{\TOPREX} {\textsc{TopReX}\xspace}
\newcommand{\XDAQ} {{\textsc{xdaq}}\xspace}
\newcommand{\MGvATNLO}{\MADGRAPH{}5\_a\MCATNLO}


%  Experiments
\newcommand {\DZERO}{D0\xspace}     %etc.


% Measurements and units...

\newcommand{\de}{\ensuremath{^\circ}}
\newcommand{\ten}[1]{\ensuremath{\times \text{10}^\text{#1}}}
\newcommand{\unit}[1]{\ensuremath{\text{\,#1}}\xspace}
\newcommand{\mum}{\ensuremath{\,\mu\text{m}}\xspace}
\newcommand{\micron}{\ensuremath{\,\mu\text{m}}\xspace}
\newcommand{\cm}{\ensuremath{\,\text{cm}}\xspace}
\newcommand{\mm}{\ensuremath{\,\text{mm}}\xspace}
\newcommand{\mus}{\ensuremath{\,\mu\text{s}}\xspace}
\newcommand{\keV}{\ensuremath{\,\text{ke\hspace{-.08em}V}}\xspace}
\newcommand{\MeV}{\ensuremath{\,\text{Me\hspace{-.08em}V}}\xspace}
\newcommand{\MeVns}{\ensuremath{\text{Me\hspace{-.08em}V}}\xspace} % no leading thinspace
\newcommand{\GeV}{\ensuremath{\,\text{Ge\hspace{-.08em}V}}\xspace}
\newcommand{\GeVns}{\ensuremath{\text{Ge\hspace{-.08em}V}}\xspace} % no leading thinspace
\newcommand{\gev}{\GeV}
\newcommand{\TeV}{\ensuremath{\,\text{Te\hspace{-.08em}V}}\xspace}
\newcommand{\TeVns}{\ensuremath{\text{Te\hspace{-.08em}V}}\xspace} % no leading thinspace
\newcommand{\PeV}{\ensuremath{\,\text{Pe\hspace{-.08em}V}}\xspace}
\newcommand{\keVc}{\ensuremath{{\,\text{ke\hspace{-.08em}V\hspace{-0.16em}/\hspace{-0.08em}}c}}\xspace}
\newcommand{\MeVc}{\ensuremath{{\,\text{Me\hspace{-.08em}V\hspace{-0.16em}/\hspace{-0.08em}}c}}\xspace}
\newcommand{\GeVc}{\ensuremath{{\,\text{Ge\hspace{-.08em}V\hspace{-0.16em}/\hspace{-0.08em}}c}}\xspace}
\newcommand{\GeVcns}{\ensuremath{{\text{Ge\hspace{-.08em}V\hspace{-0.16em}/\hspace{-0.08em}}c}}\xspace} % no leading thinspace
\newcommand{\TeVc}{\ensuremath{{\,\text{Te\hspace{-.08em}V\hspace{-0.16em}/\hspace{-0.08em}}c}}\xspace}
\newcommand{\keVcc}{\ensuremath{{\,\text{ke\hspace{-.08em}V\hspace{-0.16em}/\hspace{-0.08em}}c^\text{2}}}\xspace}
\newcommand{\MeVcc}{\ensuremath{{\,\text{Me\hspace{-.08em}V\hspace{-0.16em}/\hspace{-0.08em}}c^\text{2}}}\xspace}
\newcommand{\GeVcc}{\ensuremath{{\,\text{Ge\hspace{-.08em}V\hspace{-0.16em}/\hspace{-0.08em}}c^\text{2}}}\xspace}
\newcommand{\GeVccns}{\ensuremath{{\text{Ge\hspace{-.08em}V\hspace{-0.16em}/\hspace{-0.08em}}c^\text{2}}}\xspace} % no leading thinspace
\newcommand{\TeVcc}{\ensuremath{{\,\text{Te\hspace{-.08em}V\hspace{-0.16em}/\hspace{-0.08em}}c^\text{2}}}\xspace}

\newcommand{\pbinv} {\mbox{\ensuremath{\,\text{pb}^\text{$-$1}}}\xspace}
\newcommand{\fbinv} {\mbox{\ensuremath{\,\text{fb}^\text{$-$1}}}\xspace}
\newcommand{\nbinv} {\mbox{\ensuremath{\,\text{nb}^\text{$-$1}}}\xspace}
\newcommand{\mubinv} {\ensuremath{\,\mu\mathrm{b}^{-1}}\xspace}
\newcommand{\mbinv} {\ensuremath{\,\mathrm{mb}^{-1}}\xspace}
\newcommand{\percms}{\ensuremath{\,\text{cm}^\text{$-$2}\,\text{s}^\text{$-$1}}\xspace}
\newcommand{\lumi}{\ensuremath{\mathcal{L}}\xspace}
\newcommand{\Lumi}{\ensuremath{\mathcal{L}}\xspace}%both upper and lower
%
% Need a convention here:
\newcommand{\LvLow}  {\ensuremath{\mathcal{L}=\text{10}^\text{32}\,\text{cm}^\text{$-$2}\,\text{s}^\text{$-$1}}\xspace}
\newcommand{\LLow}   {\ensuremath{\mathcal{L}=\text{10}^\text{33}\,\text{cm}^\text{$-$2}\,\text{s}^\text{$-$1}}\xspace}
\newcommand{\lowlumi}{\ensuremath{\mathcal{L}=\text{2}\times \text{10}^\text{33}\,\text{cm}^\text{$-$2}\,\text{s}^\text{$-$1}}\xspace}
\newcommand{\LMed}   {\ensuremath{\mathcal{L}=\text{2}\times \text{10}^\text{33}\,\text{cm}^\text{$-$2}\,\text{s}^\text{$-$1}}\xspace}
\newcommand{\LHigh}  {\ensuremath{\mathcal{L}=\text{10}^\text{34}\,\text{cm}^\text{$-$2}\,\text{s}^\text{$-$1}}\xspace}
\newcommand{\hilumi} {\ensuremath{\mathcal{L}=\text{10}^\text{34}\,\text{cm}^\text{$-$2}\,\text{s}^\text{$-$1}}\xspace}

% Physics symbols ...

\newcommand{\PT}{\ensuremath{p_{\mathrm{T}}}\xspace}
\newcommand{\pt}{\ensuremath{p_{\mathrm{T}}}\xspace}
\newcommand{\ET}{\ensuremath{E_{\mathrm{T}}}\xspace}
\newcommand{\HT}{\ensuremath{H_{\mathrm{T}}}\xspace}
\newcommand{\et}{\ensuremath{E_{\mathrm{T}}}\xspace}
\newcommand{\Em}{\ensuremath{E\hspace{-0.6em}/}\xspace}
\newcommand{\Pm}{\ensuremath{p\hspace{-0.5em}/}\xspace}
\newcommand{\PTm}{\ensuremath{{p}_\mathrm{T}\hspace{-1.02em}/\kern 0.5em}\xspace}
\newcommand{\PTslash}{\PTm}
\newcommand{\ETm}{\ensuremath{E_{\mathrm{T}}^{\text{miss}}}\xspace}
\newcommand{\MET}{\ETm}
\newcommand{\ETmiss}{\ETm}
\newcommand{\ptmiss}{\ensuremath{\pt^\text{miss}}\xspace}
\newcommand{\ETslash}{\ensuremath{E_{\mathrm{T}}\hspace{-1.1em}/\kern0.45em}\xspace}
\newcommand{\VEtmiss}{\ensuremath{{\vec E}_{\mathrm{T}}^{\text{miss}}}\xspace}
\newcommand{\ptvec}{\ensuremath{{\vec p}_{\mathrm{T}}}\xspace}
\newcommand{\ptvecmiss}{\ensuremath{{\vec p}_{\mathrm{T}}^{\kern1pt\text{miss}}}\xspace}
\newcommand{\tauh}{\ensuremath{\PGt_\mathrm{h}}\xspace}
\newcommand{\sqrtsNN}{\ensuremath{\sqrt{\smash[b]{s_{_{\mathrm{NN}}}}}}\xspace}
\newcommand{\mht}{\ensuremath{H_{\mathrm{T}}^{\text{miss}}}\xspace}
\newcommand{\htvecmiss}{\ensuremath{\vec{H}_{\text{T}}^{\text{miss}}}\xspace}

% roman face derivative
\newcommand{\dd}[2]{\ensuremath{\frac{\mathrm{d} #1}{\mathrm{d} #2}}}
\newcommand{\ddinline}[2]{\ensuremath{\mathrm{d} #1/\mathrm{d} #2}}
\newcommand{\rd}{\ensuremath{\mathrm{d}}}
\newcommand{\re}{\ensuremath{\mathrm{e}}}
% absolute value
\newcommand{\abs}[1]{\ensuremath{\lvert #1 \rvert}}
% misc
\newcommand{\CL}{\ensuremath{\text{CL}}\xspace}
\newcommand{\CLs}{\ensuremath{\text{CL}_\text{s}}\xspace}
\newcommand{\CLsb}{\ensuremath{\text{CL}_\text{s+b}}\xspace}



%\ifthenelse{\boolean{cms@italic}}{\newcommand{\mathrm}[1]{#1}}{\newcommand{\mathrm}[1]{\mathrm{#1}}}

% Particle names which track the italic/non-italic face convention
\newcommand{\zp}{\ensuremath{{\mathrm{Z}^\prime}}\xspace} % plain Z'
\newcommand{\JPsi}{\ensuremath{{\mathrm{J}\hspace{-.08em}/\hspace{-.14em}\psi}}\xspace} % J/Psi (no mass)
%\newcommand{\Z}{\ensuremath{\mathrm{Z}}\xspace} % plain Z (no superscript 0)
\newcommand{\ttbar}{\ensuremath{{\mathrm{t}\overline{\mathrm{t}}}}\xspace} % t-tbar

% Extensions for missing names in PENNAMES % note no xspace, to match syntax in PENNAMES
\newcommand{\cPgn}{\ensuremath{\nu}} % generic neutrino
\providecommand{\Pgn}{\ensuremath{\nu}} % generic neutrino
\newcommand{\cPagn}{\ensuremath{\overline{\nu}}} % generic neutrino
\providecommand{\Pagn}{\ensuremath{\overline{\nu}}} % generic neutrino
\newcommand{\cPgg}{\ensuremath{\gamma}} % gamma
\newcommand{\cPJgy}{\ensuremath{\mathrm{J}\hspace{-.08em}/\hspace{-.14em}\psi}} % J/Psi (no mass)
\newcommand{\cPZ}{\ensuremath{\mathrm{Z}}} % plain Z (no superscript 0)
\newcommand{\cPZpr}{\ensuremath{\mathrm{Z}'}} % plain Z'
\newcommand{\cPqt}{\ensuremath{\mathrm{t}}} % t for t quark
\newcommand{\cPqb}{\ensuremath{\mathrm{b}}} % b for b quark
\newcommand{\cPqc}{\ensuremath{\mathrm{c}}} % c for c quark
\newcommand{\cPqs}{\ensuremath{\mathrm{s}}} % s for s quark
\newcommand{\cPqu}{\ensuremath{\mathrm{u}}} % u for u quark
\newcommand{\cPqd}{\ensuremath{\mathrm{d}}} % d for d quark
\newcommand{\cPq}{\ensuremath{\mathrm{q}}} % generic quark
\newcommand{\cPg}{\ensuremath{\mathrm{g}}} % generic gluon
\newcommand{\cPG}{\ensuremath{\mathrm{G}}} % Graviton
\newcommand{\cPaqt}{\ensuremath{\overline{\mathrm{t}}}} % t for t anti-quark
\newcommand{\cPaqb}{\ensuremath{\overline{\mathrm{b}}}} % b for b anti-quark
\newcommand{\cPaqc}{\ensuremath{\overline{\mathrm{c}}}} % c for c anti-quark
\newcommand{\cPaqs}{\ensuremath{\overline{\mathrm{s}}}} % s for s anti-quark
\newcommand{\cPaqu}{\ensuremath{\overline{\mathrm{u}}}} % u for u anti-quark
\newcommand{\cPaqd}{\ensuremath{\overline{\mathrm{d}}}} % d for d anti-quark
\newcommand{\cPaq}{\ensuremath{\overline{\mathrm{q}}}} % generic anti-quark
\newcommand{\cPKstz}{\ensuremath{\mathrm{K}^{\ast0}}\xspace} %note has xspace
% future symbols from heppennames2
\providecommand{\PGp}{\ensuremath{\pi}\xspace} % pi
\providecommand{\PGpp}{\ensuremath{\pi^+}\xspace} % pi
\providecommand{\PGpm}{\ensuremath{\pi^-}\xspace} % pi
\providecommand{\PGpz}{\ensuremath{\pi^0}\xspace} % pi
\providecommand{\PGr}{\ensuremath{\rho}\xspace} % pi
\providecommand{\PDast}{\ensuremath{\mathrm{D}^\ast}\xspace} % D star
\providecommand{\PH}{\ensuremath{\mathrm{H}}\xspace} % plain Higgs
\providecommand{\Ph}{\ensuremath{\mathrm{h}}\xspace} % SUSY Higgs
\providecommand{\Pa}{\ensuremath{\mathrm{a}}\xspace}
\providecommand{\PSA}{\ensuremath{\mathrm{A}}\xspace} %pseudoscalar A Higgs
\providecommand{\PJGy}{\ensuremath{\mathrm{J}\hspace{-.08em}/\hspace{-.14em}\psi}\xspace} % J/Psi (no mass)
\providecommand{\PBzs}{\ensuremath{\mathrm{B}^0_\mathrm{s}}\xspace} % B^0_s
\providecommand{\Pg}{\ensuremath{\mathrm{g}}\xspace} % generic gluon
\providecommand{\PSg}{\ensuremath{\widetilde{\mathrm{g}}}\xspace} % gluino
\providecommand{\PSQ}{\ensuremath{\widetilde{\mathrm{q}}}\xspace} % squark
\providecommand{\PSGm}{\ensuremath{\widetilde{\mu}}\xspace} % smuon
\providecommand{\PSe}{\ensuremath{\widetilde{\mathrm{e}}}\xspace} % selectron
\providecommand{\PASQ}{\ensuremath{\overline{\widetilde{\mathrm{q}}}}\xspace} % anti quark
\providecommand{\PXXA}{\ensuremath{\mathrm{A}}\xspace} % axion
\providecommand{\PXXG}{\ensuremath{\mathrm{G}}\xspace} % graviton
\providecommand{\PXXSG}{\ensuremath{\widetilde{\PXXG}}\xspace} % gravitino
\providecommand{\PSGcp}{\ensuremath{\widetilde{\chi}^+}\xspace} % lightest positive chargino
\providecommand{\PSGcm}{\ensuremath{\widetilde{\chi}^-}\xspace} % lightest negative chargino
\providecommand{\PSGc}{\ensuremath{\widetilde{\chi}}\xspace} % neutralino
\providecommand{\PSGcz}{\ensuremath{\widetilde{\chi}^0}\xspace} % neutralino with superscript 0
\providecommand{\PSGczDo}{\ensuremath{\widetilde{\chi}^{0}_{1}}\xspace} % neutralino
\providecommand{\PSGcmDo}{\ensuremath{\widetilde{\chi}^{-}_{1}}\xspace} % neutralino
\providecommand{\PSGczDt}{\ensuremath{\widetilde{\chi}^{0}_{2}}\xspace} % neutralino
\providecommand{\PSGcpm}{\ensuremath{\widetilde{\chi}^\pm}\xspace} % neutralino
\providecommand{\PSGcpmDo}{\ensuremath{\widetilde{\chi}^\pm_{1}}\xspace} % neutralino
\providecommand{\PSGcpDo}{\ensuremath{\widetilde{\chi}^{+}_{1}}\xspace} % neutralino
\providecommand{\Pl}{\ensuremath{\mathrm{l}}\xspace} % non-ell lepton
\providecommand{\PAl}{\ensuremath{\overline{\mathrm{l}}}\xspace} % non-ell anti-lepton
\providecommand{\PGnl}{\ensuremath{\nu_\mathrm{l}}\xspace} % lepton neutrino
\providecommand{\PAGnl}{\ensuremath{\overline{\nu}_\mathrm{l}}\xspace} % anti-lepton neutrino
\providecommand{\PQtpr}{\ensuremath{\mathrm{t}^{\prime}}\xspace} % t'
\providecommand{\PAQtpr}{\ensuremath{\bar{\mathrm{t}}^\prime}\xspace} % t'-bar; needs to be converted to overline-requires rework a la heppennames
\providecommand{\PQbpr}{\ensuremath{\mathrm{b}^{\prime}}\xspace} % b'
\providecommand{\PAQbpr}{\ensuremath{\bar{\mathrm{b}}^\prime}\xspace} % b'-bar; needs same as anti-t'
\providecommand{\PGg}{\ensuremath{\gamma}\xspace} % gamma
\providecommand{\PKzS}{\ensuremath{\mathrm{K}^0_\mathrm{S}}\xspace} % K short
\providecommand{\PBs}{\ensuremath{\mathrm{B}_\mathrm{s}}\xspace} % B sub s
\providecommand{\PSQu}{\ensuremath{\widetilde{\mathrm{u}}}\xspace}
\providecommand{\PSQd}{\ensuremath{\widetilde{\mathrm{d}}}\xspace}
\providecommand{\PSQc}{\ensuremath{\widetilde{\mathrm{c}}}\xspace}
\providecommand{\PSQs}{\ensuremath{\widetilde{\mathrm{s}}}\xspace}
\providecommand{\PSQt}{\ensuremath{\widetilde{\mathrm{t}}}\xspace} % stop
\providecommand{\PSQb}{\ensuremath{\widetilde{\mathrm{b}}}\xspace}
\providecommand{\PASQt}{\ensuremath{\overline{\widetilde{\mathrm{t}}}}\xspace} % anti stop
\providecommand{\PASQb}{\ensuremath{\overline{\widetilde{\mathrm{b}}}}\xspace} % anti sbottom
\providecommand{\PSGt}{\ensuremath{\widetilde{\tau}}\xspace} % stau
\providecommand{\PZ}{\ensuremath{\mathrm{Z}}\xspace} % may have some confusion with the \xspace...
\providecommand{\PZpr}{\ensuremath{\mathrm{Z}'}\xspace} % plain Z' using prime
\renewcommand{\PWpr}{\ensuremath{\mathrm{W}'}\xspace} % use prime like pennames2
\providecommand{\PWmp}{\ensuremath{\mathrm{W}^\mp}\xspace}
\providecommand{\PDstp}{\ensuremath{\mathrm{D}^{\ast+}}\xspace}
\providecommand{\PDstm}{\ensuremath{\mathrm{D}^{\ast-}}\xspace}
\providecommand{\PGn}{\ensuremath{\nu}\xspace} % generic neutrino
\providecommand{\PAGn}{\ensuremath{\overline{\nu}}\xspace} % generic neutrino
\providecommand{\PSQtDo}{\ensuremath{\widetilde{\mathrm{t}}_1}\xspace}
\providecommand{\PSQtDt}{\ensuremath{\widetilde{\mathrm{t}}_2}\xspace}
\providecommand{\PQt}{\ensuremath{\mathrm{t}}\xspace} % t
\providecommand{\PAQt}{\ensuremath{\overline{\mathrm{t}}}\xspace} %
\providecommand{\PQb}{\ensuremath{\mathrm{b}}\xspace} % b
\providecommand{\PAQb}{\ensuremath{\overline{\mathrm{b}}}\xspace} %
\providecommand{\PGm}{\ensuremath{\mu}\xspace} % muon
\providecommand{\PGmm}{\ensuremath{\mu^-}\xspace} % muon
\providecommand{\PGmp}{\ensuremath{\mu^+}\xspace} % muon
\providecommand{\PGmpm}{\ensuremath{\mu^\pm}\xspace} % muon
\providecommand{\PGt}{\ensuremath{\tau}\xspace} % tau
\providecommand{\PAGt}{\ensuremath{\overline{\tau}}\xspace} % anti-tau
\providecommand{\PGpz}{\ensuremath{\pi^0}\xspace}
\providecommand{\PQq}{\ensuremath{\mathrm{q}}\xspace} % quark (generic)
\providecommand{\PQd}{\ensuremath{\mathrm{d}}\xspace} % down quark
\providecommand{\PQu}{\ensuremath{\mathrm{u}}\xspace} % up quark
\providecommand{\PQs}{\ensuremath{\mathrm{s}}\xspace} % top quark
\providecommand{\PQc}{\ensuremath{\mathrm{c}}\xspace} % top quark
\providecommand{\PAQq}{\ensuremath{\overline{\mathrm{q}}}\xspace} % quark (generic)
\providecommand{\PAQd}{\ensuremath{\overline{\mathrm{d}}}\xspace} % down quark
\providecommand{\PAQu}{\ensuremath{\overline{\mathrm{u}}}\xspace} % up quark
\providecommand{\PAQs}{\ensuremath{\overline{\mathrm{s}}}\xspace} % top quark
\providecommand{\PAQc}{\ensuremath{\overline{\mathrm{c}}}\xspace} % top quark
\providecommand{\PGne}{\ensuremath{\nu_\mathrm{e}}\xspace} % electron neutrino
\providecommand{\PAGne}{\ensuremath{\overline{\nu}_\mathrm{e}}\xspace} % anti-electron neutrino
\providecommand{\PGnGm}{\ensuremath{\nu_\PGm}\xspace} % muon neutrino
\providecommand{\PAGnGm}{\ensuremath{\overline{\nu}_\PGm}\xspace} % anti-muon neutrino
\providecommand{\PGnGt}{\ensuremath{\nu_\PGt}\xspace} % tau neutrino
\providecommand{\PAGnGt}{\ensuremath{\overline{\nu}_\PGt}\xspace} % anti-tau neutrino
\providecommand{\PAp}{\ensuremath{\overline{\mathrm{p}}}\xspace} % anti-proton
\providecommand{\PAn}{\ensuremath{\overline{\mathrm{n}}}\xspace} % anti-neutron
\providecommand{\PGc}{\ensuremath{\chi}\xspace} % chi (charm, but also SUSY)
\providecommand{\PGcc}{\ensuremath{\chi_{\PQc}}\xspace}
\providecommand{\PGcb}{\ensuremath{\chi_{\PQb}}\xspace}
\providecommand{\PDz}{\ensuremath{\mathrm{D}^0}\xspace} % D0 meson
\providecommand{\PADz}{\ensuremath{\overline{\mathrm{D}}^0}\xspace} % anti-D0 meson
\providecommand{\PAD}{\ensuremath{\overline{\mathrm{D}}}\xspace} % anti-D meson
\providecommand{\PAK}{\ensuremath{\overline{\mathrm{K}}}\xspace} % anti-K meson
\providecommand{\PAKz}{\ensuremath{\overline{\mathrm{K}}^0}\xspace} % anti-K0 meson
\providecommand{\PABz}{\ensuremath{\overline{\mathrm{B}}^0}\xspace} % anti-B0 meson
\providecommand{\PGLb}{\ensuremath{\Lambda_\mathrm{b}}\xspace} % Lambda b

% our extensions for pennames2
\providecommand{\Pepm}{\ensuremath{\mathrm{e}^\pm}\xspace}
\providecommand{\Pemp}{\ensuremath{\mathrm{e}^\mp}\xspace}
\providecommand{\PGmpm}{\ensuremath{\mu^\pm}\xspace}
\providecommand{\PGmmp}{\ensuremath{\mu^\mp}\xspace} % not available in pennames2, AFAIK
% for APS style tables
%\ifthenelse{\boolean{cms@external}}{%
%\newenvironment{scotch}[1]{\protect\centering\ruledtabular\tabular{#1}}{\endtabular\endruledtabular}
%}{
%\newenvironment{scotch}[1]{\protect\centering\tabular{#1}\hline\hline}{\hline\endtabular}
%}
% Other
\newcommand{\MD}{\ensuremath{{M_\mathrm{D}}}\xspace}% ED mass
\newcommand{\Mpl}{\ensuremath{{M_\mathrm{Pl}}}\xspace}% Planck mass
\newcommand{\Rinv} {\ensuremath{{R}^{-1}}\xspace}

% SM (still to be classified)

\newcommand{\AFB}{\ensuremath{A_\text{FB}}\xspace}
\newcommand{\wangle}{\ensuremath{\sin^{2}\theta_{\text{eff}}^\text{lept}(M^2_{\Z})}\xspace}
\newcommand{\stat}{\ensuremath{\,\text{(stat)}}\xspace}
\newcommand{\syst}{\ensuremath{\,\text{(syst)}}\xspace}
\newcommand{\thy}{\ensuremath{\,\text{(theo)}}\xspace}
\newcommand{\lum}{\ensuremath{\,\text{(lumi)}}\xspace}
\newcommand{\kt}{\ensuremath{k_{\mathrm{T}}}\xspace}

\newcommand{\BC}{\ensuremath{\mathrm{B_{c}}}\xspace}
\newcommand{\bbarc}{\ensuremath{\PQb\PAQc}\xspace}
\newcommand{\bbbar}{\ensuremath{\PQb\PAQb}\xspace}
\newcommand{\ccbar}{\ensuremath{\PQc\PAQc}\xspace}
\newcommand{\qqbar}{\ensuremath{\PQq\PAQq}\xspace}
\newcommand{\bspsiphi}{\ensuremath{\mathrm{B_s} \to \JPsi\, \phi}\xspace}
\newcommand{\EE}{\ensuremath{\Pep\Pem}\xspace}
\newcommand{\MM}{\ensuremath{\Pgmp\Pgmm}\xspace}
\newcommand{\TT}{\ensuremath{\Pgt^{+}\Pgt^{-}}\xspace}

%%%  E-gamma definitions
\newcommand{\HGG}{\ensuremath{\mathrm{H}\to\gamma\gamma}}
\newcommand{\GAMJET}{\ensuremath{\gamma + \text{jet}}}
\newcommand{\PPTOJETS}{\ensuremath{\Pp\Pp\to\text{jets}}}
\newcommand{\PPTOGG}{\ensuremath{\Pp\Pp\to\gamma\gamma}}
\newcommand{\PPTOGAMJET}{\ensuremath{\Pp\Pp\to\gamma + \text{jet}}}
\newcommand{\MH}{\ensuremath{M_{\PH}}}
\newcommand{\RNINE}{\ensuremath{R_\mathrm{9}}}
\newcommand{\DR}{\ensuremath{\Delta R}}





%%%%%%
% From Albert
%

\newcommand{\ga}{\ensuremath{\gtrsim}}
\newcommand{\la}{\ensuremath{\lesssim}}
%
\newcommand{\swsq}{\ensuremath{\sin^2\theta_\mathrm{W}}\xspace}
\newcommand{\cwsq}{\ensuremath{\cos^2\theta_\mathrm{W}}\xspace}
\newcommand{\tanb}{\ensuremath{\tan\beta}\xspace}
\newcommand{\tanbsq}{\ensuremath{\tan^{2}\beta}\xspace}
\newcommand{\sidb}{\ensuremath{\sin 2\beta}\xspace}
\newcommand{\alpS}{\ensuremath{\alpha_S}\xspace}
\newcommand{\alpt}{\ensuremath{\tilde{\alpha}}\xspace}

\newcommand{\QL}{\ensuremath{\mathrm{Q}_\mathrm{L}}\xspace}
\newcommand{\sQ}{\ensuremath{\widetilde{\mathrm{Q}}}\xspace}
\newcommand{\sQL}{\ensuremath{\widetilde{\mathrm{Q}}_\mathrm{L}}\xspace}
\newcommand{\ULC}{\ensuremath{\mathrm{U}_\mathrm{L}^\mathrm{C}}\xspace}
\newcommand{\sUC}{\ensuremath{\widetilde{\mathrm{U}}^\mathrm{C}}\xspace}
\newcommand{\sULC}{\ensuremath{\widetilde{\mathrm{U}}_\mathrm{L}^\mathrm{C}}\xspace}
\newcommand{\DLC}{\ensuremath{\mathrm{D}_\mathrm{L}^\mathrm{C}}\xspace}
\newcommand{\sDC}{\ensuremath{\widetilde{\mathrm{D}}^\mathrm{C}}\xspace}
\newcommand{\sDLC}{\ensuremath{\widetilde{\mathrm{D}}_\mathrm{L}^\mathrm{C}}\xspace}
\newcommand{\LL}{\ensuremath{\mathrm{L}_\mathrm{L}}\xspace}
\newcommand{\sL}{\ensuremath{\widetilde{\mathrm{L}}}\xspace}
\newcommand{\sLL}{\ensuremath{\widetilde{\mathrm{L}}_\mathrm{L}}\xspace}
\newcommand{\ELC}{\ensuremath{\mathrm{E}_\mathrm{L}^\mathrm{C}}\xspace}
\newcommand{\sEC}{\ensuremath{\widetilde{\mathrm{E}}^\mathrm{C}}\xspace}
\newcommand{\sELC}{\ensuremath{\widetilde{\mathrm{E}}_\mathrm{L}^\mathrm{C}}\xspace}
\newcommand{\sEL}{\ensuremath{\widetilde{\mathrm{E}}_\mathrm{L}}\xspace}
\newcommand{\sER}{\ensuremath{\widetilde{\mathrm{E}}_\mathrm{R}}\xspace}
\newcommand{\sFer}{\ensuremath{\widetilde{\mathrm{f}}}\xspace}
\newcommand{\sQua}{\ensuremath{\widetilde{\mathrm{q}}}\xspace}
\newcommand{\sUp}{\ensuremath{\widetilde{\mathrm{u}}}\xspace}
\newcommand{\suL}{\ensuremath{\widetilde{\mathrm{u}}_\mathrm{L}}\xspace}
\newcommand{\suR}{\ensuremath{\widetilde{\mathrm{u}}_\mathrm{R}}\xspace}
\newcommand{\sDw}{\ensuremath{\widetilde{\mathrm{d}}}\xspace}
\newcommand{\sdL}{\ensuremath{\widetilde{\mathrm{d}}_\mathrm{L}}\xspace}
\newcommand{\sdR}{\ensuremath{\widetilde{\mathrm{d}}_\mathrm{R}}\xspace}
\newcommand{\sTop}{\ensuremath{\widetilde{\mathrm{t}}}\xspace}
\newcommand{\stL}{\ensuremath{\widetilde{\mathrm{t}}_\mathrm{L}}\xspace}
\newcommand{\stR}{\ensuremath{\widetilde{\mathrm{t}}_\mathrm{R}}\xspace}
\newcommand{\stone}{\ensuremath{\widetilde{\mathrm{t}}_1}\xspace}
\newcommand{\sttwo}{\ensuremath{\widetilde{\mathrm{t}}_2}\xspace}
\newcommand{\sBot}{\ensuremath{\widetilde{\mathrm{b}}}\xspace}
\newcommand{\sbL}{\ensuremath{\widetilde{\mathrm{b}}_\mathrm{L}}\xspace}
\newcommand{\sbR}{\ensuremath{\widetilde{\mathrm{b}}_\mathrm{R}}\xspace}
\newcommand{\sbone}{\ensuremath{\widetilde{\mathrm{b}}_1}\xspace}
\newcommand{\sbtwo}{\ensuremath{\widetilde{\mathrm{b}}_2}\xspace}
\newcommand{\sLep}{\ensuremath{\widetilde{\mathrm{l}}}\xspace}
\newcommand{\sLepC}{\ensuremath{\widetilde{\mathrm{l}}^\mathrm{C}}\xspace}
\newcommand{\sEl}{\ensuremath{\widetilde{\mathrm{e}}}\xspace}
\newcommand{\sElC}{\ensuremath{\widetilde{\mathrm{e}}^\mathrm{C}}\xspace}
\newcommand{\seL}{\ensuremath{\widetilde{\mathrm{e}}_\mathrm{L}}\xspace}
\newcommand{\seR}{\ensuremath{\widetilde{\mathrm{e}}_\mathrm{R}}\xspace}
\newcommand{\snL}{\ensuremath{\widetilde{\nu}_L}\xspace}
\newcommand{\sMu}{\ensuremath{\widetilde{\mu}}\xspace}
\newcommand{\sNu}{\ensuremath{\widetilde{\nu}}\xspace}
\newcommand{\sTau}{\ensuremath{\widetilde{\tau}}\xspace}
\newcommand{\Glu}{\ensuremath{\mathrm{g}}\xspace}
\newcommand{\sGlu}{\ensuremath{\widetilde{\mathrm{g}}}\xspace}
\newcommand{\Wpm}{\ensuremath{\mathrm{W}^{\pm}}\xspace}
\newcommand{\sWpm}{\ensuremath{\widetilde{\mathrm{W}}^{\pm}}\xspace}
\newcommand{\Wz}{\ensuremath{\mathrm{W}^{0}}\xspace}
\newcommand{\sWz}{\ensuremath{\widetilde{\mathrm{W}}^{0}}\xspace}
\newcommand{\sWino}{\ensuremath{\widetilde{\mathrm{W}}}\xspace}
\newcommand{\Bz}{\ensuremath{\mathrm{B}^{0}}\xspace}
\newcommand{\sBz}{\ensuremath{\widetilde{\mathrm{B}}^{0}}\xspace}
\newcommand{\sBino}{\ensuremath{\widetilde{\mathrm{B}}}\xspace}
\newcommand{\Zz}{\ensuremath{\mathrm{Z}^{0}}\xspace}
\newcommand{\sZino}{\ensuremath{\widetilde{\mathrm{Z}}^{0}}\xspace}
\newcommand{\sGam}{\ensuremath{\widetilde{\gamma}}\xspace}
\newcommand{\chiz}{\ensuremath{\widetilde{\chi}^{0}}\xspace}
\newcommand{\chip}{\ensuremath{\widetilde{\chi}^{+}}\xspace}
\newcommand{\chim}{\ensuremath{\widetilde{\chi}^{-}}\xspace}
\newcommand{\chipm}{\ensuremath{\widetilde{\chi}^{\pm}}\xspace}
\newcommand{\Hone}{\ensuremath{\mathrm{H}_\mathrm{d}}\xspace}
\newcommand{\sHone}{\ensuremath{\widetilde{\mathrm{H}}_\mathrm{d}}\xspace}
\newcommand{\Htwo}{\ensuremath{\mathrm{H}_\mathrm{u}}\xspace}
\newcommand{\sHtwo}{\ensuremath{\widetilde{\mathrm{H}}_\mathrm{u}}\xspace}
\newcommand{\sHig}{\ensuremath{\widetilde{\mathrm{H}}}\xspace}
\newcommand{\sHa}{\ensuremath{\widetilde{\mathrm{H}}_\mathrm{a}}\xspace}
\newcommand{\sHb}{\ensuremath{\widetilde{\mathrm{H}}_\mathrm{b}}\xspace}
\newcommand{\sHpm}{\ensuremath{\widetilde{\mathrm{H}}^{\pm}}\xspace}
\newcommand{\hz}{\ensuremath{\mathrm{h}^{0}}\xspace}
\newcommand{\Hz}{\ensuremath{\mathrm{H}^{0}}\xspace}
\newcommand{\Az}{\ensuremath{\mathrm{A}^{0}}\xspace}
\newcommand{\Hpm}{\ensuremath{\mathrm{H}^{\pm}}\xspace}
\newcommand{\sGra}{\ensuremath{\widetilde{\mathrm{G}}}\xspace}
%
\newcommand{\mtil}{\ensuremath{\widetilde{m}}\xspace}
%
\newcommand{\rpv}{\ensuremath{\rlap{\kern.2em/}R}\xspace}
\newcommand{\LLE}{\ensuremath{LL\bar{E}}\xspace}
\newcommand{\LQD}{\ensuremath{LQ\bar{D}}\xspace}
\newcommand{\UDD}{\ensuremath{\overline{UDD}}\xspace}
\newcommand{\Lam}{\ensuremath{\lambda}\xspace}
\newcommand{\Lamp}{\ensuremath{\lambda'}\xspace}
\newcommand{\Lampp}{\ensuremath{\lambda''}\xspace}
%
\newcommand{\spinbd}[2]{\ensuremath{\bar{#1}_{\dot{#2}}}\xspace}

\endinput

\begin{document}

\begin{figure} % ZZ
 \centering
 \begin{tikzpicture} % ZZ to 4l LO
  \begin{feynman}
   \vertex (q1) {\(\boldsymbol{\Pq}\)};
   \vertex [below= 2cm of q1] (q2) {\(\boldsymbol{\Paq}\)};
   \vertex [right= 2cm of q1] (a);
   \vertex [right= 2cm of q2] (b);
   \vertex [right= 2cm of a] (c);
   \vertex [right= 2cm of b] (d);
   \vertex [right= 1.5cm of c] (e);
   \vertex [right= 1.5cm of d] (f);
   \vertex [above= 0.3cm of e] (f1) {\(\boldsymbol{\ell}\)};
   \vertex [below= 0.3cm of e] (f2) {\(\boldsymbol{\bar{\ell}}\)};
   \vertex [above= 0.3cm of f] (f3) {\(\boldsymbol{\ell}\)};
   \vertex [below= 0.3cm of f] (f4) {\(\boldsymbol{\bar{\ell}}\)};
   
   \diagram* {
    (q1) -- [fermion, very thick] (a),
    (q2) -- [anti fermion, very thick] (b),
    (a) -- [fermion, very thick] (b),
    (a) -- [boson, very thick, edge label'=\(\boldsymbol{\Z}\)] (c),
    (b) -- [boson, very thick, edge label'=\(\boldsymbol{\Z}\)] (d),
    (c) -- [fermion, very thick] (f1),
    (c) -- [anti fermion, very thick] (f2),
    (d) -- [fermion, very thick] (f3),
    (d) -- [anti fermion, very thick] (f4),
   };
  \end{feynman}
 \end{tikzpicture} \hspace{1cm}
 \begin{tikzpicture} %% ZZ to 4l NLO QCD
  \begin{feynman}
   \vertex (q1) {\(\boldsymbol{\Pq}\)};
   \vertex [below= 2cm of q1] (q2) {\(\boldsymbol{\Paq}\)};
   \vertex [right= 2cm of q1] (a);
   \vertex [right= 2cm of q2] (b);
   \vertex [right= 2cm of a] (c);
   \vertex [right= 2cm of b] (d);
   \vertex [right= 1.5cm of c] (e);
   \vertex [right= 1.5cm of d] (f);
   \vertex [above= 0.3cm of e] (f1) {\(\boldsymbol{\ell}\)};
   \vertex [below= 0.3cm of e] (f2) {\(\boldsymbol{\bar{\ell}}\)};
   \vertex [above= 0.3cm of f] (f3) {\(\boldsymbol{\ell}\)};
   \vertex [below= 0.3cm of f] (f4) {\(\boldsymbol{\bar{\ell}}\)};
   \vertex [right= 0.8cm of q1] (g1);
   \vertex [below= 0.7cm of a] (g2);
   
   \diagram* {
    (q1) -- [fermion, very thick] (a),
    (q2) -- [anti fermion, very thick] (b),
    (a) -- [fermion, very thick] (b),
    (a) -- [boson, very thick, edge label'=\(\boldsymbol{\Z}\)] (c),
    (b) -- [boson, very thick, edge label'=\(\boldsymbol{\Z}\)] (d),
    (c) -- [fermion, very thick] (f1),
    (c) -- [anti fermion, very thick] (f2),
    (d) -- [fermion, very thick] (f3),
    (d) -- [anti fermion, very thick] (f4),
    (g1) -- [gluon, very thick, bend right, edge label'=\(\boldsymbol{g}\)] (g2),
   };
  \end{feynman}
 \end{tikzpicture}  \vspace{1cm}
 
 \begin{tikzpicture} %% ZZ to 4l NLO only EW
  \begin{feynman}
   \vertex (q1) {\(\boldsymbol{\Pq}\)};
   \vertex [below= 2cm of q1] (q2) {\(\boldsymbol{\Paq}\)};
   \vertex [right= 2cm of q1] (a);
   \vertex [right= 2cm of q2] (b);
   \vertex [right= 2cm of a] (c);
   \vertex [right= 2cm of b] (d);
   \vertex [right= 1.5cm of c] (e);
   \vertex [right= 1.5cm of d] (f);
   \vertex [above= 0.3cm of e] (f1) {\(\boldsymbol{\ell}\)};
   \vertex [below= 0.3cm of e] (f2) {\(\boldsymbol{\bar{\ell}}\)};
   \vertex [above= 0.3cm of f] (f3) {\(\boldsymbol{\ell}\)};
   \vertex [below= 0.3cm of f] (f4) {\(\boldsymbol{\bar{\ell}}\)};
   \vertex [right= 0.8cm of q2] (V1);
   \vertex [above= 0.7cm of b] (V2);
   
   \diagram* {
    (q1) -- [fermion, very thick] (a),
    (q2) -- [anti fermion, very thick] (b),
    (a) -- [fermion, very thick] (b),
    (a) -- [boson, very thick, edge label'=\(\boldsymbol{\Z}\)] (c),
    (b) -- [boson, very thick, edge label'=\(\boldsymbol{\Z}\)] (d),
    (c) -- [fermion, very thick] (f1),
    (c) -- [anti fermion, very thick] (f2),
    (d) -- [fermion, very thick] (f3),
    (d) -- [anti fermion, very thick] (f4),
    (V1) -- [boson, very thick, bend left, edge label=\(\boldsymbol{\Z/\W/\gamma}\)] (V2),
   };
  \end{feynman}
 \end{tikzpicture} \hspace{1cm}
 \begin{tikzpicture} %% ZZ to 4l NLO EW and QCD
  \begin{feynman}
   \vertex (q1) {\(\boldsymbol{\Pq}\)};
   \vertex [below= 2cm of q1] (q2) {\(\boldsymbol{\Paq}\)};
   \vertex [right= 2cm of q1] (a);
   \vertex [right= 2cm of q2] (b);
   \vertex [right= 2cm of a] (c);
   \vertex [right= 2cm of b] (d);
   \vertex [right= 1.5cm of c] (e);
   \vertex [right= 1.5cm of d] (f);
   \vertex [above= 0.3cm of e] (f1) {\(\boldsymbol{\ell}\)};
   \vertex [below= 0.3cm of e] (f2) {\(\boldsymbol{\bar{\ell}}\)};
   \vertex [above= 0.3cm of f] (f3) {\(\boldsymbol{\ell}\)};
   \vertex [below= 0.3cm of f] (f4) {\(\boldsymbol{\bar{\ell}}\)};
   \vertex [right= 0.8cm of q1] (V1);
   \vertex [below= 0.7cm of a] (V2);
   \vertex [right= 1.2cm of q2] (g1);
   \vertex [below= 1cm of g1] (g2);
   \vertex [right= 1.3cm of g2] (g3);
   
   \diagram* {
    (q1) -- [fermion, very thick] (a),
    (q2) -- [anti fermion, very thick] (b),
    (a) -- [fermion, very thick] (b),
    (a) -- [boson, very thick, edge label'=\(\boldsymbol{\Z}\)] (c),
    (b) -- [boson, very thick, edge label'=\(\boldsymbol{\Z}\)] (d),
    (c) -- [fermion, very thick] (f1),
    (c) -- [anti fermion, very thick] (f2),
    (d) -- [fermion, very thick] (f3),
    (d) -- [anti fermion, very thick] (f4),
    (V1) -- [boson, very thick, bend right, edge label'=\(\boldsymbol{\Z/\W/\gamma}\)] (V2),
    (g1) -- [gluon, very thick, edge label'=\(\boldsymbol{g}\)] (g3),
   };
  \end{feynman}
 \end{tikzpicture}

 \caption{Clockwise from upper left: ZZ production at leading order; ZZ production at NLO in QCD; ZZ production at NLO in both QCD and EW; ZZ production at NLO only in EW.} \label{fig:ZZto4l}
\end{figure}

\begin{figure} % WZ
 \centering
 \begin{tikzpicture} % WZ at LO (s-channel)
  \begin{feynman}
   \vertex (q1) {\(\boldsymbol{\Pq}\)};
   \vertex [below= 3cm of q1] (q2) {\(\boldsymbol{\Paq'}\)};
   \vertex [right= 1.6cm of q1] (a1);
   \vertex [below= 1.5cm of a1] (a2);
   \vertex [right= 1.6cm of a2] (b1);
   \vertex [right= 1.6cm of b1] (c1);
   \vertex [above= 1.3cm of c1] (c2);
   \vertex [below= 1.3cm of c1] (c3);
   \vertex [right= 1cm of c2] (d1);
   \vertex [right= 1cm of c3] (d2);
   \vertex [above= 0.3cm of d1] (f1) {\(\boldsymbol{\ell}\)};
   \vertex [below= 0.3cm of d1] (f2) {\(\boldsymbol{\bar{\ell}}\)};
   \vertex [above= 0.3cm of d2] (f3) {\(\boldsymbol{\ell}\)};
   \vertex [below= 0.3cm of d2] (f4) {\(\boldsymbol{\bar{\nu_\ell}}\)};
   
   \diagram* {
    (q1) -- [fermion, very thick] (a2),
    (q2) -- [anti fermion, very thick] (a2),
    (a2) -- [boson, very thick, edge label'=\(\boldsymbol{\PWm}\)] (b1),
    (b1) -- [boson, very thick, edge label=\(\boldsymbol{\Z}\)] (c2),
    (b1) -- [boson, very thick, edge label=\(\boldsymbol{\PWm}\)] (c3),
    (c2) -- [fermion, very thick] (f1),
    (c2) -- [anti fermion, very thick] (f2),
    (c3) -- [fermion, very thick] (f3),
    (c3) -- [anti fermion, very thick] (f4),
   };
  \end{feynman}
 \end{tikzpicture} \hspace{1cm}
 \begin{tikzpicture} % WZ at LO (t-channel)
  \begin{feynman}
   \vertex (q1) {\(\boldsymbol{\Pq}\)};
   \vertex [below= 2cm of q1] (q2) {\(\boldsymbol{\Paq'}\)};
   \vertex [right= 2cm of q1] (a);
   \vertex [right= 2cm of q2] (b);
   \vertex [right= 2cm of a] (c);
   \vertex [right= 2cm of b] (d);
   \vertex [right= 1.5cm of c] (e);
   \vertex [right= 1.5cm of d] (f);
   \vertex [above= 0.3cm of e] (f1) {\(\boldsymbol{\ell}\)};
   \vertex [below= 0.3cm of e] (f2) {\(\boldsymbol{\bar{\ell}}\)};
   \vertex [above= 0.3cm of f] (f3) {\(\boldsymbol{\nu_\ell}\)};
   \vertex [below= 0.3cm of f] (f4) {\(\boldsymbol{\bar{\ell}}\)};
   
   \diagram* {
    (q1) -- [fermion, very thick] (a),
    (q2) -- [anti fermion, very thick] (b),
    (a) -- [fermion, very thick] (b),
    (a) -- [boson, very thick, edge label'=\(\boldsymbol{\Z}\)] (c),
    (b) -- [boson, very thick, edge label'=\(\boldsymbol{\PWp}\)] (d),
    (c) -- [fermion, very thick] (f1),
    (c) -- [anti fermion, very thick] (f2),
    (d) -- [fermion, very thick] (f3),
    (d) -- [anti fermion, very thick] (f4),
   };
  \end{feynman}
 \end{tikzpicture}  
 \caption{Leading order WZ production mechanisms in the $s$-channel and the $t$-channel.} \label{fig:WZLO}
\end{figure}

\begin{figure} % WZ NLO EW
 \centering
 \begin{tikzpicture} % WZ at NLO EW (s-channel) internal loop
  \begin{feynman}
   \vertex (q1) {\(\boldsymbol{\Pq}\)};
   \vertex [below= 3cm of q1] (q2) {\(\boldsymbol{\Paq'}\)};
   \vertex [right= 1.5cm of q1] (a1);
   \vertex [below= 1.5cm of a1] (a2);
   \vertex [right= 1.6cm of a2] (b1);
   \vertex [right= 1.6cm of b1] (c1);
   \vertex [above= 1.3cm of c1] (c2);
   \vertex [below= 1.3cm of c1] (c3);
   \vertex [right= 1cm of c2] (d1);
   \vertex [right= 1cm of c3] (d2);
   \vertex [above= 0.3cm of d1] (f1) {\(\boldsymbol{\ell}\)};
   \vertex [below= 0.3cm of d1] (f2) {\(\boldsymbol{\bar{\ell}}\)};
   \vertex [above= 0.3cm of d2] (f3) {\(\boldsymbol{\ell}\)};
   \vertex [below= 0.3cm of d2] (f4) {\(\boldsymbol{\bar{\nu_\ell}}\)};
   \vertex [below right= 0.9cm of q1] (V1);
   \vertex [above right= 0.9cm of q2] (V2);
   
   \diagram* {
    (q1) -- [fermion, very thick] (a2),
    (q2) -- [anti fermion, very thick] (a2),
    (a2) -- [boson, very thick, edge label'=\(\boldsymbol{\PWm}\)] (b1),
    (b1) -- [boson, very thick, edge label=\(\boldsymbol{\Z}\)] (c2),
    (b1) -- [boson, very thick, edge label=\(\boldsymbol{\PWm}\)] (c3),
    (c2) -- [fermion, very thick] (f1),
    (c2) -- [anti fermion, very thick] (f2),
    (c3) -- [fermion, very thick] (f3),
    (c3) -- [anti fermion, very thick] (f4),
    (V1) -- [boson, very thick, edge label'=\(\boldsymbol{\Z/\W/\gamma}\)] (V2),
   };
  \end{feynman}
 \end{tikzpicture} \hspace{1cm}
 \begin{tikzpicture} % WZ at NLO (t-channel) internal loop
  \begin{feynman}
   \vertex (q1) {\(\boldsymbol{\Pq}\)};
   \vertex [below= 2cm of q1] (q2) {\(\boldsymbol{\Paq'}\)};
   \vertex [right= 2cm of q1] (a);
   \vertex [right= 2cm of q2] (b);
   \vertex [right= 2cm of a] (c);
   \vertex [right= 2cm of b] (d);
   \vertex [right= 1.5cm of c] (e);
   \vertex [right= 1.5cm of d] (f);
   \vertex [above= 0.3cm of e] (f1) {\(\boldsymbol{\ell}\)};
   \vertex [below= 0.3cm of e] (f2) {\(\boldsymbol{\bar{\ell}}\)};
   \vertex [above= 0.3cm of f] (f3) {\(\boldsymbol{\nu_\ell}\)};
   \vertex [below= 0.3cm of f] (f4) {\(\boldsymbol{\bar{\ell}}\)};
   \vertex [right= 0.8cm of q2] (V1);
   \vertex [above= 0.7cm of b] (V2);
   
   \diagram* {
    (q1) -- [fermion, very thick] (a),
    (q2) -- [anti fermion, very thick] (b),
    (a) -- [fermion, very thick] (b),
    (a) -- [boson, very thick, edge label'=\(\boldsymbol{\Z}\)] (c),
    (b) -- [boson, very thick, edge label'=\(\boldsymbol{\W^+}\)] (d),
    (c) -- [fermion, very thick] (f1),
    (c) -- [anti fermion, very thick] (f2),
    (d) -- [fermion, very thick] (f3),
    (d) -- [anti fermion, very thick] (f4),
    (V1) -- [boson, very thick, bend left, edge label=\(\boldsymbol{\Z/\W/\gamma}\)] (V2),
   };
  \end{feynman}
 \end{tikzpicture} \vspace{1cm}

 \begin{tikzpicture} % WZ at NLO EW (s-channel) y-induced
  \begin{feynman}
   \vertex (y1) {\(\boldsymbol{\gamma}\)};
   \vertex [below= 3cm of y1] (q2) {\(\boldsymbol{\Pq}\)};
   \vertex [right= 1.5cm of q1] (a1);
   \vertex [below= 1.5cm of a1] (a2);
   \vertex [right= 1.6cm of a2] (b1);
   \vertex [right= 1.6cm of b1] (c1);
   \vertex [above= 1.3cm of c1] (c2);
   \vertex [below= 1.3cm of c1] (c3);
   \vertex [right= 1cm of c2] (d1);
   \vertex [right= 1cm of c3] (d2);
   \vertex [above= 0.3cm of d1] (f1) {\(\boldsymbol{\ell}\)};
   \vertex [below= 0.3cm of d1] (f2) {\(\boldsymbol{\bar{\ell}}\)};
   \vertex [above= 0.3cm of d2] (f3) {\(\boldsymbol{\ell}\)};
   \vertex [below= 0.3cm of d2] (f4) {\(\boldsymbol{\bar{\nu_\ell}}\)};
   \vertex [below right= 1.2cm of y1] (y2);
   \vertex [right= 1cm of y2] (fsrq1);
   \vertex [above= 0.6cm of fsrq1] (fsrq2) {\(\boldsymbol{\Pq'}\)};
   
   \diagram* {
    (y1) -- [boson, very thick] (y2),
    (y2) -- [anti fermion, very thick] (a2),
    (q2) -- [fermion, very thick] (a2),
    (a2) -- [boson, very thick, edge label'=\(\boldsymbol{\PWm}\)] (b1),
    (b1) -- [boson, very thick, edge label=\(\boldsymbol{\Z}\)] (c2),
    (b1) -- [boson, very thick, edge label=\(\boldsymbol{\PWm}\)] (c3),
    (c2) -- [fermion, very thick] (f1),
    (c2) -- [anti fermion, very thick] (f2),
    (c3) -- [fermion, very thick] (f3),
    (c3) -- [anti fermion, very thick] (f4),
    (y2) -- [fermion, very thick] (fsrq2),
   };
  \end{feynman}
 \end{tikzpicture} \hspace{1cm}
 \begin{tikzpicture} % WZ at NLO EW (t-channel) y-induced
  \begin{feynman}
   \vertex (y1) {\(\boldsymbol{\gamma}\)};
   \vertex [right= 1cm of y1] (y2);
   \vertex [below= 0.3cm of y2] (y3);
   \vertex [right= 1.8cm of y3] (fsrq1);
   \vertex [above= 0.2cm of fsrq1] (fsrq2) {\(\boldsymbol{\Pq'}\)};
   \vertex [below= 3.6cm of y1] (q2) {\(\boldsymbol{\Pq}\)};
   \vertex [right= 1cm of y3] (a1);
   \vertex [below= 0.5cm of a1] (a2);
   \vertex [right= 2cm of q2] (a3);
   \vertex [above= 0.8cm of a3] (a4);
   \vertex [right= 2cm of a2] (c);
   \vertex [right= 2cm of a4] (d);
   \vertex [right= 1.5cm of c] (e);
   \vertex [right= 1.5cm of d] (f);
   \vertex [above= 0.3cm of e] (f1) {\(\boldsymbol{\ell}\)};
   \vertex [below= 0.3cm of e] (f2) {\(\boldsymbol{\bar{\ell}}\)};
   \vertex [above= 0.3cm of f] (f3) {\(\boldsymbol{\nu_\ell}\)};
   \vertex [below= 0.3cm of f] (f4) {\(\boldsymbol{\bar{\ell}}\)};
   
   \diagram* {
    (y1) -- [boson, very thick] (y3),
    (y3) -- [anti fermion, very thick] (a2),
    (q2) -- [fermion, very thick] (a4),
    (a2) -- [anti fermion, very thick] (a4),
    (a2) -- [boson, very thick, edge label'=\(\boldsymbol{\Z}\)] (c),
    (a4) -- [boson, very thick, edge label'=\(\boldsymbol{\PWp}\)] (d),
    (c) -- [fermion, very thick] (f1),
    (c) -- [anti fermion, very thick] (f2),
    (d) -- [fermion, very thick] (f3),
    (d) -- [anti fermion, very thick] (f4),
    (y3) -- [fermion, very thick] (fsrq2),
   };
  \end{feynman}
 \end{tikzpicture}  
 

 \caption{WZ production at NLO in EW by internal loop processes (upper row) and photon-quark induced processes (lower row).} \label{fig:WZNLO}
\end{figure}

\begin{figure} % BSM models
 \centering
 \begin{tikzpicture} % ZH(inv)
  \begin{feynman}
   \vertex (q1) {\(\boldsymbol{\Pq}\)};
   \vertex [below= 3cm of q1] (q2) {\(\boldsymbol{\Paq}\)};
   \vertex [right= 1.6cm of q1] (a1);
   \vertex [below= 1.5cm of a1] (a2);
   \vertex [right= 1.6cm of a2] (b1);
   \vertex [right= 1.6cm of b1] (c1);
   \vertex [above= 1.3cm of c1] (c2);
   \vertex [below= 1.3cm of c1] (c3);
   \vertex [right= 1cm of c2] (d1);
   \vertex [right= 1cm of c3] (d2);
   \vertex [above= 0.3cm of d1] (f1) {\(\boldsymbol{\ell}\)};
   \vertex [below= 0.3cm of d1] (f2) {\(\boldsymbol{\bar{\ell}}\)};
   \vertex [above= 0.3cm of d2] (f3) {\(\boldsymbol{\chi}\)};
   \vertex [below= 0.3cm of d2] (f4) {\(\boldsymbol{\bar{\chi}}\)};
   
   \diagram* {
    (q1) -- [fermion, very thick] (a2),
    (q2) -- [anti fermion, very thick] (a2),
    (a2) -- [boson, very thick, edge label'=\(\boldsymbol{\Z^*}\)] (b1),
    (b1) -- [boson, very thick, edge label=\(\boldsymbol{\Z}\)] (c2),
    (b1) -- [scalar, very thick, edge label=\(\boldsymbol{\Hi}\)] (c3),
    (c2) -- [fermion, very thick] (f1),
    (c2) -- [anti fermion, very thick] (f2),
    (c3) -- [fermion, very thick] (f3),
    (c3) -- [anti fermion, very thick] (f4),
   };
  \end{feynman}
 \end{tikzpicture} \hspace{1cm}
 \begin{tikzpicture} % Simplified model spin-0 mediator
  \begin{feynman}
   \vertex (g1) {\(\boldsymbol{}\)};
   \vertex [below= 3.6cm of g1] (g2) {\(\boldsymbol{}\)};
   \vertex [right= 1.4cm of g1] (a1);
   \vertex [right= 1.4cm of g2] (a2);
   \vertex [below= 0.8cm of a1] (a3);
   \vertex [above= 0.8cm of a2] (a4);
   \vertex [right= 1.4cm of a3] (b1);
   \vertex [right= 1.4cm of a4] (b2);
   \vertex [right= 1.4cm of b1] (c1);
   \vertex [right= 1.4cm of b2] (c2);
   \vertex [right= 1cm of c1] (d1);
   \vertex [right= 1cm of c2] (d2);
   \vertex [above= 0.3cm of d1] (f1) {\(\boldsymbol{\ell}\)};
   \vertex [below= 0.3cm of d1] (f2) {\(\boldsymbol{\bar{\ell}}\)};
   \vertex [above= 0.3cm of d2] (f3) {\(\boldsymbol{\chi}\)};
   \vertex [below= 0.3cm of d2] (f4) {\(\boldsymbol{\bar{\chi}}\)};
   \vertex [below = 0.2cm of b2] (dm1) {\(\boldsymbol{g_\Pq}\)};
   \vertex [below = 0.2cm of c2] (dm2) {\(\boldsymbol{g_\mathrm{DM}}\)};
   
   \diagram* {
    (g1) -- [gluon, very thick] (a3),
    (g2) -- [gluon, very thick] (a4),
    (a3) -- [fermion, very thick, edge label'=\(\boldsymbol{\bar{\Top}}\)] (a4) -- [fermion, very thick, edge label=\(\boldsymbol{\Top}\)] (b2) -- [fermion, very thick, edge label=\(\boldsymbol{\bar{\Top}}\)] (b1) -- [fermion, very thick, edge label'=\(\boldsymbol{\Top}\)] (a3),
    (b1) -- [boson, very thick, edge label=\(\boldsymbol{\Z}\)] (c1),
    (b2) -- [scalar, very thick, edge label=\(\boldsymbol{\phi}\)] (c2),
    (c1) -- [fermion, very thick] (f1),
    (c1) -- [anti fermion, very thick] (f2),
    (c2) -- [fermion, very thick] (f3),
    (c2) -- [anti fermion, very thick] (f4),
   };
  \draw[fill=blue,line width=0pt] (b2) circle(1.5mm);
  \draw[fill=violet,line width=0pt] (c2) circle(1.5mm);
  \end{feynman}
 \end{tikzpicture} \vspace{1cm}
 
 \begin{tikzpicture} % Simplified model spin-1 mediator
  \begin{feynman}
   \vertex (q1) {\(\boldsymbol{\Pq}\)};
   \vertex [below= 3.6cm of q1] (q2) {\(\boldsymbol{\Paq}\)};
   \vertex [right= 2cm of q1] (a1);
   \vertex [right= 2cm of q2] (a2);
   \vertex [below= 0.8cm of a1] (a3);
   \vertex [above= 0.8cm of a2] (a4);
   \vertex [right= 2cm of a3] (b1);
   \vertex [right= 2cm of a4] (b2);
   \vertex [right= 1.5cm of b1] (c1);
   \vertex [right= 1.5cm of b2] (c2);
   \vertex [above= 0.3cm of c1] (f1) {\(\boldsymbol{\ell}\)};
   \vertex [below= 0.3cm of c1] (f2) {\(\boldsymbol{\bar{\ell}}\)};
   \vertex [below= 0.2cm of c2] (U) {\(\boldsymbol{\mathcal{U}/\mathcal{G}}\)};
   
   \diagram* {
    (q1) -- [fermion, very thick] (a3),
    (a3) -- [fermion, very thick] (a4),
    (q2) -- [anti fermion, very thick] (a4),
    (a3) -- [boson, very thick, edge label=\(\boldsymbol{\Z}\)] (b1),
    (a4) -- [ghost, very thick] (U),
    (b1) -- [fermion, very thick] (f1),
    (b1) -- [anti fermion, very thick] (f2),
   };
  \draw[fill=black,line width=0pt] (a4) circle(2.2mm);
  \draw[pattern=north east lines, pattern color=white] (a4) circle(2.2mm);
  \end{feynman}
 \end{tikzpicture} \hspace{1cm}  
 \begin{tikzpicture} % ADD/unparticles
  \begin{feynman}
   \vertex (q1) {\(\boldsymbol{\Pq}\)};
   \vertex [below= 3.6cm of q1] (q2) {\(\boldsymbol{\Paq}\)};
   \vertex [right= 2cm of q1] (a1);
   \vertex [right= 2cm of q2] (a2);
   \vertex [below= 0.8cm of a1] (a3);
   \vertex [above= 0.8cm of a2] (a4);
   \vertex [right= 2cm of a3] (b1);
   \vertex [right= 2cm of a4] (b2);
   \vertex [right= 1.5cm of b1] (c1);
   \vertex [right= 1.5cm of b2] (c2);
   \vertex [above= 0.3cm of c1] (f1) {\(\boldsymbol{\ell}\)};
   \vertex [below= 0.3cm of c1] (f2) {\(\boldsymbol{\bar{\ell}}\)};
   \vertex [above= 0.3cm of c2] (f3) {\(\boldsymbol{\chi}\)};
   \vertex [below= 0.3cm of c2] (f4) {\(\boldsymbol{\bar{\chi}}\)};
   \vertex [below = 0.2cm of a4] (dm1) {\(\boldsymbol{g_\Pq}\)};
   \vertex [below = 0.2cm of b2] (dm2) {\(\boldsymbol{g_\mathrm{DM}}\)};
   
   \diagram* {
    (q1) -- [fermion, very thick] (a3),
    (a3) -- [fermion, very thick] (a4),
    (q2) -- [anti fermion, very thick] (a4),
    (a3) -- [boson, very thick, edge label=\(\boldsymbol{\Z}\)] (b1),
    (a4) -- [boson, very thick, edge label=\(\boldsymbol{\mathcal{A}}\)] (b2),
    (b1) -- [fermion, very thick] (f1),
    (b1) -- [anti fermion, very thick] (f2),
    (b2) -- [fermion, very thick] (f3),
    (b2) -- [anti fermion, very thick] (f4),
   };
  \draw[fill=blue,line width=0pt] (a4) circle(1.5mm);
  \draw[fill=violet,line width=0pt] (b2) circle(1.5mm);
  \end{feynman}
 \end{tikzpicture}
 \caption{Some diagrams beyond the Standard Model in which are produced two charged leptons and missing energy. Clockwise from upper left: associated production of an invisible Higgs boson; gluon-induced production of a Z boson and a massive spin-0 dark matter mediator via top-quark loop; production of a Z boson and a massive spin-1 dark matter mediator; production of a Z boson in association with gravitons (ADD model) or unparticles.} \label{fig:BSMdiagrams}
 \end{figure}

\begin{figure} % 3GC
 \centering
 \begin{tikzpicture} % WWZ
  \begin{feynman}
   \vertex (a2) {\(\boldsymbol{\W^\pm}\)};
   \vertex [right= 1.5cm of a2] (b1);
   \vertex [right= 1.1cm of b1] (c1);
   \vertex [above= 0.9cm of c1] (c2) {\(\boldsymbol{\Z}\)};
   \vertex [below= 0.9cm of c1] (c3) {\(\boldsymbol{\W^\pm}\)};
   
   \diagram* {
    (a2) -- [boson, very thick] (b1),
    (b1) -- [boson, very thick] (c2),
    (b1) -- [boson, very thick] (c3),
   };
  \end{feynman}
 \end{tikzpicture} \hspace{0.5cm}
 \begin{tikzpicture} % WWy
  \begin{feynman}
   \vertex (a2) {\(\boldsymbol{\W^\pm}\)};
   \vertex [right= 1.5cm of a2] (b1);
   \vertex [right= 1.1cm of b1] (c1);
   \vertex [above= 0.9cm of c1] (c2) {\(\boldsymbol{\gamma}\)};
   \vertex [below= 0.9cm of c1] (c3) {\(\boldsymbol{\W^\pm}\)};
   
   \diagram* {
    (a2) -- [boson, very thick] (b1),
    (b1) -- [boson, very thick] (c2),
    (b1) -- [boson, very thick] (c3),
   };
  \end{feynman}
 \end{tikzpicture} 
 \caption{Triple gauge boson vertices in the electroweak theory.} \label{fig:3GC}
\end{figure}

\begin{figure} % 4GC
 \centering
 \begin{tikzpicture} % WWWW
  \begin{feynman}
   \vertex (a1);
   \vertex [above= 0.8cm of a1] (a2) {\(\boldsymbol{\W}\)};
   \vertex [below= 0.8cm of a1] (a3) {\(\boldsymbol{\W}\)};
   \vertex [right= 1.2cm of a1] (b1);
   \vertex [right= 1.2cm of b1] (c1);
   \vertex [above= 0.8cm of c1] (c2) {\(\boldsymbol{\W}\)};
   \vertex [below= 0.8cm of c1] (c3) {\(\boldsymbol{\W}\)};
   
   \diagram* {
    (a2) -- [boson, very thick] (b1),
    (a3) -- [boson, very thick] (b1),
    (b1) -- [boson, very thick] (c2),
    (b1) -- [boson, very thick] (c3),
   };
  \end{feynman}
 \end{tikzpicture} \hspace{0.5cm}
 \begin{tikzpicture} % WWyy
  \begin{feynman}
   \vertex (a1);
   \vertex [above= 0.8cm of a1] (a2) {\(\boldsymbol{\W}\)};
   \vertex [below= 0.8cm of a1] (a3) {\(\boldsymbol{\W}\)};
   \vertex [right= 1.2cm of a1] (b1);
   \vertex [right= 1.2cm of b1] (c1);
   \vertex [above= 0.8cm of c1] (c2) {\(\boldsymbol{\gamma}\)};
   \vertex [below= 0.8cm of c1] (c3) {\(\boldsymbol{\gamma}\)};
   
   \diagram* {
    (a2) -- [boson, very thick] (b1),
    (a3) -- [boson, very thick] (b1),
    (b1) -- [boson, very thick] (c2),
    (b1) -- [boson, very thick] (c3),
   };
  \end{feynman}
 \end{tikzpicture} \hspace{0.5cm}
 \begin{tikzpicture} % WWyZ
  \begin{feynman}
   \vertex (a1);
   \vertex [above= 0.8cm of a1] (a2) {\(\boldsymbol{\W}\)};
   \vertex [below= 0.8cm of a1] (a3) {\(\boldsymbol{\W}\)};
   \vertex [right= 1.2cm of a1] (b1);
   \vertex [right= 1.2cm of b1] (c1);
   \vertex [above= 0.8cm of c1] (c2) {\(\boldsymbol{\gamma}\)};
   \vertex [below= 0.8cm of c1] (c3) {\(\boldsymbol{\Z}\)};
   
   \diagram* {
    (a2) -- [boson, very thick] (b1),
    (a3) -- [boson, very thick] (b1),
    (b1) -- [boson, very thick] (c2),
    (b1) -- [boson, very thick] (c3),
   };
  \end{feynman}
 \end{tikzpicture} \hspace{0.5cm}
 \begin{tikzpicture} % WWZZ
  \begin{feynman}
   \vertex (a1);
   \vertex [above= 0.8cm of a1] (a2) {\(\boldsymbol{\W}\)};
   \vertex [below= 0.8cm of a1] (a3) {\(\boldsymbol{\W}\)};
   \vertex [right= 1.2cm of a1] (b1);
   \vertex [right= 1.2cm of b1] (c1);
   \vertex [above= 0.8cm of c1] (c2) {\(\boldsymbol{\Z}\)};
   \vertex [below= 0.8cm of c1] (c3) {\(\boldsymbol{\Z}\)};
   
   \diagram* {
    (a2) -- [boson, very thick] (b1),
    (a3) -- [boson, very thick] (b1),
    (b1) -- [boson, very thick] (c2),
    (b1) -- [boson, very thick] (c3),
   };
  \end{feynman}
 \end{tikzpicture}
 \caption{Non-Abelian quartic gauge couplings in the electroweak theory.} \label{fig:4GC}
\end{figure}

\begin{figure} % Higgs couplings
 \centering
\begin{tikzpicture} % VVH
  \begin{feynman}
   \vertex (a2) {\(\boldsymbol{\mathrm{V}}\)};
   \vertex [right= 1.3cm of a2] (b1);
   \vertex [right= 1.1cm of b1] (c1);
   \vertex [above= 0.9cm of c1] (c2) {\(\boldsymbol{\Hi}\)};
   \vertex [below= 0.9cm of c1] (c3) {\(\boldsymbol{\mathrm{V}}\)};
   
   \diagram* {
    (a2) -- [boson, very thick] (b1),
    (b1) -- [scalar, very thick] (c2),
    (b1) -- [boson, very thick] (c3),
   };
  \end{feynman}
 \end{tikzpicture} \hspace{0.5cm}
 \begin{tikzpicture} % Hff
  \begin{feynman}
   \vertex (a2) {\(\boldsymbol{f}\)};
   \vertex [right= 1.3cm of a2] (b1);
   \vertex [right= 1.1cm of b1] (c1);
   \vertex [above= 0.9cm of c1] (c2) {\(\boldsymbol{\Hi}\)};
   \vertex [below= 0.9cm of c1] (c3) {\(\boldsymbol{f}\)};
   
   \diagram* {
    (a2) -- [fermion, very thick] (b1),
    (b1) -- [scalar, very thick] (c2),
    (b1) -- [fermion, very thick] (c3),
   };
  \end{feynman}
 \end{tikzpicture} \hspace{0.5cm}
 \begin{tikzpicture} % HHH
  \begin{feynman}
   \vertex (a2) {\(\boldsymbol{\Hi}\)};
   \vertex [right= 1.3cm of a2] (b1);
   \vertex [right= 1.1cm of b1] (c1);
   \vertex [above= 0.9cm of c1] (c2) {\(\boldsymbol{\Hi}\)};
   \vertex [below= 0.9cm of c1] (c3) {\(\boldsymbol{\Hi}\)};
   
   \diagram* {
    (a2) -- [scalar, very thick] (b1),
    (b1) -- [scalar, very thick] (c2),
    (b1) -- [scalar, very thick] (c3),
   };
  \end{feynman}
 \end{tikzpicture} \hspace{0.5cm}
 \begin{tikzpicture} % HHHH
  \begin{feynman}
   \vertex (a1);
   \vertex [above= 0.8cm of a1] (a2) {\(\boldsymbol{\Hi}\)};
   \vertex [below= 0.8cm of a1] (a3) {\(\boldsymbol{\Hi}\)};
   \vertex [right= 1.2cm of a1] (b1);
   \vertex [right= 1.2cm of b1] (c1);
   \vertex [above= 0.8cm of c1] (c2) {\(\boldsymbol{\Hi}\)};
   \vertex [below= 0.8cm of c1] (c3) {\(\boldsymbol{\Hi}\)};
   
   \diagram* {
    (a2) -- [scalar, very thick] (b1),
    (a3) -- [scalar, very thick] (b1),
    (b1) -- [scalar, very thick] (c2),
    (b1) -- [scalar, very thick] (c3),
   };
  \end{feynman}
 \end{tikzpicture}
 \caption{Higgs boson couplings. V is a $\W^\pm$ or $\Z$ boson. $f$ are fermions.}
\end{figure}

\end{document}
