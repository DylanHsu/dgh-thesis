\chapter{Introduction}
At the energy scales of everyday life in the $21^{\mathrm{st}}$ century, the preponderance of the human experience is described by the electromagnetic and gravitational forces.
As it fights gravitational attraction walking on the ground of the Earth,
the human body is held together by chemical bonds having energy of several electron volts (eV),
and it observes its environment by absorbing photons of comparable energies.
The saga of the evolution of the Universe far surpasses these energy scales.
According to the theory of the Big Bang, it began approximately 13.8 billion years ago.
The energy of the fundamental interactions occurring in the first few seconds after the Big Bang was much higher.

The Standard Model of particle physics is a unified theory of three of the four fundamental forces of nature.
It describes the strong and weak nuclear forces alongside the electromagnetic force.
These interactions occur among elementary particles, and the forces amongst them are carried by gauge bosons.
The Standard Model is not yet a Grand Unified Theory, which would describe a single, unified force at the time of the Big Bang.
Besides unification, there are several unanswered questions.
What is the internal structure of the proton, one of the building blocks of all chemical elements?
Is there additional undetected dark matter in our Universe?
Are there more undiscovered fundamental particles?
Are there more than four spacetime dimensions?

Compared to the electromagnetic force, the weak nuclear force is stronger, but only at very short ranges ($< 10^{{-}18}$ m).
It is mediated by the extremely massive $\mathrm{W^\pm}$ and Z bosons.
By the mass-energy equivalence principle, they weigh around 80 and 91 GeV (giga-electron volts), respectively.
Compared to the weak force, the strong nuclear force is even more powerful and short-ranged.
It is mediated by the massless gluon.
It overcomes the electrostatic repulsion of nucleons to hold together nuclei at distances measured in femtometers.

To study the strong and weak nuclear forces, it is necessary to observe
the products of interactions on the energy scales of hundreds of GeV or even TeV (tera-electron volts).
In order to produce a large multitude of these interactions,
machines such as the Large Hadron Collider (LHC) are built.
The purpose of the LHC is to collide protons having energies of several TeV.
These supercolliders are financial undertakings amounting to billions of US dollars.
Today, their construction and operation are directed by international organizations such as CERN (European Center for Nuclear Research).
Meanwhile, to observe those interactions, detector systems such as the CMS Detector are built to capture and measure most of the interaction products.
These function as very fast cameras which provide a detailed picture of what particles came out of a particular proton-proton collision event.

To date, this work contains the most precise measurement of the production of Z bosons via proton collisions at center-of-mass energy 13 TeV.
Also presented is a search for the production of invisible particles at the LHC, which could
point to dark matter, extra spacetime dimensions, or some other new phenomenon.
Both results make use of the experimentally clean signature where a Z boson decays leptonically into an electron-positron or muon-antimuon pair\footnote{Strictly speaking, the term ``lepton'' refers to electrons, muons, taus, and neutrinos. In experimental high-energy physics, it is frequently used to refer to the first two generations of charged leptons: electrons and muons.}.

\section{Differential measurements of gauge bosons}
The production of charged lepton pairs via the Drell--Yan (DY) process 
is essential for the physics program of the CERN LHC.  
The large cross section and clean experimental signature %,
%two leptons with large transverse momentum, 
provide important precision
tests of the standard model (SM),
as well as constraints on the 
%The DY process has high sensitivity to 
parton distribution functions (PDFs) of the proton. In addition, the study of 
the DY process allows for stringent constraints on physics beyond the standard 
model (BSM).
Furthermore, dilepton events are valuable for calibrating the 
detector and monitoring the stability of the luminosity.

%Figure: The Drell-Yan process.
\begin{figure}[hbtp] % BSM models
 \centering
 \begin{tikzpicture} % ZH(inv)
  \begin{feynman}
   \vertex (q1) {\(\boldsymbol{\Pq}\)};
   \vertex [below= 3cm of q1] (q2) {\(\boldsymbol{\Paq}\)};
   \vertex [right= 1.6cm of q1] (a1);
   \vertex [below= 1.5cm of a1] (a2);
   \vertex [right= 1.6cm of a2] (b1);
   \vertex [right= 1.6cm of b1] (c1);
   \vertex [above= 1.3cm of c1] (c2) {\(\boldsymbol{\ell}\)};
   \vertex [below= 1.3cm of c1] (c3) {\(\boldsymbol{\bar{\ell}}\)};
   %\vertex [right= 1cm of c2] (d1);
   %\vertex [right= 1cm of c3] (d2);
   %\vertex [above= 0.3cm of d1] (f1) {\(\boldsymbol{\ell}\)};
   %\vertex [below= 0.3cm of d1] (f2) {\(\boldsymbol{\bar{\ell}}\)};
   %\vertex [above= 0.3cm of d2] (f3) {\(\boldsymbol{\chi}\)};
   %\vertex [below= 0.3cm of d2] (f4) {\(\boldsymbol{\bar{\chi}}\)};
   
   \diagram* {
    (q1) -- [fermion, very thick] (a2),
    (q2) -- [anti fermion, very thick] (a2),
    (a2) -- [boson, very thick, edge label'=\(\boldsymbol{\cPZ/\gamma^*}\)] (b1),
    (b1) -- [fermion, very thick] (c2),
    (b1) -- [anti fermion, very thick] (c3),
    %(c2) -- [fermion, very thick] (f1),
    %(c2) -- [anti fermion, very thick] (f2),
    %(c3) -- [fermion, very thick] (f3),
    %(c3) -- [anti fermion, very thick] (f4),
   };
  \end{feynman}
 \end{tikzpicture}
 \caption{The Drell-Yan process.} \label{fig:DYdiagram}
\end{figure}
The intermediate vector bosons $\Wpm$ and $\cPZ/\gamma^*$ (referred to as 
$\cPZ$ boson) can have non-zero momentum transverse to the beam direction 
($\pt$). This is due to the intrinsic $\pt$ of the initial-state partons 
inside the proton as well as the initial-state radiation of gluons and quarks. 
Measurements of the $\pt$ distributions of the $\Wpm$ and $\cPZ$ bosons probe 
various aspects of the strong interaction. In addition, accurate theoretical 
prediction of the $\pt$ distribution is a key ingredient for a precise 
measurement of the $\Wpm$ boson mass at the LHC.

The measurement of the W boson mass relies on a fit to the experimental data, with the
target distribution being either the $\pt$ of the visible charged lepton daughter
or the transverse mass of the system of this lepton and the missing energy.
Both of these methods depend on the modeling of the W boson $\pt$ distribution,
which suffers from uncertainties in the QCD renormalization and factorization scale and the proton parton distribution functions.
To mitigate this, it is possible to extrapolate from a measured Z boson $\pt$ distribution, using
the \textit{ratio} of theoretical predictions for W and Z $\pt$ distributions:
\begin{equation}
\frac{\mathrm{d}\sigma(W)}{\mathrm{d} \pt} = \left[ \frac{\mathrm{d}\sigma(W)/\mathrm{d}\pt }{\mathrm{d}\sigma(Z)/\mathrm{d} \pt } \right]_\mathrm{theory} \times \left[ \frac{\mathrm{d}\sigma(Z)}{\mathrm{d} \pt} \right]_\mathrm{measured}
\end{equation}
This strategy was previously employed for the W mass measurements using the Tevatron~\cite{Abazov:2012bv,Aaltonen:2012bp} and LHC Run-I data at 7 TeV~\cite{Aaboud:2017svj}.
One benefit is that some uncertainties in the ``theory'' term partially cancel.
The price is that the Z differential $\pt$ spectrum must be measured experimentally.
Therefore, it is essential to do this using the LHC Run-II data at center-of-mass energy 13 TeV,
on the path toward a future W mass measurement using those data.
      
Theoretical predictions of both the DY production total cross section and 
differential distributions are available up to next-to-next-to-leading order 
(NNLO) accuracy in perturbative quantum chromodynamics 
(QCD)~\cite{Melnikov:2006kv,Catani:2009sm}. The complete NNLO calculations of 
vector boson production in association with a jet in hadronic collisions have 
recently become available at $\mathcal{O}(\alpS^3)$ accuracy in the strong 
coupling constant~\cite{Ridder:2015dxa,Boughezal:2015ded,Boughezal:2015dva}. 
These calculations significantly reduce the factorization and renormalization scale 
uncertainties which in turn reduce theoretical uncertainties in the prediction 
of the $\pt$ distribution in the high-$\pt$ region to the order of one 
percent. Electroweak (EW) corrections are non-negligible at 
high-$\pt$~\cite{Dittmaier:2014qza,Lindert:2017olm}.      

The fixed order calculations are unreliable at low-$\pt$ region due to soft 
and collinear gluon radiation, resulting in the appearance of large logarithmic 
corrections.
Resummation of the logarithmically divergent terms up to next-to-next-to-leading logarithmic (NNLL) accuracy has been matched with the fixed order predictions.
This matching is done by subtracting resummed terms of the same order as the fixed order calculations to avoid double-counting.
It allows accurate predictions for the entire range of $\pt$~\cite{Balazs:1995nz,Catani:2015vma}.
Fixed order perturbative calculations can also be combined with parton shower models to obtain 
fully-exclusive predictions~\cite{MCatNLO,Nason:2004rx,Frixione:2002ik,Alioli:2010xd} with
no fiducial requirements.      

The $\cPZ$ boson $\pt$ and rapidity distributions have been previously 
measured, using $\Pe^+\Pe^-$ and $\mu^+\mu^-$ pairs, by the ATLAS, CMS, and 
LHCb Collaborations for proton-proton ($\Pp\Pp$) collisions at $7$, $8$, 
and~$13\TeV$ at the LHC~\cite{ATLAS_ZpT7TeV,ATLAS_ZptEta7TeV,Aad:2015auj,Aaboud:2016btc,Sirunyan:2018owv,CMS_ZpT7TeV,CMS_ZpT8TeV,CMS:2014jea,Khachatryan:2016nbe,LHCb_WZ7TeV,LHCb_Zee7TeV,LHCb_ZpT7TeV,LHCb_WZ8TeV,Aaij:2016mgv}, 
and by the CDF and D0 Collaborations at the Tevatron for proton-antiproton ($\Pp\bar\Pp$)
collisions at $\sqrt{s} = 1.96\TeV$~\cite{Affolder:1999jh,Abbott:1999yd,TevatronWZ:D0PhysRevLett2008_100,TevatronWZ:D0PhysLettB2010_693,TevatronWZ:D0PhysRevLett2011_106}. 

\section{Dark matter searches}

One of the most significant puzzles in modern physics is the nature of dark matter.
In the culmination of over a century of observations, the ``$\Lambda_{\mathrm{CDM}}$'' standard model of cosmology
has established that, in the total cosmic energy budget,
known matter only accounts for about 5\%, dark matter corresponds to 27\%, and the rest is dark energy~\cite{Hinshaw_2013}.
Although several astrophysical observations indicate that dark matter exists and interacts gravitationally with known matter,
there is no evidence yet for nongravitational interactions between dark matter and Standard Model particles.
While the nature of dark matter remains a mystery, there are a number of models that predict a particle physics origin. 

A promising possibility is that dark matter may take the form of weakly interacting massive particles.
If these particles exist, they can possibly be produced directly from, annihilate into, or scatter off Standard Model particles.
Recent dark matter searches have exploited various methods including direct~\cite{Cushman:2013zza} and indirect~\cite{Buckley:2013bha} detection.
If dark matter can be observed in direct detection experiments,
it must have substantial couplings to quarks and/or gluons, and could also be produced at the LHC~\cite{Beltran:2010ww,Goodman:2010yf,Bai:2010hh,Goodman:2010ku,Fox:2011pm,Rajaraman:2011wf}.

In this work we search for such a mechanism producing a Z boson recoiling against a pair of dark matter particles.
This final state is well-suited to probe such beyond-the-Standard Model scenarios, as
it has relatively small and precisely known Standard Model backgrounds.
Previous searches of this nature have been performed using the LHC Run-I datasets at center-of-mass energies 7 and 8 TeV.
Some make use of the same Z boson signature~\cite{Chatrchyan:2014tja,Khachatryan:2015bbl,Aad:2014vka},
while other rely on final states with hadrons, photons, or a single energetic lepton~\cite{Aad:2013oja,ATLAS:2014wra,Khachatryan:2014tva,Aad:2014vea,Aad:2014tda,Khachatryan:2014uma,Aad:2014wza,Aad:2015yga,Khachatryan:2015nua,Khachatryan:2014rra,Aad:2015zva,Khachatryan:2014rwa,Khachatryan:2016mdm}.

