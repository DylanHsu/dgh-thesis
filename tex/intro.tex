\chapter{Introduction}
\section{Differential measurements of gauge bosons}
The production of lepton pairs via the Drell--Yan (DY) process 
is essential for the physics program of the CERN Large Hadron Collider (LHC).  
The large cross section and clean experimental signature %,
%two leptons with large transverse momentum, 
provide important precision
tests of the standard model (SM),
as well as constraints on the 
%The DY process has high sensitivity to 
parton distribution functions (PDFs) of the proton. In addition, the study of 
the DY process allows for stringent constraints on physics beyond the standard 
model (BSM).
Moreover, dilepton events are valuable for calibrating the 
detector and monitoring the stability of the luminosity.

%Figure: The Drell-Yan process.
\begin{figure} % BSM models
 \centering
 \begin{tikzpicture} % ZH(inv)
  \begin{feynman}
   \vertex (q1) {\(\boldsymbol{\Pq}\)};
   \vertex [below= 3cm of q1] (q2) {\(\boldsymbol{\Paq}\)};
   \vertex [right= 1.6cm of q1] (a1);
   \vertex [below= 1.5cm of a1] (a2);
   \vertex [right= 1.6cm of a2] (b1);
   \vertex [right= 1.6cm of b1] (c1);
   \vertex [above= 1.3cm of c1] (c2) {\(\boldsymbol{\ell}\)};
   \vertex [below= 1.3cm of c1] (c3) {\(\boldsymbol{\bar{\ell}}\)};
   %\vertex [right= 1cm of c2] (d1);
   %\vertex [right= 1cm of c3] (d2);
   %\vertex [above= 0.3cm of d1] (f1) {\(\boldsymbol{\ell}\)};
   %\vertex [below= 0.3cm of d1] (f2) {\(\boldsymbol{\bar{\ell}}\)};
   %\vertex [above= 0.3cm of d2] (f3) {\(\boldsymbol{\chi}\)};
   %\vertex [below= 0.3cm of d2] (f4) {\(\boldsymbol{\bar{\chi}}\)};
   
   \diagram* {
    (q1) -- [fermion, very thick] (a2),
    (q2) -- [anti fermion, very thick] (a2),
    (a2) -- [boson, very thick, edge label'=\(\boldsymbol{\cPZ/\gamma^*}\)] (b1),
    (b1) -- [fermion, very thick] (c2),
    (b1) -- [anti fermion, very thick] (c3),
    %(c2) -- [fermion, very thick] (f1),
    %(c2) -- [anti fermion, very thick] (f2),
    %(c3) -- [fermion, very thick] (f3),
    %(c3) -- [anti fermion, very thick] (f4),
   };
  \end{feynman}
 \end{tikzpicture}
 \caption{The Drell-Yan process.} \label{fig:DYdiagram}
\end{figure}
The intermediate vector bosons $\Wpm$ and $\cPZ/\gamma^*$ (referred to as 
$\cPZ$ boson) can have non-zero momentum transverse to the beam direction 
($\pt$). This is due to the intrinsic $\pt$ of the initial-state partons 
inside the proton as well as the initial-state radiation of gluons and quarks. 
Measurements of the $\pt$ distributions of the $\Wpm$ and $\cPZ$ bosons probe 
various aspects of the strong interaction. In addition, accurate theoretical 
prediction of the $\pt$ distribution is a key ingredient for a precise 
measurement of the $\Wpm$ boson mass at the Tevatron and the LHC.  
      
Theoretical predictions of both the DY production total cross section and 
differential distributions are available up to next-to-next-to-leading order 
(NNLO) accuracy in perturbative quantum chromodynamics 
(QCD)~\cite{Melnikov:2006kv,Catani:2009sm}. The complete NNLO calculations of 
vector boson production in association with a jet in hadronic collisions have 
recently become available at $\mathcal{O}(\alpS^3)$ accuracy in the strong 
coupling constant~\cite{Ridder:2015dxa,Boughezal:2015ded,Boughezal:2015dva}. 
These calculations significantly reduce the factorization and renormalization scale 
uncertainties which in turn reduce theoretical uncertainties in the prediction 
of the $\pt$ distribution in the high-$\pt$ region to the order of one 
percent. Electroweak (EW) corrections are non-negligible at 
high-$\pt$~\cite{Dittmaier:2014qza,Lindert:2017olm}.      

The fixed order calculations are unreliable at low-$\pt$ region due to soft 
and collinear gluon radiation, resulting in the appearance of large logarithmic 
corrections. Resummation of the logarithmically divergent terms up to 
next-to-next-to-leading logarithmic (NNLL) accuracy has been matched with the 
fixed order predictions to achieve accurate predictions for the entire range 
of $\pt$~\cite{Balazs:1995nz,Catani:2015vma}. Fixed order perturbative 
calculations can also be combined with parton shower models to obtain 
fully-exclusive 
predictions~\cite{MCatNLO,Nason:2004rx,Frixione:2002ik,Alioli:2010xd} with
no fiducial requirements.      

The $\cPZ$ boson $\pt$ and rapidity distributions have been previously 
measured, using $\Pe^+\Pe^-$ and $\mu^+\mu^-$ pairs, by the ATLAS, CMS, and 
LHCb Collaborations for proton-proton ($\Pp\Pp$) collisions at $7$, $8$, 
and~$13\TeV$ at the LHC~\cite{ATLAS_ZpT7TeV,ATLAS_ZptEta7TeV,Aad:2015auj,Aaboud:2016btc,Sirunyan:2018owv,CMS_ZpT7TeV,CMS_ZpT8TeV,CMS:2014jea,Khachatryan:2016nbe,LHCb_WZ7TeV,LHCb_Zee7TeV,LHCb_ZpT7TeV,LHCb_WZ8TeV,Aaij:2016mgv}, 
and by the CDF and D0 Collaborations at the Tevatron for $\Pp\bar\Pp$ 
collisions at $\sqrt{s} = 1.96\TeV$~\cite{Affolder:1999jh,Abbott:1999yd,TevatronWZ:D0PhysRevLett2008_100,TevatronWZ:D0PhysLettB2010_693,TevatronWZ:D0PhysRevLett2011_106}. 

\section{Dark matter searches}

One of the most significant puzzles in modern physics is the nature of dark matter.
In the culmination of over a century of observations, the ``$\Lambda_{\mathrm{CDM}}$'' standard model of cosmology
has established that, in the total cosmic energy budget,
known matter only accounts for about 5\%, dark matter corresponds to 27\%, and the rest is dark energy~\cite{Hinshaw_2013}.
Although several astrophysical observations indicate that dark matter exists and interacts gravitationally with known matter,
there is no evidence yet for nongravitational interactions between dark matter and Standard Model particles.
While the nature of dark matter remains a mystery, there are a number of models that predict a particle physics origin. 

A promising possibility is that dark matter may take the form of weakly interacting massive particles.
If these particles exist, they can possibly be produced directly from, annihilate into, or scatter off Standard Model particles.
Recent dark matter searches have exploited various methods including direct~\cite{Cushman:2013zza} and indirect~\cite{Buckley:2013bha} detection.
If dark matter can be observed in direct detection experiments,
it must have substantial couplings to quarks and/or gluons, and could also be produced at the LHC~\cite{Beltran:2010ww,Goodman:2010yf,Bai:2010hh,Goodman:2010ku,Fox:2011pm,Rajaraman:2011wf}.
In this work we search for such a mechanism producing a Z boson recoiling against a pair of dark matter particles.
This final state is well-suited to probe such beyond-the-Standard Model scenarios, as
it has relatively small and precisely known Standard Model backgrounds.
