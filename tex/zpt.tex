\chapter{Measurements of differential Z boson production cross sections}
\section{Event selection}
The leptonic final state in the $\dyll$ channel consists of two opposite 
charged same-flavor high $\pt$ isolated leptons, muons or electrons, 
compatible with a $\Z$ boson decay. Therefore, the selection of the 
$\Z$~boson candidates are required to have either two muons or two electrons 
with a reconstructed mass ($\mll$) within 15 $\GeV$ the nominal $\Z$ boson mass. In addition, 
both leptons are required to have $\pt>25~\GeV$ to ensure $\sim$100\% trigger 
efficiency, and $\abs{\eta}<2.4$ for muons and electrons. 
After applying this selection the background level from non single-$\Z$ boson 
processes is below 1\%. A few distributions at the reconstruction level 
for dimuons and dielectrons after applying the full selection are 
shown in Figures~\ref{fig:recodist1} and~\ref{fig:recodist2}. 
Figures~\ref{fig:gendist1},~\ref{fig:gendist2},~\ref{fig:gendist3}, and~\ref{fig:gendist4} show the generator-level
correlations between $\pt^\Z$, $\phi^\star$, $y^\Z$, and the opening angles between the lepton pairs $\Delta\phi$, $\Delta y$, and 
$\Delta R = \sqrt{ \Delta\phi^{2} + \Delta\eta^{2} }$.

\begin{figure}
	\centering
	\includegraphics[width=0.49\textwidth]{figures/zpt/zmm_mll.pdf}
	\includegraphics[width=0.49\textwidth]{figures/zpt/zee_mll.pdf}
	\includegraphics[width=0.49\textwidth]{figures/zpt/zmm_rap.pdf}
	\includegraphics[width=0.49\textwidth]{figures/zpt/zee_rap.pdf}
	\includegraphics[width=0.49\textwidth]{figures/zpt/zmm_ptll.pdf}
	\includegraphics[width=0.49\textwidth]{figures/zpt/zee_ptll.pdf}
	\caption{Distributions at the reconstruction level of $\mll$ (top), $|\rapidity^\Z|$ (center), and 
	$\pt^\Z$ (bottom) for dimuons (left) and dielectrons (right) after applying the full selection.}
	\label{fig:recodist1}
\end{figure}

\begin{figure}
	\centering
	\includegraphics[width=0.49\textwidth]{figures/zpt/zmm_cos_theta_star.pdf}
	\includegraphics[width=0.49\textwidth]{figures/zpt/zee_cos_theta_star.pdf}
	\includegraphics[width=0.49\textwidth]{figures/zpt/zmm_phi_star.pdf}
	\includegraphics[width=0.49\textwidth]{figures/zpt/zee_phi_star.pdf}
	\caption{Distributions at the reconstruction level of $cos(\theta^\star$ (top) and $\phi^\star$ (bottom) for dimuons 
	(left) and dielectrons (right) after applying the full selection.}
	\label{fig:recodist2}
\end{figure}

\begin{figure}
	\centering
	\includegraphics[width=0.49\textwidth]{figures/zpt/phiVpt.pdf}
	\includegraphics[width=0.49\textwidth]{figures/zpt/phiVrap.pdf}
	\includegraphics[width=0.49\textwidth]{figures/zpt/ptVrap.pdf}
	\caption{Distributions at the generator level of $\phi^\star$ vs $\pt^\Z$ (top left), $\phi^\star$ vs $y^\Z$ (top right), and $\pt^\Z$ vs $y^\Z$ (bottom) for dilepton pairs.}
	\label{fig:gendist1}
\end{figure}

\begin{figure}
	\centering
	\includegraphics[width=0.49\textwidth]{figures/zpt/ptVdphi.pdf}
	\includegraphics[width=0.49\textwidth]{figures/zpt/phiVdphi.pdf}
	\includegraphics[width=0.49\textwidth]{figures/zpt/rapVdphi.pdf}
	\caption{Distributions at the generator level of $\Delta\phi$ with $\pt^\Z$ (top left), $\phi^\star$ (top right), and $y^\Z$ (bottom) for dilepton pairs.}
	\label{fig:gendist2}
\end{figure}

\begin{figure}
	\centering
	\includegraphics[width=0.49\textwidth]{figures/zpt/ptVdy.pdf}
	\includegraphics[width=0.49\textwidth]{figures/zpt/phiVdy.pdf}
	\includegraphics[width=0.49\textwidth]{figures/zpt/rapVdy.pdf}
	\caption{Distributions at the generator level of $\Delta y$ with $\pt^\Z$ (top left), $\phi^\star$ (top right), and $y^\Z$ (bottom) for dilepton pairs.}
	\label{fig:gendist3}
\end{figure}

\begin{figure}
	\centering
	\includegraphics[width=0.49\textwidth]{figures/zpt/ptVdr.pdf}
	\includegraphics[width=0.49\textwidth]{figures/zpt/phiVdr.pdf}
	\includegraphics[width=0.49\textwidth]{figures/zpt/rapVdr.pdf}
	\caption{Distributions at the generator level of $\Delta R$ with $\pt^\Z$ (top left), $\phi^\star$ (top right), and $y^\Z$ (bottom) for dilepton pairs.}
	\label{fig:gendist4}
\end{figure}

\section{Background estimation}
After the full selection, the amount of background processes in the data sample is 
rather small thanks to the clean signature and the relatively tight selection. 
As mentioned in Section ~\ref{sec:simbkg}, the background processes can be split in 
two components: the resonant background and the nonresonant background. 
The resonant background comes from events with a real $\Z$ boson in the final state, 
e.g. $\W\Z$ diboson production; while the nonresonant background comes from events 
which do not have a $\Z$ boson in the final state. 
The first set of backgrounds in estimated using a simulated events, as described in 
Section ~\ref{sec:simbkg}. 

Nonresonant backgrounds (NRB) consist mainly 
 of leptonic $\PW$ decays in $\ttbar$, $\cPqt\PW$ decays and $\PW\PW$ events. 
 Small contributions from single top-quark events produced from
$s$-channel and $t$-channel processes, and $\cPZ\rightarrow \Pgt\Pgt$
events are also considered in this NRB estimation. 
This contribution of the non-resonant backgrounds is estimated from a control
sample of dilepton events of different flavor ($\Pe^{\pm}\Pgm^{\mp}$) 
that pass all other selection criteria.
The method has been previously used in many analyses and assumes the lepton flavor symmetry in the final states of these processes.
Since the leptonic decay branching ratios for the $ee$, $\mu\mu$ and $e\mu$ final states from NRB are 1:1:2,
the $e\mu$ events selected inside the $\Z$-mass window can be extrapolated to the $ee$ and $\mu\mu$ channels.
To account for differences in efficiency for electrons and muons, 
a correction factor $k_{\mu\mu}$ can be derived by comparing the NRB yields in the $ee$ and $\mu\mu$ channels,
exactly as is done in Formulas~\ref{eq:kee} and~\ref{eq:nrb_xfer} from Section~\ref{ss:dm_nrb}:

\begin{equation}
k_{\mu\mu} = \frac{\epsilon_\mu}{\epsilon_{\Pe}} \approx \sqrt{\frac{N^{\mu\mu}_{NRB}}{N^{\Pe\Pe}_{NRB}}}
\end{equation}
once again under the assumption is that each lepton leg acts independently.
The factor $k_{\mu\mu}$ is found to be about $1.3$ at the final selection, 
consistent between data and simulation. 
With this correction factor, the relation between the NRB yields in the signal and control region is:
\begin{equation}
\begin{split}
  N^{\mu\mu}_{NRB}     &= \frac{1}{2} k_{\mu\mu} N^{e\mu}_{NRB} \\
  N^{\Pe\Pe}_{NRB} &= \frac{1}{2} \frac{1}{k_{\mu\mu}} N^{e\mu}_{NRB} \\
\rightarrow N^{2\ell}_{NRB}  &= \frac{1}{2} \left( k_{\mu\mu} + \frac{1}{k_{\mu\mu}} \right) N^{e\mu}_{NRB}
\end{split}
\end{equation}

A summary of the data, signal, and background yields after the full selection for the dimuon and dielectron 
final states is shown in Table~\ref{tab:zpt_yields}.

\begin{table}[hbtp]
  \begin{center}
\caption{Summary of data, signal, and background yields after the full selection. 
The signal yields are quoted using \textsc{MadGraph5\_aMC@NLO}.\label{tab:zpt_yields}}
\begin{tabular}{ccccc}
\hline
Final state      & Data & $\Z\to\ell\ell$ & Resonant bkg. & Nonresonant bkg. \\
\hline
$\mu\mu$         & $\sim 20.4 \times 10^{6}$ & $\sim 20.7 \times 10^{6}$ & $\sim 30 \times 10^{3}$ & $\sim 41 \times 10^{3}$ \\
$\Pe\Pe$         & $\sim 12.1 \times 10^{6}$ & $\sim 12.0 \times 10^{6}$ & $\sim 19 \times 10^{3}$ & $\sim 26 \times 10^{3}$ \\
\hline
  \end{tabular}
  \end{center}
\end{table}

\section{Systematic uncertainties}

In this section the systematic uncertainties taken into
account in the Z differential measurement are summarized. 
Uncertainties do not only influence the overall normalization
(\eg the uncertainty in the luminosity measurement), but also the
distribution of relevant kinematic observables (\eg the uncertainty in
the lepton momentum scale), are treated as shape uncertainties. Their
impact is evaluated by performing the full analysis not only for the
central value of the relevant parameter, but also with its value
shifted up and down by one standard deviation. For
each source of uncertainty, the impact in different bins of the final 
distribution is thus considered fully correlated, while
independent sources of uncertainty are treated as uncorrelated.

%%%%%%%%%%%%%%%%%%%%%%%%%%%%%%%%%%
\subsection{Sources of systematic uncertainties}
%%%%%%%%%%%%%%%%%%%%%%%%%%%%%%%%%%
The different sources of the systematic uncertainties are summarized in this section

%%%%%%%%%%%%%%%%%%%%%%%%%%%%%%%%%%
\subsubsection{Luminosity}
%%%%%%%%%%%%%%%%%%%%%%%%%%%%%%%%%%

The assigned uncertainty to the integrated luminosity measurement for
the data set used in this analysis is $\lumiunc$.

%%%%%%%%%%%%%%%%%%%%%%%%%%%%%%%%%%
\subsubsection{Trigger, lepton reconstruction and identification efficiencies}
%%%%%%%%%%%%%%%%%%%%%%%%%%%%%%%%%%

Discrepancies in the lepton reconstruction and identification
efficiencies between data and simulation are corrected by applying
to all MC samples 
data-to-simulation scale factors measured using $\dyll$ events in the $\cPZ$
peak region~\cite{wzxs} that are recorded with unbiased triggers. These factors
depend on the lepton $\pt$ and $\eta$ and are within 2\% for electrons and muons.
The uncertainty in the determination of the trigger efficiency leads to an uncertainty 
smaller than 1\% in the expected signal yield. Residual difference between the analysis 
lepton requirements with respect to the trigger selections is well covered by 
the uncertainty in the trigger efficiency. The uncertainty due to 
the muon (electron) reconstruction efficiency varies between 0.1\% (0.2\%) in the central 
part of the detector up to (0.4\%) (1.0\%) at large $|\eta|$ values. 
The uncertainty due to the lepton identification selection is about 0.4\% per muon leg, and 
about 1.1\% per electron leg, although with a sizable dependence on $\eta$ and $\pt$.
The precise methodology of determining these identification scale factors has been expounded above,
in Chapter~\ref{chap:efficiency}.
The trigger and reconstruction efficiencies are measured in similar ways,
but the fitting procedure is comparatively trivial due to high purity.

%%%%%%%%%%%%%%%%%%%%%%%%%%%%%%%%%%
\subsubsection{Lepton momentum scale and resolution}
%%%%%%%%%%%%%%%%%%%%%%%%%%%%%%%%%%

The lepton momentum scale uncertainty is computed by varying the
momentum of the leptons by their uncertainties. The effect 
on the analysis is rather small, except for very low or very high dilepton 
$\pt$.

%%%%%%%%%%%%%%%%%%%%%%%%%%%%%%%%%%
\subsubsection{L1 pre-firing trigger inneficiency}
%%%%%%%%%%%%%%%%%%%%%%%%%%%%%%%%%%
The L1 pre-firing trigger inefficiency is an experimental issue in 2016 which
reduces the number of selected events. In addition, it introduces 
a sizable uncertainty since it is not easy to precisely measure this effect.

There are a few aspects to take into accout. The rate of un-prefirable events 
is rather small compared to the total data sample, about $0.1\%$, and therefore 
the statistical precision of that effect is a concern. When an ECAL TP pre-fires, 
its entire energy is assigned to bunch crossing 1 (BX-1).
If the energy of the TP is large enough to pre-fire of a L1 trigger, the event is lost. 
In contrast, if the BX-1 is not accepted, 
a residual effect on BX0 is present because a null energy is associated to the 
early TP. Finally, the TP inputs are used by the ECAL selective readout units 
(SRPs) to decide whether a certain region of the detector needs to be fully 
readout or zero-suppressed. Crystals associated to the early TP will be readout 
in zero-suppression mode, injecting a bias in the HLT/offline energy measurement.

Overall, the recommended uncertainty is about 20\% of the correction. This translates 
to a rather small impact in the inclusive analysis, but it leads to a sizable impact 
in the $\Z$ boson rapidity measurement, and this is all which can be done.

Since this effect directly affects the lepton and event reconstruction, the
corresponding uncertainty has been merged together with the standard lepton
reconstruction efficiency uncertainties.

%%%%%%%%%%%%%%%%%%%%%%%%%%%%%%%%%%
\subsubsection{Background subtraction}
%%%%%%%%%%%%%%%%%%%%%%%%%%%%%%%%%%

Since the resonant background processes are estimated from simulation, the uncertainties on 
the normalization are derived from variations of the QCD scale, $\alpha_{s}$ and 
parton distribution functions (PDFs) variations~\cite{Botje:2011sn,Alekhin:2011sk,Lai:2010vv,Martin:2009iq,Ball:2011mu,MCFM}. 
The PDF and $\alpha_s$ uncertainties for signal and background processes are estimated 
from the standard deviation of weights from the replicas provided in the 
NNPDF3.0 parton distribution set~\cite{nnpdf}.

The procedure for estimating uncertainties arising from Parton
Distribution Functions (PDF uncertainties) follows the recommendations
issued by the PDF4LHC group \cite{Butterworth:2015oua}.
For the second run of the LHC, the PDF4LHC group has provides a set of combined
PDF sets -- the PDF4LHC pdf sets -- which are used in the following for the estimation
of PDF uncertainty.

The uncertainty in the nonresonant background is estimated to the be 
about 5\%, which is rather conservative, but it makes barely any difference in
this analysis.

%%%%%%%%%%%%%%%%%%%%%%%%%%%%%%%%%%
\subsubsection{Model dependence in unfolding}
%%%%%%%%%%%%%%%%%%%%%%%%%%%%%%%%%%

Model dependence of unfolding is due to the folding of the detector resolution 
with the expected spectra. Unfolding is expected to work properly if these 
differences are small and the predicted migration relative contribution between 
the bins close to the correct value. In order to keep this effect under control 
a different matrix can be constructed starting from a different \MC{} generator; 
this generator should be full reconstructed in order to correctly account for 
the bin-migrations. Alternatively, one could try to reweight the \MC{} to 
match other predicted spectra. 
Model dependence is evaluated unfolding the data with the different models and 
symmetrizing the uncertainties.

%%%%%%%%%%%%%%%%%%%%%%%%%%%%%%%%%%
\subsubsection{Statistical uncertainty estimation}
%%%%%%%%%%%%%%%%%%%%%%%%%%%%%%%%%%

The computation of the statistical uncertainty estimation after unfolding is 
complicated by the bias introduced by the unfolding itself. A ``proper'' way 
of doing this step is not fully established from the statistical point of view, 
making this an active field of research \cite{Lyons:unfolding,kuusela}.
Different toys and resampling techniques are available in order to perform such 
operation, that usually provide a better coverage with respect to the analytical 
error propagation. The toy generation is performed by \RooUnfold{} smearing the 
data distribution using Gaussian uncertainties corresponding to the histogram 
errors.

%%%%%%%%%%%%%%%%%%%%%%%%%%%%%%%%%%
\subsection{Correlations} 
%%%%%%%%%%%%%%%%%%%%%%%%%%%%%%%%%%
The correlations among the different bins and the two final states used in the 
analysis are described in this section.

All the systematic uncertainties are considered to be correlated among the 
different bins for a single variable, with the exception of the uncertainties with statistical nature. 
The statistical uncertainties of the MC{} samples are treated 
as uncorrelated quantities among the different bins. In addition, the statistical 
uncertainties in the lepton efficiency scale factors for each lepton $\eta$ and $\pt$ 
bin used for such measurements are treated as uncorrelated quantities. 

When combining the muon and electron channels, the luminosity, background estimation, 
and modelling uncertainties are treated as correlated parameters, all others are 
considered as uncorrelated between them.

%%%%%%%%%%%%%%%%%%%%%%%%%%%%%%%%%%
\subsection{Total systematic uncertainty} 
%%%%%%%%%%%%%%%%%%%%%%%%%%%%%%%%%%

A graphical representation of the different contributions to the systematic uncertainty 
is shown in Figures~\ref{fig:zpt_syst0},~\ref{fig:zpt_syst1}, and~\ref{fig:zpt_syst2}. 
A summary of the systematic uncertainties related to the lepton efficiency measurements 
of muons and electrons is shown in Figure~\ref{fig:zpt_systeff}.

\begin{figure}
	\centering
	\includegraphics[width=0.49\textwidth]{figures/zpt/histoUnfoldingSystPt_nsel0_dy3.pdf}
        \includegraphics[width=0.49\textwidth]{figures/zpt/histoUnfoldingSystPt_nsel1_dy3.pdf}
	\includegraphics[width=0.49\textwidth]{figures/zpt/histoUnfoldingSystRap_nsel0_dy3.pdf}
        \includegraphics[width=0.49\textwidth]{figures/zpt/histoUnfoldingSystRap_nsel1_dy3.pdf}
	\includegraphics[width=0.49\textwidth]{figures/zpt/histoUnfoldingSystPhiStar_nsel0_dy3.pdf}
        \includegraphics[width=0.49\textwidth]{figures/zpt/histoUnfoldingSystPhiStar_nsel1_dy3.pdf}
	\caption{Summary of the systematic uncertainties of muons (left) and electrons (right) 
	for the $\pt^{\Z}$ analysis (top), the $|\rapidity^\Z|$ analysis (center), and the $\phi^\star$ analysis (bottom).}
	\label{fig:zpt_syst0}
\end{figure}

\begin{figure}
	\centering
	\includegraphics[width=0.49\textwidth]{figures/zpt/histoUnfoldingSystPtRap0_nsel0_dy3.pdf}
	\includegraphics[width=0.49\textwidth]{figures/zpt/histoUnfoldingSystPtRap1_nsel0_dy3.pdf}
	\includegraphics[width=0.49\textwidth]{figures/zpt/histoUnfoldingSystPtRap2_nsel0_dy3.pdf}
	\includegraphics[width=0.49\textwidth]{figures/zpt/histoUnfoldingSystPtRap3_nsel0_dy3.pdf}
	\includegraphics[width=0.49\textwidth]{figures/zpt/histoUnfoldingSystPtRap4_nsel0_dy3.pdf}
	\caption{Summary of the systematic uncertainties for the $\pt^{\Z}$ analysis of muons in the 
	$0.0 < |\rapidity^\Z| < 0.4$ region (top left), $0.4 < |\rapidity^\Z| < 0.8$ region (top right),
	$0.8 < |\rapidity^\Z| < 1.2$ region (center left), $1.2 < |\rapidity^\Z| < 1.6$ region (center right), and 
	$1.6 < |\rapidity^\Z| < 2,4$ region (bottom).
	}
	\label{fig:zpt_syst1}
\end{figure}

\begin{figure}
	\centering
	\includegraphics[width=0.49\textwidth]{figures/zpt/histoUnfoldingSystPtRap0_nsel1_dy3.pdf}
	\includegraphics[width=0.49\textwidth]{figures/zpt/histoUnfoldingSystPtRap1_nsel1_dy3.pdf}
	\includegraphics[width=0.49\textwidth]{figures/zpt/histoUnfoldingSystPtRap2_nsel1_dy3.pdf}
	\includegraphics[width=0.49\textwidth]{figures/zpt/histoUnfoldingSystPtRap3_nsel1_dy3.pdf}
	\includegraphics[width=0.49\textwidth]{figures/zpt/histoUnfoldingSystPtRap4_nsel1_dy3.pdf}
	\caption{Summary of the systematic uncertainties for the $\pt^{\Z}$ analysis of electrons in the 
	$0.0 < |\rapidity^\Z| < 0.4$ region (top left), $0.4 < |\rapidity^\Z| < 0.8$ region (top right),
	$0.8 < |\rapidity^\Z| < 1.2$ region (center left), $1.2 < |\rapidity^\Z| < 1.6$ region (center right), and 
	$1.6 < |\rapidity^\Z| < 2,4$ region (bottom).
	}
	\label{fig:zpt_syst2}
\end{figure}

\begin{figure}
	\centering
	\includegraphics[width=0.49\textwidth]{figures/zpt/histoUnfoldingSystEffPt_nsel0_dy3.pdf}
        \includegraphics[width=0.49\textwidth]{figures/zpt/histoUnfoldingSystEffPt_nsel1_dy3.pdf}
	\includegraphics[width=0.49\textwidth]{figures/zpt/histoUnfoldingSystEffRap_nsel0_dy3.pdf}
        \includegraphics[width=0.49\textwidth]{figures/zpt/histoUnfoldingSystEffRap_nsel1_dy3.pdf}
	\includegraphics[width=0.49\textwidth]{figures/zpt/histoUnfoldingSystEffPhiStar_nsel0_dy3.pdf}
        \includegraphics[width=0.49\textwidth]{figures/zpt/histoUnfoldingSystEffPhiStar_nsel1_dy3.pdf}
	\caption{Summary of the systematic uncertainties related to the lepton efficiency measurements of muons (left) and electrons (right) 
	for the $\pt^{\Z}$ analysis (top), the $|\rapidity^\Z|$ analysis (center), and the $\phi^\star$ analysis (bottom).}
	\label{fig:zpt_systeff}
\end{figure}

The differential cross section measurements can also performed with respect to 
the inclusive cross section. Therefore, in that case the uncertainties are 
largely reduced, in particular the lepton reconstruction selection 
efficiency mostly cancel out, and the effect due to the integrated luminosity 
completely cancel out. Those uncertainties are summarized in Figure~\ref{fig:zpt_syst_xratio}.

\begin{figure}
	\centering
	\includegraphics[width=0.49\textwidth]{figures/zpt/histoUnfolding_XSRatioSystPt_nsel0_dy3.pdf}
        \includegraphics[width=0.49\textwidth]{figures/zpt/histoUnfolding_XSRatioSystPt_nsel1_dy3.pdf}
	\includegraphics[width=0.49\textwidth]{figures/zpt/histoUnfolding_XSRatioSystRap_nsel0_dy3.pdf}
        \includegraphics[width=0.49\textwidth]{figures/zpt/histoUnfolding_XSRatioSystRap_nsel1_dy3.pdf}
	\includegraphics[width=0.49\textwidth]{figures/zpt/histoUnfolding_XSRatioSystPhiStar_nsel0_dy3.pdf}
        \includegraphics[width=0.49\textwidth]{figures/zpt/histoUnfolding_XSRatioSystPhiStar_nsel1_dy3.pdf}
	\caption{Summary of the systematic uncertainties of muons (left) and electrons (right) 
	for the $\pt^{\Z}$ analysis (top), the $|\rapidity^\Z|$ analysis (center), and the $\phi^\star$ analysis (bottom) 
	for the differential cross section measurements with respect to the inclusive cross section.}
	\label{fig:zpt_syst_xratio}
\end{figure}
