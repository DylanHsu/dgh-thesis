\chapter{Physics objects}
\label{chap:objects}
\section{Electrons}
\subsection{Electron reconstruction}
Electrons in the CMS Detector are reconstructed through association of a track from the silicon detector with a cluster of energy in the ECAL.
Electron tracks are formed from initial seeds likely to correspond to initial electron trajectories, which are then used to build tracks by collecting hits in the silicon tracker using the combinatorial Kalman filter procedure.
Next, a track fitting procedure is undertaken using a Gaussian Sum Filter (GSF), in which the energy loss in each tracker layer is approximated by a mixture of Gaussian distributions.
Meanwhile, the electron energy has been collected in several crystals of the ECAL. 
These depositions undergo two steps of clustering.
The first step finds clusters from crystal arrays of $5 \times 1$ in $\eta \times \phi$ for the ECAL barrel, and $5 \times 5$ crystals for the ECAL endcaps.
The second step forms a supercluster (SC) comprising the energy of constituent clusters.
The supercluster position and energy, along with the GSF track, reconstruct the electron in the detector.\cite{ElectronReco2015}
The reconstruction procedure is also informed by the Particle-Flow algorithm, described later in Section \ref{sec:pf}.

The final reconstructed electron momentum is found using a multivariate regression tuned on simulation to give better energy resolution.
A residual energy scale correction is applied to the reconstructed electrons based on time, electron $\eta$, and shower shape variables
such that the Z mass peak resolution is enhanced.  In addition, a smearing is applied to simulation such that the energy resolution in data and simulation match.


\subsection{Electron identification variables}

The shower shape of the hits in the ECAL provides discriminating variables for identifying electrons.
One such variable is $\sigma_{i\eta i\eta}$.
To understand $\sigma_{i\eta i\eta}$, we first discuss its predecessor, $\sigma_{\eta\eta}$,
which is a weighted second moment of the energies deposited in the $5\times5$ crystals
centered on the ECAL supercluster seed crystal. Here, $\eta_i$ is the pseudorapidity
coordinate of the $i^{\mathrm{th}}$ ECAL crystal. The formula for $\sigma_{\eta\eta}$ is:
\begin{equation}
  \sigma_{\eta\eta} = \sqrt{\frac{\sum_{i}^{\mathrm{5\times5}}w_{i}(\eta_{i}-\bar{\eta}_{\mathrm{5\times5}})^{2}}{\sum_{i}^{\mathrm{5\times5}}w_{i}}}
  \quad\mathrm{with}\quad w_{i} = 4.2 + \mathrm{ln} \frac{E_i}{E_{\mathrm{5\times5}}}
\end{equation}

This variable does not perform well near cracks in the ECAL, where there is a gap in $eta$.
A $\mathrm{5\times5}$ supercluster containing this gap will arbitrarily have a higher value of $\sigma_{\eta\eta}$.
For this reason, a simple correction is made where we consider instead the crystal index instead of absolute $\eta$, and multiply it by the average crystal $\eta$ size of 0.0175.
Counting the $i^{\mathrm{th}}$ crystal ordering in the $\eta$ direction, $N(i)$, we derive the variable $\sigma_{i\eta i\eta}$ (using the same weights $w_{i}$:

\begin{equation}
  \sigma_{i\eta i\eta} = \sqrt{\frac{\sum_{i}^{\mathrm{5\times5}}w_{i}(0.0175 N(i) + \eta_{\mathrm{Seed}} - \bar{\eta}_{\mathrm{5\times5}})^{2}}{\sum_{i}^{\mathrm{5\times5}}w_{i}}}
\end{equation}

Other ECAL-related variables are:
$H/E_{\rm SC}$, the ratio of HCAL to ECAL energy deposition associated
with a supercluster seed; and $|1/E_{\rm SC} - 1/p|$,
which is technically computed as $|(1 - E_{SC}/p_{\mathrm{Track}})/E_{\mathrm{ECAL}}|$.
Compared to charged hadron fakes, the true electron will deposit all or most of its energy in the ECAL.

Another class of useful variables are related to matching attributes of the ECAL supercluster
and the tracker track. $\Delta\eta(SC,trk)$ is the difference in $\eta$ between the 
supercluster and the track at the track's point-of-closest-approach to the supercluster, 
extrapolated from the innermost track state. $\Delta\phi(SC,trk)$ is the analagous quantity
in $\phi$-coordinate.

The reconstructed electron candidate's impact parameter is important for suppressing pileup.
Thus, we apply cuts on the transverse impact parameter  $d_0(\mathrm{vtx})$ and
the longitudinal impact parameter $d_z(\mathrm{vtx})$ with respect to the best primary vertex.

The term ``missing hits'' refers to how many hits we would expect to see in the
inner tracker due to this particle which were missed. A conversion veto is applied which
also accounts for missing hits, along with the minimum distance between conversion tracks
and the difference in $\mathrm{cot} \theta$ between the partner tracks.

Lastly, we apply a combined isolation requirement, 
$I_{rel}^{e}$, relative to the electron transverse momentum and corrected for
effective area. This bounds the amount of hadronic activity and other EM activity
arriving near the electron.
The computation of the isolation relies on information from the Particle Flow algorithm, described further below in Section~\ref{sec:pf}.
The isolation of the lepton candidates is computed from the flux of particle candidates
found within a cone of $\Delta R = 0.4$ built around the lepton direction~\cite{CMS-PAS-PFT-10-002}.
The flux of particles is computed independently for the charged hadrons, neutral hadrons and photon candidates.
The contribution from neutral hadron candidates is corrected for the influence of pileup by using the \textit{effective area} approach.  
The average energy density per area due to pileup ($\rho$) is multiplied with an effective area ($A_{\rm eff}$) and subtracted from the isolation sum. 
$A_{\rm eff}$ is chosen in such a way that the isolation is independent of the number of pileup interactions.
The relative electron isolation sum is defined as: 

\begin{equation}
I^{e}_{\rm rel}=\frac{1}{p_{\rm T}}  \left[ I_{\rm ch}+\max(I_{\rm nh}+I_{\rm g}-A_{\rm eff}\cdot\rho,0) \right]
\label{eq:eleisol}
\end{equation}

\subsection{Electron identification working points}

After the electron reconstruction procedure, there are still fake electrons from 
other objects such as charged pions which can be mistaken for electrons.
The purpose of electron identification is to improve the purity of electrons used in a physics analysis while maintaining a high efficiency.
Different use cases require differing amounts of stringency.
An extreme example would be a resonant final state with four electrons.
Due to the combinatorics, and the rarity of three-electron final state processes, it will be very difficult to have a false positive.
In such a case it is best to maximize the identification efficiency.
On the other extreme end, consider the final state with only a single electron,
as seen in $mathrm{W^\pm}$ boson production. There, due to the huge fake rate, it is best to maximize the purity at the cost of efficiency by using tighter identification criteria.
The different working points for electron identification are standardized in the CMS experiment.
Those relevant to this work are described below.

For the purpose of vetoing events with more than two leptons, the electrons are required to fulfill the so-called {\em ``Veto''} working point. It consists of a series of requirements
on the supercluster shape, impact parameter,
and geometric distance between the electron track and the supercluster.
The ratio of energy deposits in HCAL and ECAL is also used in the definition of the {\em ``Veto''} ID.
In this case, no veto for reconstructed photon conversions in the vicinity of the electron is applied.

Only electrons passing the {\em ``Medium''} ID are considered for the $Z$ candidate. The {\em ``Medium''} ID
imposes tighter cuts than the {\em ``Veto''} ID and suppresses potential electrons from photon conversions.
Electrons used to build the $Z$ candidate must have a \pt$>$20~GeV and be reconstructed in the tracker acceptance, i.e. $\abs{\eta}< 2.5$.
In addition to the ID requirements, there must be no tracker or global muon with at least $10$ tracker
hits reconstructed inside a cone of $\Delta R = 0.1$ built around the momentum of the electron candidate.

Lastly, the {\em ``Tight''} ID is used to identify Tags for the tag-and-probe method used
to determine the electron identification efficiencies, seen in Chapter~\ref{chap:efficiency}.

Table~\ref{tab:electronid} summarizes the identification criteria used for electrons.

\begin{table}[hbtp]
\begin{center} {\scriptsize
\begin{tabular}{l|lll|lll} \hline\hline
\multirow{2}{*}{Requirement} & \multicolumn{3}{c}{Barrel}                                                   & \multicolumn{3}{c}{Endcap} \\
                             & \multicolumn{3}{c}{$\abs{\eta}<1.479$}                                       & \multicolumn{3}{c}{$1.479<\abs{\eta}<2.5$} \\\hline
                             & {\em ``Veto''}                         & {\em ``Medium''} & {\em ``Tight''}  & {\em ``Veto''}   & {\em ``Medium''} & {\em ``Tight''}   \\\hline
$\sigma_{i\eta i\eta}$       & $<0.0115$                              & $< 0.00998 $     & $<0.00998 $      & $<0.037$         & $<0.0298$        & $<0.0292$         \\
$\Delta\eta(SC,trk)$         & $<0.00749$                             & $<0.00311$       & $<0.00308$       & $<0.00895$       & $<0.00609$       & $<0.00605$        \\
$\Delta\phi(SC,trk)$         & $<0.228 $                              & $<0.103 $        & $<0.0816$        & $<0.213$         & $<0.045$         & $<0.0394$         \\
$H/E_{\rm SC}$               & $<0.356$                               & $<0.253 $        & $<0.0414$        & $<0.211$         & $<0.0878$        & $<0.0641$         \\
$I_{rel}^{e}$                & $<0.175$                               & $<0.0695 $       & $<0.0588 $       & $<0.159$         & $<0.0821 $       & $<0.0571$         \\
$|1/E_{\rm SC} - 1/p|$       & $<0.299$                               & $<0.134$         & $<0.129$         & $<0.15$          & $<0.14$          & $<0.13$           \\
$d_0(vtx)$ (cm)              & $<0.05$                                & $<0.05$          & $<0.05$          & $<0.1$           & $<0.1$           & $<0.1$            \\
$d_z(vtx)$ (cm)              & $<0.1$                                 & $<0.1$           & $<0.1$           & $<0.2$           & $<0.2$           & $<0.2$            \\
Missing hits                 & $\leq$3                                & $\leq$1          & $\leq$1          & $\leq$3          & $\leq$1          & $\leq$1           \\
pass conversion veto         & \checkmark                             & \checkmark       & \checkmark       & \checkmark       & \checkmark       & \checkmark        \\
\hline\hline
\end{tabular}
}
\end{center}
\caption{Electron ID requirements.}
\label{tab:electronid}
\end{table}

\section{Muons}
\subsection{Muon reconstruction}
Muon tracks are reconstructed in the CMS Detector in both the silicon tracker and the muon spectrometer,
resulting respectively in tracker tracks and standalone-muon tracks.
Subsequently, these tracks inform two reconstruction approaches:
\begin{enumerate}
  \item {\em Global Muon reconstruction (outside-in):} starting from a standalone
    muon in the muon system, a matching tracker track is found and a
    {\em global-muon track} is fitted combining hits from the tracker track
    and standalone-muon track. At large transverse momenta ($\pt \gsim 200$~GeV/$c$),
    the global-muon fit can improve the momentum resolution compared
    to the tracker-only fit. 
  \item {\em Tracker Muon reconstruction (inside-out):} in this approach, all
    tracker tracks with $\pt >$ 0.5~GeV/$c$ and $p >$ 2.5~GeV/$c$ are
    considered as possible muon candidates and are extrapolated to the muon system, taking into
    account the expected energy loss and the uncertainty due to multiple
    scattering. If at least one muon segment (i.e. a short track stub made of DT or CSC hits)
    matches the extrapolated track,
    the corresponding tracker track qualifies as a {\em tracker-muon track}.
    The extrapolated track and the segment are considered to be matched if the distance
    between them in local $x$ is less than 3~cm or if the value of the pull
    for local $x$ is less than 4.
    At low momentum (roughly $p < 5$~GeV/$c$) this approach is more
    efficient than the global muon 
    reconstruction, since it requires only a single muon segment in
    the muon system, whereas global muon reconstruction is designed to have
    high efficiency for muons penetrating through more than one muon station.
\end{enumerate}

The majority of muons from collisions (with sufficient momentum) are reconstructed either 
as a Global Muon or a Tracker Muon, or very often as both.
A third, rare case is where both approaches fail and only a {\em standalone-muon track} is found~\cite{CMS-PAS-MUO-10-004}.
Standalone muons suffer from low purity and poor momentum resolution, so they are not used in this work.

\subsection{Muon identification variables}

There are several sources of real or fake reconstructed muons in CMS:
\begin{enumerate}
\item Primary Interaction
\item Decay-in-Flight (Kaon, Pion)
\item Cosmic Muon
\item Hadronic Punch-Through
\end{enumerate}
For the purposes of this work, we are interested in identifying real muons
from a hard primary interaction, and rejecting the nonprompt or fake reconstructed muons.
The chance that a kaon or pion is reconstructed as a muon increases along with its transverse momentum. It goes from from less than one percent at low momentum, to ten percent at high momentum.
Cosmic muons come from the atmosphere, not the interaction point,
and are in general easy to deal with using vertexing. Hadronic punch-through results from imperfect absorption
of hadronic cascades in the hadron calorimeter, which reach the muon stations.

Track-related quality cuts are effective against these backgrounds. We place requirements on
the fraction of valid tracker hits, the number of valid pixel hits, and the number of
tracker layers with measurement. For Global-reconstructed muons, we also compute the
normalized global track $\chi^2$ and compare the position match of the tracker and the standalone muon
reconstruction algorithms.

Decays in flight introduce a kink in the muon candidate track.
We use a kink-finding algorithm to reject such candidates, operating as follows.
At each tracker layer, the extrapolated states from the two halves of the tracker tracks are compared
to get a $\chi^2$. The maximum $\chi^2$ across all layers is used as the per-track discriminator.
This is effective even at low $p_{T}$, allowing rejection of 10-20\% of such background while
maintaining greater than 98\% efficiency for real muons.
% DGH: Yes this is accurate and the numbers are from slides by Giovanni Petrucciani.

The segment compatibility is another useful discrimination quantity for the muons.
It is computed by evaluating which crossed stations have matching muon statements,
and the expected uncertainty of the extrapolated position due to multiple scattering.

Analagous to the electrons, for muons we also compute the transverse and longitudinal impact parameters,
and a corrected relative isolation quantity, $I_{rel}^{\mu}$.
For the muon isolation, in the same manner as for electrons, a cone of $\Delta R = 0.4$ is built to compute the flux of particle flow candidates,
the ``delta-beta'' correction is applied to correct for pileup contamination. This correction is achieved by
subtracting half the sum of the \pt of the charged particles in the cone of interest but with particles not originating from the primary vertex.
The muon isolation is therefore defined as:

\begin{equation}
I^{\mu}_{\rm rel}=\frac{1}{p_{\rm T}}  \left[ I_{\rm ch}+\max(I_{\rm nh}+I_{\rm g}-0.5\cdot I_{\rm chPU} ,0) \right]
\label{eq:muisol}
\end{equation}

The factor $0.5$ corresponds to a approximate average of neutral to charged particles and has been measured in jets in~\cite{CMS-PAS-PFT-10-002}.

\subsection{Muon identification working points}
Similar to the electron identification working points described above,
a variety of muon identification working points are designed for different use cases.
The Medium muon algorithm allows for both global and tracker muons, with different requirements for each.
It is used in this work for the reconstruction of Z boson candidates decaying to muons.
The Tight muon algorithm only allows for global muons.
It is used to identify Tags for measuring muon identification efficiency in the tag-and-probe method of Chapter~\ref{chap:efficiency}.
The specific cut values for the Medium and Tight algorithms are quoted in Table~\ref{tab:muonid}.

\begin{table}[htp]
\begin{center} {\scriptsize
\begin{tabular}{r|l|l|l} \hline\hline
\multirow{2}{*}{Requirement}      & {\em Medium}      & {\em Medium}      & \multirow{2}{*}{\em Tight}   \\
                                  & {\em (global)}    & {\em (tracker)}   & \\\hline
Reconstruction                    & Global muon       & Tracker muon      & Global muon     \\
Fraction of valid tracker hits    & $>0.49$           & $>0.49$           & -               \\
Segment compatibility             & $>0.303$          & $>0.451$          & -               \\
Track kink                        & $<20$             & -                 & -               \\
Normalized global track $\chi^2$  & $<3$              & -                 & $<10$           \\
Tracker-standalone position match & $<12$             & -                 & -               \\
$d_0(\mathrm{vtx})$ (cm)          & -                 & -                 & $<0.2$          \\
$d_z(\mathrm{vtx})$ (cm)          & -                 & -                 & $<0.5$          \\
Number of valid pixel hits        & -                 & -                 & $\geq 1$        \\
Tracker layers with measurement   & -                 & -                 & $\geq 5$        \\
$I_{rel}^{\mu}$                   & $<0.15$           & $<0.15$           & $<0.15$         \\
\hline\hline
\end{tabular}}
\end{center}
\caption{Muon ID requirements.}
\label{tab:muonid}
\end{table}

\section{Jets}
\label{sec:jets}
Due to the nature of QCD, it is impossible to observe a free quark or gluon in the final state.
Suppose two color charged objects are created in a collision and begin to move apart.
The QCD interaction between them is transmitted via gluons.
Since gluons themselves carry color charge, the force between the particles does not fall off as their separation increases.
The gluon field maintaining this force over the ever-increasing distance is contained in a narrow flux tube.
As the distance continues to increase, it becomes energetically favorable to produce a new quark-antiquark pair instead of further lenghtening the flux tube.
Through this so-called ``color confinement'' process, many quarks and antiquarks are formed.
In turn, it is energetically preferred for them to assemble into color-neutral configurations of multiple quarks. This is known as hadronization.
The result of this is that, in a hard process producing a high momentum quark, the observable
experimental signature of that quark is a collimated spray of hadrons. That spray is called a (hadronic) jet.

Much work has been done on the reconstruction and classification of jets.
This is briefly described below.
However, the focus of this work is not in the predominantly hadronic final states.

\subsection{Jet reconstruction}
Jets are reconstructed from all the particle candidates using the anti-$k_{\rm t}$ clustering algorithm~\cite{Cacciari:2008gp}
with a distance parameter of 0.4, as implemented in the FASTJET package~\cite{Cacciari:2011ma,Cacciari:2006gp}.
The reconstruction may be seeded using all reconstructed particle candidates
after having removed the charged hadron candidates which are not associated to the primary vertex of the event
(charge hadron subtracted AK4PFchs).
The energy of the reconstructed jets is corrected in 3 steps: L1FastJet (for pileup/underlying event subtraction),
L2 (for relative corrections), and L3 (for absolute scale corrections).
For data, an extra residual correction is included in the absolute scale correction,
derived by the JetMET working group within the CMS collaboration.

An extra correction is applied for the simulated jets in order to reproduce the measured jet energy resolution.
For each jet the transverse momentum is smeared using using the transverse momentum of
the generator level matched jet and the measured Data/MC resolution ratio. The correction transformation is given by: 

\begin{equation}
p_{\rm T} \rightarrow \max \left[0 , p_{T}^{\rm gen} + c\cdot (p_{\rm T}-p_{\rm T}^{\rm gen}) \right]
\label{eq:jersmear}
\end{equation}
in which $c = \sigma_{\mathrm{data}} / \sigma_{\mathrm{MC}}$ are the data-to-MC resolution scale factors.

\subsection{Jet identification}

We consider for analysis all jets with \pt$>$20~GeV and $|\eta|<5$ passing a set of loose identification requirements given in Table~\ref{tab:loose_jet_id}.
Jets that are within $\Delta R < 0.4$ of one of the identified leptons are disregarded, this is referred to as jet cleaning.
The number of jets is used as a selection variable.
Here, we define it as how many of these nominal jets have \pt$>$30~GeV and $|\eta|<2.4$.
\begin{table}[htbp]
  \begin{center}
 {\small
  \begin{tabular} {ll}
\hline
  Quantity                  & Requirement \\
  \hline
    Neutral Hadron Fraction   & < 0.99      \\
    Neutral EM Fraction       & < 0.99      \\
    Number of Constituents    & > 1         \\
    Charged Hadron Fraction   & > 0         \\
    Charged Multiplicity      & > 0         \\
    Charged EM Fraction       & < 0.99      \\
  \hline
  \end{tabular}
}
  \caption{Loose jet identification criteria for jets having $|\eta|<2.4$. \label{tab:loose_jet_id}}
  \end{center}
\end{table}

\subsection{b-tagging}
\label{ss:btagging}

Jets originating from $b$-quark fragmentation (b-jets) are identified by ``b-tagging.''
B-tagging is useful for enhancing physics signatures with one or more boosted $b$-quarks in the final state.
Such signatures include flavor changing decays of the top quark, and 
resonant Higgs or Z boson decays into $b\bar{b}$.
In this work, the b-tagging technique employed is based on the ``combined secondary vertex'' CSVv2 algorithm~\cite{Sirunyan:2017ezt}.
The algorithm provides high efficiency for b-jets, with a low probability of light-flavor jet misidentification.
The B-Tagging and Vertexing Group within the CMS Experiment is responsible for calibrating the algorithm working points.
Loose, medium, and tight working points of the CSVv2 algorithm were calibrated to provide misidentification probabilities of 10\%, 1\%, and 0.1\%, respectively.
%The b tagging technique employed is based on the ``combined secondary vertex'' CSVv2 algorithm~\cite{Chatrchyan:2012jua,CMS-PAS-BTV-15-001}.
For the studies presented here, b-tagging serves the purpose of rejecting combinatorial
background processes involving top quark production.
The medium and tight CSVv2 working points are used, corresponding to values of 0.8484 and 0.9535.


\section{Photons}
Photon candidates are reconstructed from energy deposits in the ECAL using algorithms
that constrain the clusters to the size and shape expected from a photon~\cite{CMS:EGM-14-001} .
The identification of the candidates is based on shower-shape and isolation variables, and depends on
whether the energy deposit was in the ECAL barrel or endcaps.
For isolated photons, scalar sums of the \pt of PF candidates within a cone of $\Delta R < 0.3$
around the photon candidate are required to be below the bounds defined. Only the PF candidates
that do not overlap with the EM shower of the candidate photon are included in the isolation sums.
The photon candidates used in this analysis are required to have a minimum \pt~of 15~\GeV and to  
be within $|\eta| < 2.5$ passing the loose identification criteria given in Table~\ref{tab:PhotonIDLoose}.

\begin{table}[htbp]
  \begin{center}
 {\footnotesize
  \begin{tabular} {lll}
\hline
Quantity                                   &  ECAL barrel req. & ECAL endcap req.  \\
\hline
Full 5x5 $\sigma_{i\eta i\eta}$            & $< 0.0103$ & $< 0.0301$     \\ \hline 
H/E                                        & $< 0.0597$ & $< 0.0481$     \\ \hline 
charged hadron isolation                   & $< 1.295$  & $< 1.011$      \\ \hline 
\multirow{2}{*}{neutral hadron isolation}  & $< 10.92 + (0.0148\,{\GeV}^{-1})p_T$          & $< 5.931 + (0.0163\,{\GeV}^{-1})p_T$        \\  
                                           & $+ (1.7\times 10^{-5}\,{\GeV}^{-2}){p_T}^2$   & $+(1.4\times 10^{-5}\,{\GeV}^{-2}){p_T}^2$  \\ \hline
photon isolation                           & $< 3.630 + (0.0047\,{\GeV}^{-1})p_T$          & $< 6.641 + (0.0034\,{\GeV}^{-1})p_T$        \\ \hline 
Conversion safe electron veto              & Yes & Yes           \\
\hline
\end{tabular}
}
\caption{Loose photon identification criteria.}
\label{tab:PhotonIDLoose}
  \end{center}
\end{table}


\section{Particle-flow reconstruction}
\label{sec:pf}

The particle-flow (PF) event reconstruction algorithm~\cite{CMS-PRF-14-001} is used.
It is designed to leverage information from all CMS Detector components to reconstruct
and identify individual particles, namely: electrons, muons, photons, and charged and neutral hadrons.
This collection of individual particles is used for reconstructing jets,
determining the missing transverse energy, reconstructing tau leptons from their decay products,
identifying b-jets, and more.

The CMS Detector is well suited for this reconstruction algorithm. 
The strong magnetic field and large silicon tracker allow us to reconstruct charged particle tracks
with high efficiency and low fake rate at transverse momenta as low as 150 MeV, at pseudorapidities as large as $|\eta|=2.6$.
The reconstructed vertex of origin with the largest value of summed track $\pt^2$ is taken to be the primary proton-proton interaction vertex.

Electrons are reconstructed by a combination of a track and of several energy deposits in the
ECAL, from the electron itself and from possible bremsstrahlung photons radiated by the elec-
tron in the tracker material on its way to the ECAL.
Muons are reconstructed and identified inside and outside of jets with very large efficiency and
purity from a combination of the tracker and muon chamber information.
Photons are reconstructed with an excellent energy resolution by the hermetic electromagnetic
calorimeter at pseudorapidities as large as $|\eta|=3.0$.
Together with the magnetic field, the ECAL granularity allows photons to be separated from charged-particle energy deposits by way of shower shape discrimination.
It is functional even in jets with transverse momenta of several hundred GeV.

Charged and neutral hadrons deposit their energy in the HCAL.
The granularity of the HCAL is 25 times coarser than that of the ECAL.
This does not allow charged and neutral hadrons to be spatially separated in
jets with transverse momentum above 100 GeV.
But the combined ECAL-HCAL system has hadron energy resolution of around 10\% at 100 GeV.
This allows neutral hadrons to be detected as an energy excess on top of the energy deposited
by the charged hadrons pointing to the same calorimeter cells. 
Charged hadrons are reconstructed with the superior angular and energy resolution of the tracker.
Particles with pseudorapidities in the range $3<|\eta|<5$ are more coarsely measured with the HF forward calorimeter.

The presence of neutrinos and other weakly-interacting particles can be detected by formulating
the missing transverse energy, described below.

\section{Missing transverse energy}
The missing transverse momentum vector, $\vec{\mathbf{p}}_\mathrm{T}^\mathrm{miss}$, is defined as the projection
of the negative vector sum of the momenta of all reconstructed PF candidates in an event
onto the plane perpendicular to the beams.
Its magnitude is referred to as $\met$.
Since all visible objects contribute to this quantity, the mismeasurement or failed reconstruction
of any object adversely affects its accuracy.
Several event-level filters are applied to discard events with anomalous $\met$ arising from specific well-understood issues with the detector components or event reconstruction~\cite{CMS-PAS-JME-16-004}.
Jet energy corrections are propagated to the missing transverse momentum by
adjusting the momentum vector of the PF candidate constituents of each reconstructed jet.
Any particle which is produced outside the acceptance of the relevant detector subsystems will also be counted towards missing energy, limiting the resolution.


