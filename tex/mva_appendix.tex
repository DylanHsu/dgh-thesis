\chapter{Multivariate analysis validation studies}
\label{app:mva}

\section{Gradient boosting}

\section{Choice of input variables}

\begin{figure}[htbp]
\begin{center}
\includegraphics[width=0.48\textwidth]{figures/mutual_information_MVAvars_Signal.pdf}
\includegraphics[width=0.48\textwidth]{figures/mutual_information_MVAvars_ZZ.pdf}
\includegraphics[width=0.48\textwidth]{figures/mutual_information_MVAvars_WZ.pdf}
\includegraphics[width=0.48\textwidth]{figures/mutual_information_MVAvars_Non-prompt.pdf}
\caption{Mutual information between the twelve BDT input variables for the invisible Higgs, ZZ, WZ, and flavor-symmetric (non-prompt) classes. The matrix for the Drell-Yan class is statistically irrelevant.}
\label{fig:bdt_mutual_information}
\end{center}
\end{figure}

\begin{figure}[htbp]
\begin{center}
\includegraphics[width=0.48\textwidth]{figures/mva_MET_nice.pdf}
\includegraphics[width=0.48\textwidth]{figures/mva_mll_minus_mZ_nice.pdf}
%\includegraphics[width=0.48\textwidth]{figures/mva_balance_nice.pdf}
\includegraphics[width=0.48\textwidth]{figures/mva_ptl1_nice.pdf}
\includegraphics[width=0.48\textwidth]{figures/mva_ptl2_nice.pdf}
\includegraphics[width=0.48\textwidth]{figures/mva_ptll_nice.pdf}
\includegraphics[width=0.48\textwidth]{figures/mva_mTl1MET_nice.pdf}
\caption{Comparison of data and prediction in the BDT input variables after applying the training preselection. Signal strength has been enhanced by a factor of 10.}
\label{fig:bdt_inputvar_histos}
\end{center}
\end{figure}
\begin{figure}[htbp]
\begin{center}
\includegraphics[width=0.48\textwidth]{figures/mva_mTl2MET_nice.pdf}
\includegraphics[width=0.48\textwidth]{figures/mva_delphi_ptll_MET_nice.pdf}
\includegraphics[width=0.48\textwidth]{figures/mva_deltaR_ll_nice.pdf}
%\includegraphics[width=0.48\textwidth]{figures/mva_ptl1mptl2_over_ptll_nice.pdf}
%\includegraphics[width=0.48\textwidth]{figures/mva_cos_theta_star_l1_nice.pdf}
\includegraphics[width=0.48\textwidth]{figures/mva_abs_cos_theta_CS_l1_nice.pdf}
\includegraphics[width=0.48\textwidth]{figures/mva_abs_etal1_nice.pdf}
\includegraphics[width=0.48\textwidth]{figures/mva_abs_etal2_nice.pdf}
\caption{(continued, 1) Comparison of data and prediction in the BDT input variables after applying the training preselection. Signal strength has been enhanced by a factor of 10.}
\label{fig:bdt_inputvar_histos2}
\end{center}
\end{figure}


\section{Mapping the experimental uncertainties}
\label{sec:bdt_toys}
In order to perform a shape analysis in the classifier BDT spectrum, the effect of various nuisance parameters must be propagated to the BDT shape.
It is not sufficient to simply evaluate different shapes at the extremities of the nuisance variations, because a BDT is not a conformal map.
Therefore, we used a toy method to sample the distributions of the relevant nuisance parameters and found the resulting distributions in BDT value.
The nuisances considered to be relevant are the uncertainty of the lepton momentum scale (see Section~\ref{subsec:lepres}) and the uncertainty of missing energy due to the jet energy scale (JES).

The number of random toys used for the uncertainty propagation was 50 due to the high computing cost of many classifier evaluations. 
The lepton scale variations were sampled from a normal distribution with standard deviation of 0.01.
The \met variations were sampled from a normal distribution with standard deviation of 1, then multiplied by the relative size of the jet energy scale effect (this quantity varies per event).
Variations in the lepton scale affected the missing energy which was adjusted; variations in the missing energy affected many of the other variables, which were subsequently adjusted. 

After performing this procedure for all simulated events, we take the uncertainty bands from the non-normal toy distributions as the distance between the 15.9\% and 84.1\% quantiles.
These are the so-called $\pm1\sigma$ quantiles of a normal distribution.

Figures~\ref{fig:bdt_toy_envelopes_electron},~\ref{fig:bdt_toy_envelopes_muon}, and~\ref{fig:bdt_toy_envelopes_MET} show 2D maps of the nuisance variations with the BDT value on the horizontal axis and the relative variation from the nominal BDT bin yield on the vertical axis.
Figures~\ref{fig:bdt_electron_scale},~\ref{fig:bdt_muon_scale}, and~\ref{fig:bdt_MET_scale} show the resulting uncertainty shapes that enter the likelihood fit, for the electron scale, muon scale, and MET scale nuisances respectively. 
In most combinations of background process and nuisance parameter, the propagated uncertainty is irrelevant compared to the statistical uncertainty from the number of simulated events.

We do not propagate this uncertainty for the non-resonant backgrounds or the Drell-Yan process, 
since they are not highly signal-like and the other extrapolation uncertainties are sufficiently conservative.
Furthermore, when performing the likelihood fit to determine the exclusion limits, 
we ignore these propagated uncertainties in the ratio of the diboson control regions, 
and fully correlate them across the shape bins in the signal region. 

\begin{figure}[htbp]
\begin{center}
\includegraphics[width=0.48\textwidth]{figures/syst_BDT_ZH_hinv_sm_toyenvelope_electron.pdf}
\includegraphics[width=0.48\textwidth]{figures/syst_BDT_ggZH_hinv_toyenvelope_electron.pdf}
\includegraphics[width=0.48\textwidth]{figures/syst_BDT_ZZ_toyenvelope_electron.pdf}
\includegraphics[width=0.48\textwidth]{figures/syst_BDT_WZ_toyenvelope_electron.pdf}
\includegraphics[width=0.48\textwidth]{figures/syst_BDT_VVV_toyenvelope_electron.pdf}
\caption{2D maps of the relative toy variations from the nominal BDT shape versus the BDT value, for the electron scale. The hashed bands represent statistical uncertainty on the simulated events.}
\label{fig:bdt_toy_envelopes_electron}
\end{center}
\end{figure}

\begin{figure}[htbp]
\begin{center}
\includegraphics[width=0.48\textwidth]{figures/syst_BDT_ZH_hinv_sm_toyenvelope_muon.pdf}
\includegraphics[width=0.48\textwidth]{figures/syst_BDT_ggZH_hinv_toyenvelope_muon.pdf}
\includegraphics[width=0.48\textwidth]{figures/syst_BDT_ZZ_toyenvelope_muon.pdf}
\includegraphics[width=0.48\textwidth]{figures/syst_BDT_WZ_toyenvelope_muon.pdf}
\includegraphics[width=0.48\textwidth]{figures/syst_BDT_VVV_toyenvelope_muon.pdf}
\caption{2D maps of the relative toy variations from the nominal BDT shape versus the BDT value, for the muon scale. The hashed bands represent statistical uncertainty on the simulated events.}
\label{fig:bdt_toy_envelopes_muon}
\end{center}
\end{figure}

\begin{figure}[htbp]
\begin{center}
\includegraphics[width=0.48\textwidth]{figures/syst_BDT_ZH_hinv_sm_toyenvelope_MET.pdf}
\includegraphics[width=0.48\textwidth]{figures/syst_BDT_ggZH_hinv_toyenvelope_MET.pdf}
\includegraphics[width=0.48\textwidth]{figures/syst_BDT_ZZ_toyenvelope_MET.pdf}
\includegraphics[width=0.48\textwidth]{figures/syst_BDT_WZ_toyenvelope_MET.pdf}
\includegraphics[width=0.48\textwidth]{figures/syst_BDT_VVV_toyenvelope_MET.pdf}
\caption{2D maps of the relative toy variations from the nominal BDT shape versus the BDT value, for the \met scale due to the JES uncertainty. The hashed bands represent statistical uncertainty on the simulated events.}
\label{fig:bdt_toy_envelopes_MET}
\end{center}
\end{figure}

\begin{figure}[htbp]
\begin{center}
\includegraphics[width=0.48\textwidth]{figures/syst_BDT_ZH_hinv_sm_toys_electron.pdf}
\includegraphics[width=0.48\textwidth]{figures/syst_BDT_ggZH_hinv_toys_electron.pdf}
\includegraphics[width=0.48\textwidth]{figures/syst_BDT_ZZ_toys_electron.pdf}
\includegraphics[width=0.48\textwidth]{figures/syst_BDT_WZ_toys_electron.pdf}
\includegraphics[width=0.48\textwidth]{figures/syst_BDT_VVV_toys_electron.pdf}
\caption{Uncertainty shapes calculated from the toy method for the electron scale.}
\label{fig:bdt_electron_scale}
\end{center}
\end{figure}

\begin{figure}[htbp]
\begin{center}
\includegraphics[width=0.48\textwidth]{figures/syst_BDT_ZH_hinv_sm_toys_muon.pdf}
\includegraphics[width=0.48\textwidth]{figures/syst_BDT_ggZH_hinv_toys_muon.pdf}
\includegraphics[width=0.48\textwidth]{figures/syst_BDT_ZZ_toys_muon.pdf}
\includegraphics[width=0.48\textwidth]{figures/syst_BDT_WZ_toys_muon.pdf}
\includegraphics[width=0.48\textwidth]{figures/syst_BDT_VVV_toys_muon.pdf}
\caption{Uncertainty shapes calculated from the toy method for the muon scale.}
\label{fig:bdt_muon_scale}
\end{center}
\end{figure}

\begin{figure}[htbp]
\begin{center}
\includegraphics[width=0.48\textwidth]{figures/syst_BDT_ZH_hinv_sm_toys_MET.pdf}
\includegraphics[width=0.48\textwidth]{figures/syst_BDT_ggZH_hinv_toys_MET.pdf}
\includegraphics[width=0.48\textwidth]{figures/syst_BDT_ZZ_toys_MET.pdf}
\includegraphics[width=0.48\textwidth]{figures/syst_BDT_WZ_toys_MET.pdf}
\includegraphics[width=0.48\textwidth]{figures/syst_BDT_VVV_toys_MET.pdf}
\caption{Uncertainty shapes calculated from the toy method for the \met due to the JES uncertainty.}
\label{fig:bdt_MET_scale}
\end{center}
\end{figure}


