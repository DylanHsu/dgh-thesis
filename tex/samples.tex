\chapter{Data samples}
\section{Experimental data}
This analysis uses a sample of pp collisions collected in 2016 with the CMS 
experiment at the LHC at $\sqrt{s} = 13~\rm{TeV}$. Five primary datasets  
are used to ensure a very high trigger efficiency: MuEG, DoubleMuon, 
DoubleElectron, SingleMuon, and SingleElectron. All data are analyzed using the
\\
\centerline{\small Cert\_271036-284044\_13TeV\_23Sep2016ReReco\_Collisions16\_JSON.txt}
\\
file in order to select the good luminosity sections. 
The total integrated luminosity with the associated uncertainty is $\usedLumiWithSyst$.

\subsection{Silicon strip hit efficiency loss}
\label{ss:hips}
During 2015, a reduced silicon strip hit efficiency and a reduced tracking efficiency were observed.
It was found to be correlated with the increase of the LHC instantaneous luminosity.
This is due to the effect of highly ionizing particles (HIPs) which were studied in the test beams
before the tracker construction. A HIP saturates the readout chips, or APVs, of the hit strips.
The baseline is shifted toward negative vlues, and several of the strips from the hit cluster saturate.
The chips are fully blinded in the next few bunch crossings, and remain partially blind for some time after, until they fully recover.

In total, as the instantaneous luminosity continued to increase, we observed a lower cluster charge, lower signal-to-noise ratio, lower hit efficiency, shorter tracks, and lower track efficiency.
The tracker modules most affected were wider and of higher occupancy. %mainly TOB L1 and L2. 
Any particles identified using tracks were affected by the so-called "HIP effect."
In particular, the electrons and muons are reconstructed from tracks and they are a crucial part of this work.

The solution was found in the electronics parameters of the silicon strip readout chips.
In particular, the preamplifier feedback voltage bias (VFP) affects the duration
of the signal at the output of the preamplifier.
Reducing the VFP from 0.3 Volts to 0 Volts was found to mitigate the HIP effect. 

The data taken in 2016 suffered from the HIP effect until a task force was
convened to mitigate the decline in tracking performance.
These 2016 data were subdivided into Run Eras labeled alphabetically.
The Run Eras used in this work are B through H. 

The HIP mitigation fix was deployed close to the demarcation of Run Eras F and G,
in the data run numbered 278803.
Run Era F comprised runs numbered (277772, 278808) and Run Era G comprised runs numbered (278820, 280385).
Due to the era-dependent nature of the HIP effect, which had a significant effect on the lepton identification, 
separate empirical corrections for leptons were derived.
They are discussed later in Chapter~\ref{chap:efficiency}.
For brevity, the corrections intended to represent the data before the fix are denoted as
"B-F" or "BCDEF", despite the fact that the mitigation did not occur precisely at the Run Era changeover.
Similarly, the corrections representing the data after the fix are denoted as "G-H" or "GH." 

\section{Simulated samples}
\label{sec:mcsamples}
Several Monte Carlo (MC) event generators are used to simulate the signal and
background processes. For all processes, the detector response is simulated using a detailed
description of the CMS detector, based on the \textsc{GEANT4} 
package~\cite{Agostinelli:2002hh}, and event reconstruction is performed with
the same algorithms as used for data.
The simulated samples include additional interactions per bunch crossing (pileup).
The simulated events are weighted so that the pileup distribution matches the data,
with an average pileup of about 25 interactions per bunch crossing.

\subsection{Standard Model processes}
Resonant Z boson background processes ($\W\Z$, $\Z\Z$, tribosons, etc.) are
estimated using Monte Carlo (MC) samples. Nonresonant background processes
($\ttbar$, $\tw$, $\W\W$, $\dytt$, etc.) are estimated using $\Pe\mu$ data events.

The $\W\Z$ and $\Z\Z$ processes, via $\Pq\Paq$ annihilation, 
are generated at next-to-leading-order (NLO) with 
\textsc{POWHEG2.0}~\cite{Alioli:2008gx,Nason:2004rx,Frixione:2007vw,powheg:2010}. 
The $\Pg\Pg \to \Z\Z$ process is simulated with MCFM~\cite{MCFM}. 
The $\Z\gamma$, $\ttbar \Z$, $\W\W\Z$, $\W\Z\Z$, and $\Z\Z\Z$ 
processes are generated with \textsc{MadGraph5\_aMC@NLO}~\cite{Alwall:2014hca}.
The signal samples are simulated using \textsc{MadGraph5\_aMC@NLO} at next-to-leading-order (NLO), 
\textsc{MadGraph5} at leading-order (LO), and \textsc{POWHEG} at NLO. The 
default MC generator is \textsc{MadGraph5\_aMC@NLO}. 
The \textsc{PYTHIA8}~\cite{Sjostrand:2006za,Sjostrand:2015} package is used 
for parton showering, hadronization, and the underlying event simulation,
with tune CUETP8M1.
The NNPDF 3.0~\cite{nnpdf} set is used as the default set of parton distribution 
functions (PDFs). 
The processed dataset names and cross sections for the Standard Model processes considered 
in the analysis are shown in Table~\ref{tab:mcprocess}.
Simulated datasets from the RunIISummer16MiniAODv2 campaign are used.
\clearpage
\begin{sidewaystable}[hbtp]
  \caption{Processed dataset names and cross sections for the Standard Model processes considered in the analysis.
           The processes are grouped in the following way: Resonant Diboson, Triboson, Nonresonant, and Drell-Yan.\label{tab:mcprocess}}
    \begin{center}
    {\scriptsize
      \begin{tabular}{|p{3.0cm}|p{13cm}|p{3.5cm}|}
      \hline 
       \bf{Process} & \bf{Simulated sample name} & \bf{Cross section [pb]}\\
       %\multicolumn{3}{|c|}{Background} \\
       \hline 
       $\W\Z\to 3\ell\nu$               & WZTo3LNu\_TuneCUETP8M1\_13TeV-powheg-pythia8 & 4.42965 $\times$ 1.109 \\ 
       $\W\Z\to 2\ell 2q$               & WZTo2L2Q\_13TeV\_amcatnloFXFX\_madspin\_pythia8 & 5.595 $\times$ 1.109 \\ 
       $\Z\Z\to 4\ell$                  & ZZTo4L\_13TeV\_powheg\_pythia8 & 1.256 \\ 
       $gg \to \Z\Z \to 4\mu$           & GluGluToContinToZZTo4mu\_13TeV\_MCFM701\_pythia8     & 0.001586 $\times$ 2.3  \\
       $gg \to \Z\Z \to 4e$             & GluGluToContinToZZTo4e\_13TeV\_MCFM701\_pythia8      & 0.001586 $\times$ 2.3  \\
       $gg \to \Z\Z \to 4\tau$          & GluGluToContinToZZTo4tau\_13TeV\_MCFM701\_pythia8    & 0.001586 $\times$ 2.3  \\
       $gg \to \Z\Z \to 2\mu 2e$        & GluGluToContinToZZTo2e2mu\_13TeV\_MCFM701\_pythia8   & 0.003194 $\times$ 2.3  \\
       $gg \to \Z\Z \to 2\mu 2\tau$     & GluGluToContinToZZTo2mu2tau\_13TeV\_MCFM701\_pythia8 & 0.003194 $\times$ 2.3  \\
       $gg \to \Z\Z \to 2e 2\tau$       & GluGluToContinToZZTo2e2tau\_13TeV\_MCFM701\_pythia8  & 0.003194 $\times$ 2.3  \\
       $gg \to \Z\Z \to 2e 2\nu$        & GluGluToContinToZZTo2mu2nu\_13TeV\_MCFM701\_pythia8  & 0.001720 $\times$ 2.3  \\
       $gg \to \Z\Z \to 2\mu 2\nu$       & GluGluToContinToZZTo2e2nu\_13TeV\_MCFM701\_pythia8   & 0.001720 $\times$ 2.3  \\
       $\Z\Z\to 2\ell 2\nu$             & ZZTo2L2Nu\_13TeV\_powheg\_pythia8 & 0.564 \\ 
       $\Z\Z\to 2\ell 2q$               & ZZTo2L2Q\_13TeV\_amcatnloFXFX\_madspin\_pythia8 & 3.220 \\ 
       $\Z+\gamma$                      & ZGTo2LG\_TuneCUETP8M1\_13TeV-amcatnloFXFX-pythia8 &  117.864 \\
       \hline 
       $\W\Z\Z$                         & WZZ\_TuneCUETP8M1\_13TeV-amcatnlo-pythia8 & 0.05565 \\ 
       $\W\W\Z$                         & WWZ\_TuneCUETP8M1\_13TeV-amcatnlo-pythia8 & 0.16510 \\ 
       $\Z\Z\Z$                         & ZZZ\_TuneCUETP8M1\_13TeV-amcatnlo-pythia8 & 0.01398 \\ 
       \hline 
       $\Z \to \tau\tau \to e\mu$       & DYJetsToTauTau\_ForcedMuEleDecay\_M-50\_TuneCUETP8M1\_13TeV-amcatnloFXFX-pythia8 & $1921.8 * (0.1741 + 0.1783)^2$ \\
       $qq \to \W\W\to 2\ell 2\nu$      & WWTo2L2Nu\_13TeV-powheg & (118.7-3.974) $\times$ 0.1086 $\times$ 0.1086 $\times$ 9 \\ 
       $gg \to \W\W\to 2\ell 2\nu$      & GluGluWWTo2L2Nu\_MCFM\_13TeV & (3.974 $\times$ 0.1086 $\times$ 0.1086 $\times$ 9 $\times$ 1.4 \\ 
       $\ttbar\Z(\ell\ell+\nu\nu)$      & TTZToLLNuNu\_M-10\_TuneCUETP8M1\_13TeV-amcatnlo-pythia8 & 0.2529 \\ 
       $\ttbar\Z(qq)$                   & TTZToLLNuNu\_M-10\_TuneCUETP8M1\_13TeV-amcatnlo-pythia8 & 0.5297 \\ 
       $\ttbar\W(\ell\nu)$              & TTWJetsToLNu\_TuneCUETP8M1\_13TeV-amcatnloFXFX-madspin-pythia8 & 0.2043 \\ 
       $\ttbar\W(qq)$                   & TTWJetsToQQ\_TuneCUETP8M1\_13TeV-amcatnloFXFX-madspin-pythia8 & 0.4062 \\ 
       $\ttbar \to 2\ell 2\nu 2b$       & TTTo2L2Nu\_13TeV-powheg & 831.76 $\times$ 0.1086 $\times$ 0.1086 $\times$ 9 \\ 
       $\tw$                            & ST\_tW\_top\_5f\_inclusiveDecays\_13TeV-powheg-pythia8 & 35.6 \\ 
       $\ensuremath{\bar{\mathrm{t}}\mathrm{W}}$          & ST\_tW\_antitop\_5f\_inclusiveDecays\_13TeV-powheg-pythia8 & 35.6 \\ 
       \hline 
       %\multicolumn{3}{c}{Signal} \\
       %\hline 
       \multirow{9}{*}{$\dyll$}
       %$\dyll$  \textsc{MadGraph5\_aMC@NLO} & DYJetsToLL\_M-50\_TuneCUETP8M1\_13TeV-amcatnloFXFX-pythia8   & 2008.4 $\times$ 3 \\ 
       %$\dyll$ \textsc{MadGraph5}          & DYJetsToLL\_M-50\_TuneCUETP8M1\_13TeV-madgraphMLM            & 2008.4 $\times$ 3 \\ 
       %$\dyll$ \textsc{powheg}             & ZToLL\_NNPDF30\_13TeV-powheg\_M\_50\_120                     & 1975.0 $\times$ 2 \\ 
       & DYJetsToLL\_M-50\_TuneCUETP8M1\_13TeV-amcatnloFXFX-pythia8         & 2008 $\times$ 3 \\ 
       & DYJetsToLL\_M-10to50\_TuneCUETP8M1\_13TeV-amcatnloFXFX-pythia8 & 2008.4 $\times$ 3 $\times$ 3.78 \\ 
       & DYJetsToLL\_M-50\_TuneCUETP8M1\_13TeV-madgraphMLM                  & 2008 $\times$ 3 \\ 
       & ZToLL\_NNPDF30\_13TeV-powheg\_M\_50\_120                           & 1975 $\times$ 2 \\ 
       & DYJetsToLL\_Pt-50To100\_TuneCUETP8M1\_13TeV-amcatnloFXFX-pythia8   & 375             \\ 
       & DYJetsToLL\_Pt-100To250\_TuneCUETP8M1\_13TeV-amcatnloFXFX-pythia8  & 86.5            \\ 
       & DYJetsToLL\_Pt-250To400\_TuneCUETP8M1\_13TeV-amcatnloFXFX-pythia8  & 3.32            \\ 
       & DYJetsToLL\_Pt-400To650\_TuneCUETP8M1\_13TeV-amcatnloFXFX-pythia8  & 0.449           \\ 
       & DYJetsToLL\_Pt-650ToInf\_TuneCUETP8M1\_13TeV-amcatnloFXFX-pythia8  & 0.0422          \\ 
       \hline 

         \end{tabular}
         }
           \end{center}
           \end{sidewaystable}

\subsection{Dark matter hypotheses}
Samples of simulated DM particle events in the simplified model are generated using \textsc{MadGraph5\_aMC@NLO 2.2.2}~\cite{Alwall:2014hca} at leading order (LO) and matched to
\PYTHIA8.205~\cite{Sjostrand:2007gs} using tune CUETP8M1 for parton showering and hadronization~\cite{Khachatryan:2015pea,Skands:2014pea}.
The factorization and re\-nor\-ma\-li\-zat\-ion scales are set to the geometric mean of $\sqrt{\pt^2+m^2}$ for all final-state particles~\cite{Alwall:2014hca,Abercrombie:2015wmb}, where $\pt$ and $m$ are the transverse momentum and mass of each particle.

For the simplified model of DM production, couplings are chosen according to the recommendations in Ref.~\cite{Boveia:2016mrp}.
The coupling $g_{\chi}$ is set to one. For $g_{\Pq}$, values of $1.0$ and $0.25$ are considered. The width of the mediator is assumed to be determined exclusively by the contributions from the couplings to quarks and the DM particle $\chi$. Under this assumption, the width ranges 1--5\% (30--50\%) of the mediator mass for $g_{\Pq}=0.25$ ($g_{\Pq}=1.00$).
The signal simulation samples with $g_{\Pq}=1.0$ are processed using the detector simulation described below.
Signal predictions for $g_{\Pq}=0.25$ are obtained by applying event weights based on the \ETm distribution at the generator level to the fully simulated samples with $g_{\Pq}=1.0$.
This procedure takes into account the nontrivial dependence of the mediator width on the coupling choice~\cite{Boveia:2016mrp}.  The exact dependence of the width on the model parameters is reported in~\cite{Boveia:2016mrp}.

Events for the ADD extra-dimension scenario are generated at LO using an effective field theory (EFT) implementation in \PYTHIA8~\cite{Ask:2008fh,Ask:2009pv}. Event samples are produced for $M_{D} = 1$, 2 and 3 \TeV, each with $n =$ 2, 3, 4, 5, 6, 7.
The signal is truncated for $\hat{s} > M_{D}^2$ in order to ensure the validity of the EFT.

The events for the unparticle model are generated at leading-order with \PYTHIA8~\cite{Ask:2008fh,Ask:2009pv}
assuming a cutoff scale $\Lambda_{\textsf{U}}=15\TeV$, 
using tune CUETP8M1 for parton showering and hadronization. 
We evaluate other values of $\Lambda_{\textsf{U}}$ by rescaling the cross sections as needed.
The parameter $\Lambda_{\textsf{U}}$ acts solely as a scaling factor
for the cross section and does not influence the kinematic distributions of unparticle production~\cite{Ask:2009pv}.

The $\Z\Hi$ production modes via $\Pq\Paq$ annihilation and gluon-gluon fusion are modeled the same way as for the $\W\Z$ and $\Z\Z$ processes.

A comparison of kinematic spectra for various signal models is shown in Figs.~\ref{fig:signals} and~\ref{fig:moreSignals}.

\begin{figure}[htbp]
  \centering
  \includegraphics[width=0.48\textwidth]{figures/signals_ptll_presel.pdf}
  \includegraphics[width=0.48\textwidth]{figures/signals_balance_presel.pdf}
  \includegraphics[width=0.48\textwidth]{figures/signals_ptl1_fullsel.pdf}
  \includegraphics[width=0.48\textwidth]{figures/signals_ptl2_fullsel.pdf}
  \includegraphics[width=0.48\textwidth]{figures/signals_dphiZMET_nminusone.pdf}
  \includegraphics[width=0.48\textwidth]{figures/signals_met.pdf}
  \caption{Comparison of kinematic distributions for a variety of the signal hypotheses. The full distributions are normalized to 1.
  Top left: dilepton $\pt$ (in GeV) in $\zll$ events with $\pt^{\ell\ell} > 60~\GeV$ and $\met > 40~\GeV$. Top right: 
  $|\met-\pt^{\ell\ell}|/\pt^{\ell\ell}$ in $\zll$ events with $\pt^{\ell\ell} > 60~\GeV$ and $\met > 40~\GeV$.
  Center left: leading lepton $\pt$ (in GeV) at final selection level. Center right: subleading lepton $\pt$ (in GeV) at final selection level.
  Bottom left: $\Delta \phi_{\ell\ell-\met}$ at final selection level. Bottom right: the final $\met$ shape used for the shape analysis.}
  \label{fig:signals}
\end{figure}

\begin{figure}[htbp]
  \centering
  \includegraphics[width=0.48\textwidth]{figures/compare_add.pdf}
  \includegraphics[width=0.48\textwidth]{figures/compare_unparticle.pdf}
  \caption{
    Comparison of reconstructed dilepton transverse momentum ($\pt^{\ell\ell}$) distributions for ADD (left) and unparticle (right) models.
    For the ADD models, cross sections are normalized to $1\pb$.
  }
  \label{fig:moreSignals}
\end{figure}

\section{Corrections to the simulated samples}
\subsection{Pileup reweighting}
\label{subsec:puweights}

To match the expected number of pileup interactions in simulation with data,
the reweighting is performed using the minimum bias cross section of 69mb with an uncertainty of 5\%. 
The number of vertices distribution in data and simulation in an inclusive $\Z \to \ell\ell$ sample after the pileup reweighting 
is shown in Figure~\ref{fig:pileup_distribution}.

\begin{figure}[htbp]
  \centering
  \includegraphics[width=0.48\textwidth]{figures/zll_nvtx_met40_zpt60.pdf}
  \caption{
    Number of reconstructed vertices
    for $\Z \to \ell\ell$ events with $\pt^{\ell\ell} > 60~\GeV$ and $\met > 40~\GeV$.
  }
  \label{fig:pileup_distribution}
\end{figure}

The lepton efficiency studies shown in Chapter~\ref{chap:efficiency} split the data sample in two,
at the point where the HIP effect discussed in Section~\ref{ss:hips} was mitigated.
For these studies, the pileup profile of the relevant simulated samples was reweighted to
the pileup profiles of the two subsets of the data samples, which were substantially different.

\subsection{Lepton selection efficiencies}
One of the crucial elements in the analysis is the determination of the lepton efficiency scale factors.
Algorithms for selecting good leptons can have different efficiencies in data and simulation.
The ratio of the selection efficiencies in data and simulation constitutes the scale factor.
The scale factors are computed in many bins of electron and muon kinematics to account for relevant dependent effects.
Then, the simulated event weights are multiplied once by the appropriate scale factor for every
lepton identified in that event. A detailed discussion of the methodology to determine these efficiencies is given in Chapter~\ref{chap:efficiency}.

\subsection{Lepton momentum scale and resolution}
\label{subsec:lepres}
The lepton momentum scale and resolution are affected by detector misalignment 
and miscalibration. The calibration procedure includes corrections to both 
data and simulation. In data momentum corrections are applied differentially 
in $\pt$, $\eta$, and charge in order to match the value of known resonances. 
In simulation additional smearing is applied in order to match the observed 
resolution of the resonance peak.

For muons, these corrections are known as the Rochester corrections.
The additional effect of uncertainty in the magnetic field is considered.
The methodology and validation is provided in Ref.~\cite{Bodek:2012id}.

\subsection{Higher order corrections}
\label{sec:higher-order-corrections}
The diboson MC samples, including the invisible Higgs signal, are initially generated at electroweak leading-order.

For the $\Z\Z$ background, a higher order electroweak correction is applied following~\cite{Bierweiler:2013dja,Gieseke:2014gka} (e.g. see
  Fig~\ref{fig:ewkCorrectionCompare}). An event-by-event reweighting is performed with correction weights binned in the generator-level \pt of the trailing boson.
This correction results in a net reduction of the overall ZZ yield of about 10\%, although with a strong dependence on the trailing boson $\pt$.
Overall, the $\met$ spectrum becomes softer, with a correction of down to $-40\%$ for a trailing-$\Z \pt$ of 700~$\GeV$. 
Moreover, for the $\Pq\Pq \to \Z\Z$ process, a QCD NLO to NNLO (next-to-next-to-leading order) correction factor 
is applied as a function of generator-level invariant mass of the two $\Z$ bosons.
The correction is obtained using settings similar to the CMS simulation (dilepton mass requirements, parton distribution function sets) to ensure its accuracy.
For further details, see \autoref{chap:dibosons} and consult reference~\cite{Grazzini:2015hta}. 

In the case of the $\W\Z$ process, a flat correction factor from QCD NLO to NNLO is applied. Its value is 1.109. See reference~\cite{Grazzini:2016swo}.
Following~\cite{Baglio:113005}, the corresponding NLO EWK correction to WZ production is small, and neglected as a systematic uncertainty.`

Finally, for the invisible Higgs boson signal, we apply a differential NLO electroweak correction as a function of the transverse momentum of the $\Z$ boson.
This correction only concerns the Drell-Yan-like $\Z\Hi$ production via $\Pq\Paq$ annihilation.
The small contributions to the overall $\Z\Hi$ cross section from the gluon-induced, photon-induced, and top-loop production are not corrected in this way.

\begin{figure}[htbp]
\centering
\includegraphics[width=0.60\textwidth]{figures/ZZ_ewkCorr.png}
\caption{Electroweak corrections applied as a function of $\hat{s}$, 
  $\hat{t}$, and of the quark flavours, provided by the authors of 
  Ref.~\cite{Bierweiler:2013dja}. The corrections thus computed are
  plotted as a function of the lower-boson \pt and compared to the
  average values of the corrections shown in Ref.~\cite{Bierweiler:2013dja} 
  for the $\ZZ$ process. The cross-shaped markers are obtained by
  applying the fully differential corrections, whereas the colored
  lines are taken from Ref.~\cite{Bierweiler:2013dja}.} 
\label{fig:ewkCorrectionCompare}
\end{figure}



