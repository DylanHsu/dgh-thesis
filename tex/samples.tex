\chapter{Data samples}
\section{Experimental data}
This analysis uses a sample of pp collisions collected in 2016 with the CMS 
experiment at the LHC at $\sqrt{s} = 13~\rm{TeV}$. Five primary datasets  
are used to ensure a very high trigger efficiency: MuEG, DoubleMuon, 
DoubleElectron, SingleMuon, and SingleElectron. All data are analyzed using the
\\
\centerline{\small Cert\_271036-284044\_13TeV\_23Sep2016ReReco\_Collisions16\_JSON.txt}
\\
file in order to select the good luminosity sections. 
The total integrated luminosity with the associated uncertainty is $\usedLumiWithSyst$.

\section{Simulated samples}
Several Monte Carlo (MC) event generators are used to simulate the signal and
background processes. For all processes, the detector response is simulated using a detailed
description of the CMS detector, based on the \textsc{GEANT4} 
package~\cite{Agostinelli:2002hh}, and event reconstruction is performed with
the same algorithms as used for data.
The simulated samples include additional interactions per bunch crossing (pileup).
The simulated events are weighted so that the pileup distribution matches the data,
with an average pileup of about 25 interactions per bunch crossing.

\subsection{Standard Model processes}
Resonant Z boson background processes ($\W\Z$, $\Z\Z$, tribosons, etc.) are
estimated using Monte Carlo (MC) samples. Nonresonant background processes
($\ttbar$, $\tw$, $\W\W$, $\dytt$, etc.) are estimated using $\Pe\mu$ data events.

The $\W\Z$ and $\Z\Z$ processes, via $\Pq\Paq$ annihilation, 
are generated at next-to-leading-order (NLO) with 
\textsc{POWHEG2.0}~\cite{Alioli:2008gx,Nason:2004rx,Frixione:2007vw,powheg:2010}. 
The $\Pg\Pg \to \Z\Z$ process is simulated with MCFM~\cite{MCFM}. 
The $\Z\gamma$, $\ttbar \Z$, $\W\W\Z$, $\W\Z\Z$, and $\Z\Z\Z$ 
processes are generated with \textsc{MadGraph5\_aMC@NLO}~\cite{Alwall:2014hca}.
The signal samples are simulated using \textsc{MadGraph5\_aMC@NLO} at next-to-leading-order (NLO), 
\textsc{MadGraph5} at leading-order (LO), and \textsc{POWHEG} at NLO. The 
default MC generator is \textsc{MadGraph5\_aMC@NLO}. 
The \textsc{PYTHIA8}~\cite{Sjostrand:2006za,Sjostrand:2015} package is used 
for parton showering, hadronization, and the underlying event simulation,
with tune CUETP8M1.
The NNPDF 3.0~\cite{nnpdf} set is used as the default set of parton distribution 
functions (PDFs). 
The processed dataset names and cross sections for the Standard Model processes considered 
in the analysis are shown in Table~\ref{tab:mcprocess}.
Simulated datasets from the RunIISummer16MiniAODv2 campaign are used.
\clearpage
\begin{sidewaystable}[hbtp]
  \caption{Processed dataset names and cross sections for the Standard Model processes considered in the analysis.
           The processes are grouped in the following way: Resonant Diboson, Triboson, Nonresonant, and Drell-Yan.\label{tab:mcprocess}}
    \begin{center}
    {\scriptsize
      \begin{tabular}{|p{3.0cm}|p{13cm}|p{3.5cm}|}
      \hline 
       \bf{Process} & \bf{Simulated sample name} & \bf{Cross section [pb]}\\
       %\multicolumn{3}{|c|}{Background} \\
       \hline 
       $\W\Z\to 3\ell\nu$               & WZTo3LNu\_TuneCUETP8M1\_13TeV-powheg-pythia8 & 4.42965 $\times$ 1.109 \\ 
       $\W\Z\to 2\ell 2q$               & WZTo2L2Q\_13TeV\_amcatnloFXFX\_madspin\_pythia8 & 5.595 $\times$ 1.109 \\ 
       $\Z\Z\to 4\ell$                  & ZZTo4L\_13TeV\_powheg\_pythia8 & 1.256 \\ 
       $gg \to \Z\Z \to 4\mu$           & GluGluToContinToZZTo4mu\_13TeV\_MCFM701\_pythia8     & 0.001586 $\times$ 2.3  \\
       $gg \to \Z\Z \to 4e$             & GluGluToContinToZZTo4e\_13TeV\_MCFM701\_pythia8      & 0.001586 $\times$ 2.3  \\
       $gg \to \Z\Z \to 4\tau$          & GluGluToContinToZZTo4tau\_13TeV\_MCFM701\_pythia8    & 0.001586 $\times$ 2.3  \\
       $gg \to \Z\Z \to 2\mu 2e$        & GluGluToContinToZZTo2e2mu\_13TeV\_MCFM701\_pythia8   & 0.003194 $\times$ 2.3  \\
       $gg \to \Z\Z \to 2\mu 2\tau$     & GluGluToContinToZZTo2mu2tau\_13TeV\_MCFM701\_pythia8 & 0.003194 $\times$ 2.3  \\
       $gg \to \Z\Z \to 2e 2\tau$       & GluGluToContinToZZTo2e2tau\_13TeV\_MCFM701\_pythia8  & 0.003194 $\times$ 2.3  \\
       $gg \to \Z\Z \to 2e 2\nu$        & GluGluToContinToZZTo2mu2nu\_13TeV\_MCFM701\_pythia8  & 0.001720 $\times$ 2.3  \\
       $gg \to \Z\Z \to 2\mu 2\nu$       & GluGluToContinToZZTo2e2nu\_13TeV\_MCFM701\_pythia8   & 0.001720 $\times$ 2.3  \\
       $\Z\Z\to 2\ell 2\nu$             & ZZTo2L2Nu\_13TeV\_powheg\_pythia8 & 0.564 \\ 
       $\Z\Z\to 2\ell 2q$               & ZZTo2L2Q\_13TeV\_amcatnloFXFX\_madspin\_pythia8 & 3.220 \\ 
       $\Z+\gamma$                      & ZGTo2LG\_TuneCUETP8M1\_13TeV-amcatnloFXFX-pythia8 &  117.864 \\
       \hline 
       $\W\Z\Z$                         & WZZ\_TuneCUETP8M1\_13TeV-amcatnlo-pythia8 & 0.05565 \\ 
       $\W\W\Z$                         & WWZ\_TuneCUETP8M1\_13TeV-amcatnlo-pythia8 & 0.16510 \\ 
       $\Z\Z\Z$                         & ZZZ\_TuneCUETP8M1\_13TeV-amcatnlo-pythia8 & 0.01398 \\ 
       \hline 
       $\Z \to \tau\tau \to e\mu$       & DYJetsToTauTau\_ForcedMuEleDecay\_M-50\_TuneCUETP8M1\_13TeV-amcatnloFXFX-pythia8 & $1921.8 * (0.1741 + 0.1783)^2$ \\
       $qq \to \W\W\to 2\ell 2\nu$      & WWTo2L2Nu\_13TeV-powheg & (118.7-3.974) $\times$ 0.1086 $\times$ 0.1086 $\times$ 9 \\ 
       $gg \to \W\W\to 2\ell 2\nu$      & GluGluWWTo2L2Nu\_MCFM\_13TeV & (3.974 $\times$ 0.1086 $\times$ 0.1086 $\times$ 9 $\times$ 1.4 \\ 
       $\ttbar\Z(\ell\ell+\nu\nu)$      & TTZToLLNuNu\_M-10\_TuneCUETP8M1\_13TeV-amcatnlo-pythia8 & 0.2529 \\ 
       $\ttbar\Z(qq)$                   & TTZToLLNuNu\_M-10\_TuneCUETP8M1\_13TeV-amcatnlo-pythia8 & 0.5297 \\ 
       $\ttbar\W(\ell\nu)$              & TTWJetsToLNu\_TuneCUETP8M1\_13TeV-amcatnloFXFX-madspin-pythia8 & 0.2043 \\ 
       $\ttbar\W(qq)$                   & TTWJetsToQQ\_TuneCUETP8M1\_13TeV-amcatnloFXFX-madspin-pythia8 & 0.4062 \\ 
       $\ttbar \to 2\ell 2\nu 2b$       & TTTo2L2Nu\_13TeV-powheg & 831.76 $\times$ 0.1086 $\times$ 0.1086 $\times$ 9 \\ 
       $\tw$                            & ST\_tW\_top\_5f\_inclusiveDecays\_13TeV-powheg-pythia8 & 35.6 \\ 
       $\ensuremath{\bar{\mathrm{t}}\mathrm{W}}$          & ST\_tW\_antitop\_5f\_inclusiveDecays\_13TeV-powheg-pythia8 & 35.6 \\ 
       \hline 
       %\multicolumn{3}{c}{Signal} \\
       %\hline 
       \multirow{9}{*}{$\dyll$}
       %$\dyll$  \textsc{MadGraph5\_aMC@NLO} & DYJetsToLL\_M-50\_TuneCUETP8M1\_13TeV-amcatnloFXFX-pythia8   & 2008.4 $\times$ 3 \\ 
       %$\dyll$ \textsc{MadGraph5}          & DYJetsToLL\_M-50\_TuneCUETP8M1\_13TeV-madgraphMLM            & 2008.4 $\times$ 3 \\ 
       %$\dyll$ \textsc{powheg}             & ZToLL\_NNPDF30\_13TeV-powheg\_M\_50\_120                     & 1975.0 $\times$ 2 \\ 
       & DYJetsToLL\_M-50\_TuneCUETP8M1\_13TeV-amcatnloFXFX-pythia8         & 2008 $\times$ 3 \\ 
       & DYJetsToLL\_M-10to50\_TuneCUETP8M1\_13TeV-amcatnloFXFX-pythia8 & 2008.4 $\times$ 3 $\times$ 3.78 \\ 
       & DYJetsToLL\_M-50\_TuneCUETP8M1\_13TeV-madgraphMLM                  & 2008 $\times$ 3 \\ 
       & ZToLL\_NNPDF30\_13TeV-powheg\_M\_50\_120                           & 1975 $\times$ 2 \\ 
       & DYJetsToLL\_Pt-50To100\_TuneCUETP8M1\_13TeV-amcatnloFXFX-pythia8   & 375             \\ 
       & DYJetsToLL\_Pt-100To250\_TuneCUETP8M1\_13TeV-amcatnloFXFX-pythia8  & 86.5            \\ 
       & DYJetsToLL\_Pt-250To400\_TuneCUETP8M1\_13TeV-amcatnloFXFX-pythia8  & 3.32            \\ 
       & DYJetsToLL\_Pt-400To650\_TuneCUETP8M1\_13TeV-amcatnloFXFX-pythia8  & 0.449           \\ 
       & DYJetsToLL\_Pt-650ToInf\_TuneCUETP8M1\_13TeV-amcatnloFXFX-pythia8  & 0.0422          \\ 
       \hline 

         \end{tabular}
         }
           \end{center}
           \end{sidewaystable}

\subsection{Dark matter hypotheses}
Samples of simulated DM particle events in the simplified model are generated using \textsc{MadGraph5\_aMC@NLO 2.2.2}~\cite{Alwall:2014hca} at leading order (LO) and matched to
\PYTHIA8.205~\cite{Sjostrand:2007gs} using tune CUETP8M1 for parton showering and hadronization~\cite{Khachatryan:2015pea,Skands:2014pea}.
The factorization and re\-nor\-ma\-li\-zat\-ion scales are set to the geometric mean of $\sqrt{\pt^2+m^2}$ for all final-state particles~\cite{Alwall:2014hca,Abercrombie:2015wmb}, where $\pt$ and $m$ are the transverse momentum and mass of each particle.

For the simplified model of DM production, couplings are chosen according to the recommendations in Ref.~\cite{Boveia:2016mrp}.
The coupling $g_{\chi}$ is set to one. For $g_{\Pq}$, values of $1.0$ and $0.25$ are considered. The width of the mediator is assumed to be determined exclusively by the contributions from the couplings to quarks and the DM particle $\chi$. Under this assumption, the width ranges 1--5\% (30--50\%) of the mediator mass for $g_{\Pq}=0.25$ ($g_{\Pq}=1.00$).
The signal simulation samples with $g_{\Pq}=1.0$ are processed using the detector simulation described below.
Signal predictions for $g_{\Pq}=0.25$ are obtained by applying event weights based on the \ETm distribution at the generator level to the fully simulated samples with $g_{\Pq}=1.0$.
This procedure takes into account the nontrivial dependence of the mediator width on the coupling choice~\cite{Boveia:2016mrp}.  The exact dependence of the width on the model parameters is reported in~\cite{Boveia:2016mrp}.

Events for the ADD extra-dimension scenario are generated at LO using an effective field theory (EFT) implementation in \PYTHIA8~\cite{Ask:2008fh,Ask:2009pv}. Event samples are produced for $M_{D} = 1$, 2 and 3 \TeV, each with $n =$ 2, 3, 4, 5, 6, 7.
The signal is truncated for $\hat{s} > M_{D}^2$ in order to ensure the validity of the EFT.

The events for the unparticle model are generated at leading-order with \PYTHIA8~\cite{Ask:2008fh,Ask:2009pv}
assuming a cutoff scale $\Lambda_{\textsf{U}}=15\TeV$, 
using tune CUETP8M1 for parton showering and hadronization. 
We evaluate other values of $\Lambda_{\textsf{U}}$ by rescaling the cross sections as needed.
The parameter $\Lambda_{\textsf{U}}$ acts solely as a scaling factor
for the cross section and does not influence the kinematic distributions of unparticle production~\cite{Ask:2009pv}.

The $\Z\Hi$ production modes via $\Pq\Paq$ annihilation and gluon-gluon fusion are modeled the same way as for the $\W\Z$ and $\Z\Z$ processes.

A comparison of kinematic spectra for various signal models is shown in Figs.~\ref{fig:signals} and~\ref{fig:moreSignals}.

\section{Corrections to the simulated samples}
\subsection{Pileup reweighting}
\subsection{Lepton selection efficiencies}
\subsection{Lepton momentum scale and resolution}
