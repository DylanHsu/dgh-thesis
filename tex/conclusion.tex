\chapter{Conclusion and Outlook}

In this work, measurements of differential $\mathrm{Z}/\gamma^{*}$ production cross sections in $\mathrm{p}\mathrm{p}$ collisions at $\sqrt{s}=13~\mathrm{TeV}$ in the electron and muon final state
are presented.
The dataset was collected with the CMS Detector at the LHC, corresponding to an integrated luminosity of $35.9~\mathrm{fb}^{-1}$.
A total of $32.5 \times 10^{6}$ data events were selected, versus an expectation of $32.8 \times 10^{6}$.

The measured fiducial inclusive cross section times branching fraction is: 
\begin{equation*}
\sigma_{\PZ \to \ell\ell}(\sqrt{s}=13~\mathrm{TeV}) = 699\pm5\syst\pm17\lum~\mathrm{pb}
\end{equation*}
The measurement is limited by the systematic uncertainty of the integrated luminosity.
It agrees with next-to-next-to-leading-order QCD and next-to-leading cross section calculations. 
Distributions of the transverse momentum, the angular variable $\phi^{*}_\eta$, and the rapidity of lepton pairs are also measured and compared to theoretical predictions.
The prediction is consistent with the measurements within uncertainties.

The differential measurement will contribute to future global fits for the determination of the PDFs.In particular, the Drell-Yan process is most useful in constraining the light-quark PDFs.

In the near future, this measurement will be extended to the entire Run-II dataset,
representing an integrated luminosity of $150~\mathrm{fb}^{-1}$.
It will also be combined with the decays to hadrons and neutrinos.
This will enhance the precision at higher values of momentum approaching 1 TeV.

Using this dataset, a search for new physics in events with a
leptonically decaying $\PZ$ boson and large missing transverse momentum was performed.
Using the cut-based strategy, 688 data events were selected versus 692 expected.
Using the multivariate strategy, 1586 data events were selected versus 1640 expected. 
No evidence for physics beyond the standard model is found.
Compared to the previous search in the same final state~\cite{CMS-PAPER-EXO-16-010},
the exclusion limits on dark matter and mediator masses are significantly extended for spin-1 mediators in the simplified model interpretation, and exclusion limits for unparticles are also extended.
Results for dark matter production via spin-0 mediators in the simplified model interpretation,
as well as graviton emission in a model with large extra dimensions,
are presented in this final state for the first time.
\newpage
In the case of invisible decays of a standard-model-like Higgs boson,
the upper limit at 95\% confidence level on the production cross section times branching ratio is:
\begin{equation*}
\sigma_{qq\rightarrow\mathrm{ZH}} \times \mathrm{B(H \rightarrow invisible)} < 40\%
\end{equation*}
This is competitive with the contemporary result from our sister experiment ATLAS \cite{Aaboud:2017bja}.
Using the 2016 ATLAS dataset (integrated luminosity $36.1~\mathrm{fb}^{-1}$), 
the corresponding result for invisible Higgs boson decays is:
\begin{equation*}
\left[\sigma_{qq\rightarrow\mathrm{ZH}} \times \mathrm{B(H \rightarrow invisible)}\right]_\mathrm{ATLAS~2018} < 67\%
\end{equation*}
Additionally, constraints have been placed on other models.
For the most generic of models with a massive spin-1 mediator,
mediator masses up to order of 500 \GeV have been excluded in an electroweak scale coupling scenario.
When interpreted in the context of the dark matter-nucleon scattering cross section,
this result serves to complement the efforts of the direct detection dark matter experiments.
In the context of the ADD extra dimensions model, constraints are calculated on the true Planck scale of the $n+4$ dimensional spacetime $M_D$, as a function of the number of extra dimensions $n$.
Between two and seven extra dimensions, values of $M_D$ below approximately 2.3 TeV are excluded.

In the future, this search will be performed with the remainder of the Run-II CMS dataset.
It may also be combined with the other search channels to broaden the exclusion
of the various exotic models, similar to the existing preliminary combined analysis
in Ref.~\cite{CMS-PAPER-HIG-16-016}.


