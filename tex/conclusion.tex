\chapter{Conclusion and Outlook}

In this work, measurements of differential $\mathrm{Z}/\gamma^{*}$ production cross sections in $\mathrm{p}\mathrm{p}$ collisions at $\sqrt{s}=13~\mathrm{TeV}$ in the electron and muon final state
are presented.
The dataset was collected with the CMS Detector at the LHC, corresponding to an integrated luminosity of $35.9~\mathrm{fb}^{-1}$.
A total of $32.5 \times 10^{6}$ data events were selected.
The measured fiducial inclusive cross section times branching fraction is: 
\begin{equation*}
\sigma_{\PZ \to \ell^{+}\ell^{-}}(\sqrt{s}=13~\mathrm{TeV}) = 699\pm5\syst\pm17\lum~\mathrm{pb}
\end{equation*}
The measurement is limited by the systematic uncertainty of the integrated luminosity.
It agrees with next-to-next-to-leading-order QCD and next-to-leading order electroweak cross section calculations. 

Distributions of the transverse momentum, the angular variable $\phi^{*}_\eta$, and the rapidity of lepton pairs were also measured and compared to theoretical predictions.
The prediction is consistent with the measurements within uncertainties.
The differential measurement provides a useful benchmark for ongoing research in the realm of fixed-order calculations.
It will also serve as a constraint in future global fits for the determination of the PDFs.
In particular, the Drell-Yan process is most useful in constraining the light-quark PDFs.

In the near future, this measurement will be extended to the entire Run-II dataset,
representing an integrated luminosity of $150~\mathrm{fb}^{-1}$.
It will also be combined with the decays to hadrons and neutrinos.
This will enhance the precision at higher values of momentum approaching 1 TeV.

\bigskip\par\centerline{*\,*\,*}\medskip\par

Using the same dataset, a search for new physics in events with a
leptonically decaying $\PZ$ boson and large missing transverse momentum was performed.
Using the cut-based strategy, 688 data events were selected;
using the multivariate strategy, 1586 data events were selected.
No evidence for physics beyond the standard model was found.
Compared to the previous search in the same final state~\cite{CMS-PAPER-EXO-16-010},
the exclusion limits on dark matter and mediator masses were significantly extended for spin-1 mediators in the simplified model interpretation, and exclusion limits for unparticles were also extended.
Results for dark matter production via spin-0 mediators in the simplified model interpretation,
as well as graviton emission in a model with large extra dimensions,
have been presented in this final state for the first time.

In the case of invisible decays of a Standard Model-like Higgs boson,
we assume the Standard Model production cross section, and set an upper limit of 40\% on the invisible branching ratio (at 95\% confidence level).
%\begin{equation*}
%\sigma_{qq\rightarrow\mathrm{ZH}} \times \mathrm{B(H \rightarrow invisible)} < 40\%
%\end{equation*}
This is competitive with the contemporary result from our sister experiment ATLAS \cite{Aaboud:2017bja}.
Using the 2016 ATLAS dataset (integrated luminosity $36.1~\mathrm{fb}^{-1}$), 
the corresponding upper limit for invisible Higgs boson decays is 67\%.
%\begin{equation*}
%\left[\sigma_{qq\rightarrow\mathrm{ZH}} \times \mathrm{B(H \rightarrow invisible)}\right]_\mathrm{ATLAS~2018} < 67\%
%\end{equation*}

It is important to note that using the same 2016 CMS dataset,
another search for invisible Higgs boson decays in the vector boson fusion (VBF) production mode was performed~\cite{Sirunyan:2018owy}.
There, the participant quarks each emit virtual W or Z bosons which annihilate into a Higgs boson.
The corresponding upper limit is 33\%.
A preliminary combined result was prepared which included the result from this work, that VBF result, and other CMS results~\cite{Chatrchyan:2014tja} using the Run-I datasets.
That combination set a limit of 19\%, the most stringent to date.
By comparison, the Standard Model branching ratio of Higgs to invisible particles (neutrinos) is around 0.1\%.
Much work remains to be done in order to bridge this gap,
and the first step will be a combined analysis of the full Run-II CMS and ATLAS datasets.

Beyond the pursuit of the invisible Higgs, constraints have been placed on other models.
For the most generic of models with a massive spin-1 mediator,
mediator masses up to order of 500 \GeV have been excluded in an electroweak scale coupling scenario.
When interpreted in the context of the dark matter-nucleon scattering cross section,
this result serves to complement the efforts of the direct detection dark matter experiments.
In the context of the ADD extra dimensions model, constraints were calculated on the true Planck scale of the $n+4$ dimensional spacetime $M_D$, as a function of the number of extra dimensions $n$.
Between two and seven extra dimensions, values of $M_D$ below approximately 2.3 TeV were excluded.

In the future, this search will be performed with the remainder of the Run-II CMS dataset.
It may also be combined with the other search channels to broaden the exclusion
of the various exotic models, similar to the existing preliminary combined analysis
in Ref.~\cite{CMS-PAPER-HIG-16-016}.

\bigskip\par\centerline{*\,*\,*}\medskip\par

In closing, we are entering an era where the Standard Model is tested at high precision.
The work presented here grants us further insight into the internal structure of the proton, and later we hope to use it to nail down the mass of the $mathrm{W^\pm}$ boson.
Meanwhile, the community of theorists have an additional reference point for state-of-the-art quantum field theory calculations.
These, along with plans for future colliders, will usher in a new age of precise Higgs boson measurements.
As our accumulated knowledge of the fundamental interactions begins to crystallize, there is no guarantee we will observe new physics at the LHC or elsewhere.
But we will leave no stone unturned---and we are still on the hunt.
