\chapter{Diboson studies}
\label{chap:dibosons}

The resonant diboson processes $\Z\Z$ and $\W\Z$ are an irreducible background of the
dark matter searches involving any Z boson in the final state.
In final states with little hadronic activity e.g. this work, 
the other background processes which contribute substantially are easy to reject and pare down.
Then, the diboson becomes the main problem due to our limited theoretical understanding, and nothing beside remains.

Below, I survey the progress of theoretical calculations for dibosons. 
Then, I discuss a strategy which makes use of independent control samples from the experimental data,
to improve upon inaccuracies arising from those calculations.

%\feynmandiagram [horizontal=a to b] {
%i1 -- [fermion, thick] a -- [fermion, thick] i2,
%a -- [photon, thick] b,
%f1 -- [fermion, thick] b -- [fermion, thick] f2,
%};
%
%\feynmandiagram [horizontal=a to b] {
%i1 [particle=\(\Pq\)] -- [fermion, very thick] a -- [fermion, very thick] i2 [particle=\(\Paq\)],
%a -- [red, boson, edge label=\(\Z*\)] b,
%f1 [particle=\(\mu^{+}\)] -- [fermion, very thick] b -- [fermion, very thick] f2 [particle=\(\mu^{-}\)],
%};
\begin{tikzpicture}
\begin{feynman}
\vertex (a) {\(\mu^{-}\)};
\vertex [below right=of a] (b);
\vertex [above right=of b] (f1) {\(\nu_{\mu}\)};
%\vertex [below right=of b] (c);
%\vertex [above right=of c] (f2) {\(\overline \nu_{e}\)};
%\vertex [below right=of c] (f3) {\(e^{-}\)};
\diagram* {
(a) -- [fermion] (b) -- [fermion] (f1),
%(b) -- [boson, edge label'=\(W^{-}\)] (c),
%(c) -- [anti fermion] (f2),
%(c) -- [fermion] (f3),
};
\end{feynman}
\end{tikzpicture}
\section{Limitations of simulating diboson processes}
As previously outlined in \ref{sec:higher-order-corrections}, an attempt is made to obtain simulated samples for $\Z\Z$ and $\W\Z$ at NNLO in QCD and NLO in electroweak. 
\section{Three-lepton control sample}
\section{Four-lepton control sample}
\section{Emulation of the missing energy}

